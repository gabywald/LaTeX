\documentclass[11pt,twoside,a4paper]{article}
%=========================== En-Tete =================================
%--- Insertion de paquetages (optionnel) ---
\usepackage[french]{babel}   % pour dire que le texte est en fran{\'e}ais
\usepackage{a4}	             % pour la taille   
\usepackage[T1]{fontenc}     % pour les font postscript

\usepackage{epsfig}          % pour gerer les images
\usepackage{lastpage}

\usepackage{tikz}
\usetikzlibrary{decorations.pathmorphing, shapes}

%============================= Corps =================================
\begin{document}

\hrule ~\\ 

%% https://github.com/registor/Tikz
%% https://open-freax.fr/tuto-dessiner-en-latex-tikz/
%% https://blog.dorian-depriester.fr/latex/tikz/tikz-de-magnifiques-figures-directement-sous-latex
%% https://texample.net/tikz/examples/all/date/
%% https://fr.wikibooks.org/wiki/LaTeX/Dessiner_avec_LaTeX/Dessiner_avec_PGF_et_TikZ

\begin{minipage}[ht]{0.45\textwidth}
	\begin{tikzpicture}[xscale=1,yscale=1]
		\draw [<->] (0,1) -- (0,0) -- (1,0);
		\draw[blue,thick, domain=-2:2] plot (\x, {\x*\x});
	\end{tikzpicture}
\end{minipage} \hfill \begin{minipage}[ht]{0.45\textwidth}
	\begin{tikzpicture}
		\draw[red] (0,0) -- (1,2) -- (3,3) -- (2,0);
		\draw[help lines] (0,0) grid (3,3);
		%% \fill (0,0) -- (1,1) -- (0,1) -- (1,0);
	\end{tikzpicture}
\end{minipage} ~\\

\begin{minipage}[ht]{0.45\textwidth}
	\begin{tikzpicture}
		\fill[gray] (0,2) arc (-90:90:1);
	 	\fill[red] (0,-1) arc (0:90:1cm);
		\draw (0,-1) arc (-90:-270:1);
	\end{tikzpicture}
\end{minipage} \hfill \begin{minipage}[ht]{0.45\textwidth}
	\begin{tikzpicture}
		\fill[gray] (0,0) arc (-180:0:1cm);
		\fill[black] (0,0) arc (0:180:1cm);
	\end{tikzpicture}
	\begin{tikzpicture}
		\fill[gray] (0,0) arc (0:-180:1cm);
		\fill[black] (0,0) arc (0:180:1cm);
	\end{tikzpicture}
\end{minipage} ~\\

~\\ \hrule~\\

\begin{tikzpicture}[decoration={coil},dna/.style={decorate, thick, decoration={aspect=0, segment length=0.5cm}}]
	%DNA
	\draw[dna, decoration={amplitude=.15cm}] (.1,0) -- (11,0);
	\draw[dna, decoration={amplitude=-.15cm}] (0,0) -- (11,0);
	\node at (0,0.5) {DNA};
\end{tikzpicture}

~\\ \hrule~\\

%% https://texblog.org/2014/04/28/simple-dna-protein-interaction-model-with-tikz/
\begin{tikzpicture}[decoration={coil},
dna/.style={decorate, thick, decoration={aspect=0, segment length=0.5cm}},
protein/.style={ellipse, draw=white, minimum width=1cm, minimum height=1cm}]
	%DNA
	\draw[dna, decoration={amplitude=.15cm}] (.1,0) -- (11,0);
	\draw[dna, decoration={amplitude=-.15cm}] (0,0) -- (10.8,0);
	\node at (0,0.5) {DNA};

	%Gene
	\node [rounded corners, fill=green!50, thick, inner xsep=30pt] at (8,0) (box){Gene X};

	%Promoter proteins
	\draw (4.25,0.625) node[protein,minimum width=1.5cm,fill=red!30] {p123};
	\node [protein, minimum height=.75cm,fill=blue!30] at (3,1.5) {p12};
	\node [protein, minimum height=.75cm,fill=blue!30] at (2.5,1) {p12};
	\node [protein, minimum height=.75cm,fill=blue!30] at (3.5,1) {p12};
	\node [protein, minimum height=.75cm,fill=blue!30] at (3,0.5) {p12};
	\draw (4,1.75) node[protein,fill=green!30] {p34};
	\node at (3.5, -0.5) {Promoter};
\end{tikzpicture}

~\\ \hrule~\\

%% https://tex.stackexchange.com/questions/520699/drawing-simple-dna-helix
\begin{tikzpicture}
    \begin{scope}[scale=.3, domain=0:15, samples=101]
        \foreach \x in {0,1, ...,14}{
          \draw[thick, gray] (\x,{-sin(\x*36)}) -- (\x,{sin(\x*36)});
        }
        \draw[gray, very thick] plot (\x,{sin(\x*36)});
        \draw[gray, very thick] plot (\x,{-sin(\x*36)});
    \end{scope}
\end{tikzpicture}

%% https://shantoroy.com/latex/taxonomy-tree-in-latex-for-publication/

\clearpage

\begin{minipage}[ht]{0.45\textwidth}
	\colorbox{purple!30}{
	\begin{tikzpicture}
		\draw[fill, gray] (0,0) arc (0:-180:1cm);
		\draw[fill, black] (0,0) arc (0:180:1cm);
		
		\draw[fill, red] 	(-1.25,-0.25) rectangle (-1.60,-5);
		\draw[fill, green] 	(-0.75,-0.25) rectangle (-1.10,-6);
		\draw[fill, blue] 	(-0.25,-0.25) rectangle (-0.60,-7);
	\end{tikzpicture}
	}
\end{minipage}

\end{document}
