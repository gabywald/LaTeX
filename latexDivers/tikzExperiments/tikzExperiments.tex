\documentclass[11pt,twoside,a4paper]{article}
%=========================== En-Tete =================================
%--- Insertion de paquetages (optionnel) ---
\usepackage[french]{babel}   % pour dire que le texte est en fran{\'e}ais
\usepackage{a4}	             % pour la taille   
\usepackage[T1]{fontenc}     % pour les font postscript

\usepackage{epsfig}          % pour gerer les images
\usepackage{lastpage}

\usepackage{tikz}
\usetikzlibrary{decorations.pathmorphing, shapes}
\usetikzlibrary{shapes.geometric, intersections, calc}
\usetikzlibrary{decorations.markings, arrows}

\usepackage{fdsymbol}
%% \clubsuit
%% \vardiamondsuit
%% \spadesuit
%% \varheartsuit
\DeclareSymbolFont{extraup}{U}{zavm}{m}{n}
\DeclareMathSymbol{\varheart}{\mathalpha}{extraup}{86}
\DeclareMathSymbol{\vardiamond}{\mathalpha}{extraup}{87}

% https://tex.stackexchange.com/questions/100928/how-to-make-a-cog-gear-in-latex
% #1 number of teeths
% #2 radius intern
% #3 radius extern
% #4 angle from start to end of the first arc
% #5 angle to decale the second arc from the first 
\newcommand{\gearmacro}[5]{%
	\foreach \i in {1,...,#1} {%
	  [rotate=(\i-1)*360/#1]  (0:#2)  arc (0:#4:#2) {[rounded corners=1.5pt]
	             -- (#4+#5:#3)  arc (#4+#5:360/#1-#5:#3)} --  (360/#1:#2)
}}

% #1 number of teeths
% #2 radius intern
% #3 radius extern
% #4 angle from start to end of the first arc
% #5 angle to decale the second arc from the first
% #6 Inner fill color 
% #7 Border color 
\newcommand{\mechanicalgear}[7]{%
	\fill[#6] (0,0) circle (#2);
	\filldraw[ultra thick,rotate=12,fill=#6,draw=#7] \gearmacro{#1}{#2}{#3}{#4}{#5};
	\filldraw[ultra thick,fill=white,draw=#7] (0,0) circle(#2 / 1.41421356237); %% sqrt(2)
}

% #1 number of teeths
% #2 radius intern
% #3 radius extern
% #4 angle from start to end of the first arc
% #5 angle to decale the second arc from the first
% #6 Inner fill color 
% #7 Border color 
\newcommand{\mechanicalgearINtikz}[7]{%
	\begin{tikzpicture}
		\fill[#6] (0,0) circle (#2);
		\filldraw[ultra thick,rotate=12,fill=#6,draw=#7] \gearmacro{#1}{#2}{#3}{#4}{#5};
		\filldraw[ultra thick,fill=white,draw=#7] (0,0) circle(#2 / 1.41421356237); %% sqrt(2)
	\end{tikzpicture}
}



%============================= Corps =================================
\begin{document}

\hrule ~\\ 

%% https://github.com/registor/Tikz
%% https://open-freax.fr/tuto-dessiner-en-latex-tikz/
%% https://blog.dorian-depriester.fr/latex/tikz/tikz-de-magnifiques-figures-directement-sous-latex
%% https://texample.net/tikz/examples/all/date/
%% https://fr.wikibooks.org/wiki/LaTeX/Dessiner_avec_LaTeX/Dessiner_avec_PGF_et_TikZ

\begin{minipage}[ht]{0.45\textwidth}
	\begin{tikzpicture}[xscale=1,yscale=1]
		\draw [<->] (0,1) -- (0,0) -- (1,0);
		\draw[blue,thick, domain=-2:2] plot (\x, {\x*\x});
	\end{tikzpicture}
\end{minipage} \hfill \begin{minipage}[ht]{0.45\textwidth}
	\begin{tikzpicture}
		\draw[red] (0,0) -- (1,2) -- (3,3) -- (2,0);
		\draw[help lines] (0,0) grid (3,3);
		%% \fill (0,0) -- (1,1) -- (0,1) -- (1,0);
	\end{tikzpicture}
\end{minipage} ~\\

\begin{minipage}[ht]{0.45\textwidth}
	\begin{tikzpicture}
		\fill[gray] (0, 2) arc (-90:90:1);
	 	\fill[red]  (0,-1) arc (0:90:1cm);
		\draw (0,-1) arc (-90:-270:1);
	\end{tikzpicture}
\end{minipage} \hfill \begin{minipage}[ht]{0.45\textwidth}
	\begin{tikzpicture}
		\fill[gray]  (0,0) arc (-180:0:1cm);
		\fill[black] (0,0) arc (0:180:1cm);
	\end{tikzpicture}
	\begin{tikzpicture}
		\fill[gray]  (0,0) arc (0:-180:1cm);
		\fill[black] (0,0) arc (0:180:1cm);
	\end{tikzpicture}
\end{minipage} ~\\

~\\ \hrule~\\

\begin{tikzpicture}[decoration={coil},dna/.style={decorate, thick, decoration={aspect=0, segment length=0.5cm}}]
	%DNA
	\draw[dna, decoration={amplitude=.15cm}] (.1,0) -- (11,0);
	\draw[dna, decoration={amplitude=-.15cm}] (0,0) -- (11,0);
	\node at (0,0.5) {DNA};
\end{tikzpicture}

~\\ \hrule~\\

%% https://texblog.org/2014/04/28/simple-dna-protein-interaction-model-with-tikz/
\begin{tikzpicture}[decoration={coil},
dna/.style={decorate, thick, decoration={aspect=0, segment length=0.5cm}},
protein/.style={ellipse, draw=white, minimum width=1cm, minimum height=1cm}]
	%DNA
	\draw[dna, decoration={amplitude=.15cm}] (.1,0) -- (11,0);
	\draw[dna, decoration={amplitude=-.15cm}] (0,0) -- (10.8,0);
	\node at (0,0.5) {DNA};

	%Gene
	\node [rounded corners, fill=green!50, thick, inner xsep=30pt] at (8,0) (box){Gene X};

	%Promoter proteins
	\draw (4.25,0.625) node[protein,minimum width=1.5cm,fill=red!30] {p123};
	\node [protein, minimum height=.75cm,fill=blue!30] at (3,1.5) {p12};
	\node [protein, minimum height=.75cm,fill=blue!30] at (2.5,1) {p12};
	\node [protein, minimum height=.75cm,fill=blue!30] at (3.5,1) {p12};
	\node [protein, minimum height=.75cm,fill=blue!30] at (3,0.5) {p12};
	\draw (4,1.75) node[protein,fill=green!30] {p34};
	\node at (3.5, -0.5) {Promoter};
\end{tikzpicture}

~\\ \hrule~\\

	%% https://tex.stackexchange.com/questions/520699/drawing-simple-dna-helix
	\begin{tikzpicture}
	    \begin{scope}[scale=.3, domain=0:15, samples=101]
	        \foreach \x in {0,1, ...,14}{
	          \draw[thick, gray] (\x,{-sin(\x*36)}) -- (\x,{sin(\x*36)});
	        }
	        \draw[gray, very thick] plot (\x,{sin(\x*36)});
	        \draw[gray, very thick] plot (\x,{-sin(\x*36)});
	    \end{scope}
	\end{tikzpicture}
	
	%% https://shantoroy.com/latex/taxonomy-tree-in-latex-for-publication/
	
\clearpage

	\begin{minipage}[b]{0.5\linewidth}
		\begin{tikzpicture}
			\fill[gray] (0,0) circle (2cm);
			\draw[thick,rotate=12,fill=gray] \gearmacro{8}{2}{2.4}{20}{2};
			\draw[thick,fill=white] (0cm,0cm) circle(1.35cm);
		\end{tikzpicture}
	\end{minipage}\hfill\begin{minipage}[b]{0.5\linewidth}
		% \mechanicalgear{8}{2}{2.4}{20}{2};
		\begin{tikzpicture}
			\fill[gray] (0,0) circle (2cm);
			\draw[thick,rotate=12,fill=gray] \gearmacro{8}{2}{2.4}{20}{2};
			\draw[thick,fill=white] (0cm,0cm) circle(2 / 1.41421356237); %% sqrt(2)
		\end{tikzpicture}
	\end{minipage}~\\
	
	\begin{minipage}[b]{0.5\linewidth}
		\mechanicalgearINtikz{8}{2}{2.4}{20}{2}{magenta}{cyan}
	\end{minipage}\hfill\begin{minipage}[b]{0.5\linewidth}
		\mechanicalgearINtikz{8}{2}{2.4}{20}{10}{blue}{black}
	\end{minipage}~\\
	
	\begin{minipage}[b]{0.5\linewidth}
		\mechanicalgearINtikz{8}{2}{2.4}{20}{2}{magenta}{yellow}
	\end{minipage}\hfill\begin{minipage}[b]{0.5\linewidth}
		\mechanicalgearINtikz{8}{2}{2.4}{20}{10}{blue}{red}
	\end{minipage}~\\
	
	\begin{minipage}[b]{0.5\linewidth}
		\mechanicalgearINtikz{8}{2}{2.4}{20}{2}{white}{blue}
	\end{minipage}\hfill\begin{minipage}[b]{0.5\linewidth}
		\mechanicalgearINtikz{8}{2}{2.4}{20}{10}{white}{black}
	\end{minipage}~\\
	
\clearpage

\begin{minipage}[ht]{0.45\textwidth}
	\mechanicalgearINtikz{8}{2}{2.4}{20}{2}{white}{black}
\end{minipage} \hfill \begin{minipage}[ht]{0.45\textwidth}
	\begin{tikzpicture}
		\draw [very thin, gray] (0,0) grid (5,5);
		
		%% DataBase
		\draw[ultra thick, black, fill=white] (2, 4) arc (0:180:1.0cm and 0.5cm);
		\foreach \y in {1, 2, 3, 4} { \draw[ultra thick, black, fill=white] (2, \y) arc (0:-180:1.0cm and 0.5cm); }
		\draw[ultra thick, black] (0,1) -- (0,4) ;
		\draw[ultra thick, black] (2,1) -- (2,4) ;
		
		\foreach \y in {1, 2, 3} { \draw[ultra thick, black]  (0.20, \y+0.20) arc (-180:-160:1.0cm and 0.5cm) ; }
		
		%% Circuitry Lines and Dots
		\draw[ultra thick, black] (2.5,0.5) -- (2.5,4.5) ;
		
		\draw[ultra thick, black] (2.5,0.5) -- (3.5,0.5) ;
		\draw[ultra thick, black] (2.5,4.5) -- (3.5,4.5) ;
		\draw[ultra thick, black] (2.5,1.5) -- (4.8,1.5) ;
		\draw[ultra thick, black] (2.5,3.5) -- (4.8,3.5) ;
		\draw[ultra thick, black] (2.5,2.5) -- (4.0,2.5) ;
		
		\draw [very thick, black, fill=white] (3.5, 0.5) circle (0.1);
		\draw [very thick, black, fill=white] (3.5, 4.5) circle (0.1);
		\draw [very thick, black, fill=white] (4.8, 1.5) circle (0.1);
		\draw [very thick, black, fill=white] (4.8, 3.5) circle (0.1);
		\draw [very thick, black, fill=white] (4.0, 2.5) circle (0.1);
		
		%% Dental Gear !
		\begin{scope}[xshift=2.00cm, yshift=2.50cm]
			\mechanicalgear{8}{0.80}{1.00}{20}{2}{white}{black}
		\end{scope}
	\end{tikzpicture}
\end{minipage} ~\\

\clearpage

\begin{minipage}[ht]{0.45\textwidth}
	\colorbox{purple!30}{
	\begin{tikzpicture}
		\draw[fill, gray]  (0,0) arc (0:-180:1cm);
		\draw[fill, black] (0,0) arc (0: 180:1cm);
		
		\draw[fill, red] 	(-1.25,-0.25) rectangle (-1.60,-5);
		\draw[fill, green] 	(-0.75,-0.25) rectangle (-1.10,-6);
		\draw[fill, blue] 	(-0.25,-0.25) rectangle (-0.60,-7);
	\end{tikzpicture}
	}
\end{minipage} \hfill \begin{minipage}[ht]{0.45\textwidth}
	\begin{tikzpicture}
	
		\draw[fill, red]    ( 0.0, 0.0) -- ( 1.5, 0.0) arc[start angle=  0, end angle= 90,radius=1.5cm] -- ( 0.0, 0.0);
		\draw[fill, green]  ( 0.0, 0.0) -- ( 0.0, 1.5) arc[start angle= 90, end angle=180,radius=1.5cm] -- ( 0.0, 0.0);
		\draw[fill, blue]   ( 0.0, 0.0) -- (-0.0,-1.5) arc[start angle=270, end angle=360,radius=1.5cm] -- ( 0.0, 0.0);
		\draw[fill, yellow] ( 0.0, 0.0) -- (-1.5,-0.0) arc[start angle=180, end angle=270,radius=1.5cm] -- ( 0.0, 0.0);
		
		\draw [->] (-0.0,-2.0) -- ( 0.0, 2.0);
		\draw [->] (-2.0,-0.0) -- ( 2.0, 0.0);
		
		\node at (-1.0,  2.00) {Promoter1};
		\node at ( 1.0,  2.00) {Promoter2};
		\node at (-1.0, -2.00) {Promoter3};
		\node at ( 1.0, -2.00) {Promoter4};
		
		%% \node at ( 3.5, -1.25) {Promoter5};
		%% \node at ( 3.5,  1.25) {Promoter6};
		\node[rotate= 23] at ( 3.5,  -1.25) {Promoter5};
		\node[rotate=-23] at ( 3.5,   1.25) {Promoter6};
		
		\node[fill=yellow, draw=black, text=blue, rotate=90] at (-3.0,  1.0) {Promoter7};
		\node[fill=green , draw=black, text=red , rotate=90] at (-3.0, -1.0) {Promoter8};
		
	\end{tikzpicture}
\end{minipage} ~\\


 \begin{minipage}[ht]{0.45\textwidth}
	\colorbox{purple!30}{
	\begin{tikzpicture}
		\draw[fill, gray]  (0,0) arc (0:-180:1cm);
		\draw[fill, black] (0,0) arc (0: 180:1cm);
		
		\draw[fill, red] 	(-1.25,-0.25) rectangle (-1.60,-5);
		\draw[fill, green] 	(-0.75,-0.25) rectangle (-1.10,-6);
		\draw[fill, blue] 	(-0.25,-0.25) rectangle (-0.60,-7);
	\end{tikzpicture}
	}
\end{minipage} \hfill \begin{minipage}[ht]{0.45\textwidth}
	\begin{tikzpicture}
		\draw[fill=yellow!30] (0,0) -- (2.0, 0.0) arc[start angle=0, end angle=60,radius=2cm] -- (0,0);
		\draw[fill=cyan!30]   (0,0) -- (0.0, 1.5) arc[start angle=90, delta angle=30, radius=1.5cm] -- (0,0);
		\draw[-latex] (120:0.5) arc (120:360:0.5) ;
	\end{tikzpicture}~\\

	\begin{tikzpicture}
		\draw[fill, red]    (0,0) arc (  0: 90:1.0cm);
		\draw[fill, green]  (0,0) arc ( 90:180:1.0cm);
		\draw[fill, blue]   (0,0) arc (180:270:1.0cm);
		\draw[fill, yellow] (0,0) arc (270:360:1.0cm);
	\end{tikzpicture} ~\\
\end{minipage} ~\\

\begin{minipage}[ht]{0.45\textwidth}
	\begin{tikzpicture}
		\draw [->] (-0.0,-3.0) -- ( 0.0, 3.0);
		\draw [->] (-3.0,-0.0) -- ( 3.0, 0.0);
	
		\draw[fill, red]    ( 0.0, 0.0) -- ( 2.0, 0.0) arc[start angle=  0, delta angle= 90,radius=2.0cm] -- ( 0.0, 0.0);
		\draw[fill, green]  ( 0.0, 0.0) -- ( 2.0, 2.0) arc[start angle= 90, delta angle= 90,radius=2.0cm] -- ( 1.5, 1.5);
		\draw[fill, blue]   ( 0.0, 0.0) -- ( 0.0, 2.0) arc[start angle=180, delta angle= 90,radius=2.0cm] -- ( 1.0, 1.0);
		\draw[fill, yellow] ( 0.0, 0.0) -- ( 0.0, 0.0) arc[start angle=270, delta angle= 90,radius=2.0cm] -- ( 2.0, 2.0);
	\end{tikzpicture}
\end{minipage} \hfill \begin{minipage}[ht]{0.45\textwidth}
	\begin{tikzpicture}
		\draw [->] (-0.0,-3.0) -- ( 0.0, 3.0);
		\draw [->] (-3.0,-0.0) -- ( 3.0, 0.0);
	
		\draw[fill, red]    ( 0.0, 0.0) -- ( 2.0, 0.0) arc[start angle=  0, end angle= 90,radius=2.0cm] -- ( 0.0, 0.0);
		\draw[fill, green]  ( 2.0, 0.0) -- ( 2.0, 2.0) arc[start angle= 90, end angle=180,radius=2.0cm] -- ( 1.5, 1.5);
		\draw[fill, blue]   ( 2.0, 2.0) -- ( 0.0, 2.0) arc[start angle=180, end angle=270,radius=2.0cm] -- ( 1.0, 1.0);
		\draw[fill, yellow] ( 0.0, 2.0) -- ( 0.0, 0.0) arc[start angle=270, end angle=360,radius=2.0cm] -- ( 2.0, 2.0);
	\end{tikzpicture}
\end{minipage} ~\\

\begin{minipage}[ht]{0.45\textwidth}
	\begin{tikzpicture}
		\draw [->] (-0.0,-3.0) -- ( 0.0, 3.0);
		\draw [->] (-3.0,-0.0) -- ( 3.0, 0.0);
	
		\draw[fill, red]    ( 0.0, 0.0) -- ( 2.0, 0.0) arc[start angle=  0, end angle= 90,radius=2.0cm] -- ( 0.0, 0.0);
		\draw[fill, green]  ( 0.0, 0.0) -- ( 0.0, 2.0) arc[start angle= 90, end angle=180,radius=2.0cm] -- ( 0.0, 0.0);
		\draw[fill, blue]   ( 0.0, 0.0) -- ( 2.0, 2.0) arc[start angle=180, end angle=270,radius=2.0cm] -- ( 0.0, 0.0);
		\draw[fill, yellow] ( 0.0, 0.0) -- ( 0.0, 0.0) arc[start angle=270, end angle=360,radius=2.0cm] -- ( 0.0, 0.0);
	\end{tikzpicture}
\end{minipage} \hfill \begin{minipage}[ht]{0.45\textwidth}
	\begin{tikzpicture}
		\draw [->] (-0.0,-3.0) -- ( 0.0, 3.0);
		\draw [->] (-3.0,-0.0) -- ( 3.0, 0.0);
	
		\draw[fill, red]    ( 0.0, 0.0) -- ( 2.0, 0.0) arc[start angle=  0, end angle= 90,radius=2.0cm] -- ( 0.0, 0.0);
		\draw[fill, green]  ( 0.0, 0.0) -- ( 0.0, 2.0) arc[start angle= 90, end angle=180,radius=2.0cm] -- ( 0.0, 0.0);
		\draw[fill, blue]   ( 0.0, 0.0) -- (-2.0,-0.0) arc[start angle=180, end angle=270,radius=2.0cm] -- ( 0.0, 0.0);
		\draw[fill, yellow] ( 0.0, 0.0) -- (-0.0,-2.0) arc[start angle=270, end angle=360,radius=2.0cm] -- ( 0.0, 0.0);
	\end{tikzpicture}
\end{minipage} ~\\

\begin{minipage}[ht]{0.45\textwidth}
	%% color 		white, black, red, green, blue, cyan, magenta, yellow 
	%% thickness 	ultra thin, very thin, thin, thick, very thick, ultra thick 
	\begin{tikzpicture}
		\draw [very thin, gray] (0,0) grid (6,11);
		
		\draw [very thick, blue] (5.75,10.75) -- (5.75,3) -- (5.07,1.07);
		\draw [very thick, blue] (5.75, 3)    -- (5.07,2.07);
		\draw [very thick, blue] (5.75, 3)    -- (5.07,3.00);
		\draw [very thick, blue] (5   , 1) circle (0.1);
		\draw [very thick, blue] (5   , 2) circle (0.1);
		\draw [very thick, blue] (5   , 3) circle (0.1);
		% \draw [very thin, cyan]    (5.75,10.75) circle (0.1);
		% \draw [very thin, cyan]    (5.75, 3) circle (0.1);
		% \draw [very thin, cyan]    (5.10, 1.10) circle (0.1);
		\draw [thick, blue] (5.65,10.75) -- (5.65,3.5);
		\draw [blue]        (5.55,10.75) -- (5.55,4.0);
		
		%% above, below, right, left,
		%% above left, above right, below left, below right
		\draw ( 1.00, 0.50) node[right]{$0/I - Cartouche Bas$} ;
		\draw ( 1.00,10.50) node[right]{$0/I - Cartouche Haut$} ;
		%% \draw (1,0) node[right]{$A$} ;
		%% \draw (0,1) node[above]{$B$} ;
		
	\end{tikzpicture}~\\
\end{minipage} \hfill \begin{minipage}[ht]{0.45\textwidth}
	\def\triangle{--++(120:1)--++(240:1)--cycle}
	\def\triangleReverse{--++(240:1)--cycle}
	\begin{tikzpicture}
		\draw [very thin, gray] (0,0) grid (8,1);
		
		\draw[red,thick](1,0)\triangle; %% FEU
		
		\draw[blue,thick](3,0)\triangle (2,0.40)--(3,0.40); %% AIR
		
		\draw[orange,thick](4,1)--(5,1)\triangleReverse; %% EAU
		
		\draw[green,thick](6,1)--(7,1)\triangleReverse (6,0.60)--(7,0.60); %% TERRE
	\end{tikzpicture}~\\
	
	$\heartsuit \varheart \diamondsuit \vardiamond \clubsuit \spadesuit$ ~\\
	
	\begin{tikzpicture}
		\draw [very thin, gray] (0,0) grid (8,1);
		
		%% COEUR
		\draw[fill,red,thick] (0.500, 0.700) arc (000:180:0.270);
		\draw[fill,red,thick] (1.000, 0.700) arc (000:180:0.270);
		\draw[fill,red,thick] (0.000, 0.700) -- (0.500, 0.000) -- (1.000, 0.700);
		
		%% COEUR
		%% \draw[fill,red,thick] (2.500, 0.700) arc (000:180:0.270);
		%% \draw[fill,red,thick] (3.000, 0.700) arc (000:180:0.270);
		%% \draw[fill,red,thick] (2.500, 0.000) arc (000:045:1.000);

	\end{tikzpicture}~\\
	
	\begin{tikzpicture}
		\draw [very thin, gray] (0,0) grid (8,1);
	
		%% CARREAU
		\draw[fill, red,thick] (0.000, 0.500) -- (0.500, 1.000) -- (1.000, 0.500) -- (0.500, 0.000) -- cycle;

	\end{tikzpicture}~\\
	
	\begin{tikzpicture}
		\draw [very thin, gray] (0,0) grid (8,1);
		
		%% TREFLE
		\draw[fill, black] (0.470, 0.000) rectangle (0.520, 0.500);
		\draw[fill, black] (0.250, 0.500) circle (0.225);
		\draw[fill, black] (0.500, 0.750) circle (0.225);
		\draw[fill, black] (0.750, 0.500) circle (0.225);
		
		%% PIQUE
	\end{tikzpicture}~\\
	
	\begin{tikzpicture}
		\draw [very thin, gray] (0,0) grid (8,1);
		
		%% PIQUE
		\draw[fill, black] (0.470, 0.000) rectangle (0.520, 0.500);
		\draw[fill, black,thick] (0.500, 0.350) arc (000:-180:0.270);
		\draw[fill, black,thick] (1.000, 0.350) arc (000:-180:0.270);
		\draw[fill, black,thick] (0.000, 0.350) -- (0.500, 1.000) -- (1.000, 0.350);
	\end{tikzpicture}~\\
\end{minipage} ~\\ 


%% https://tex.stackexchange.com/questions/45848/rotate-node-text-and-use-relative-positioning-in-tikz
%% \begin{tikzpicture}
%%   \node [draw] (first) {1};
%%   \foreach \o [count=\oi from 0] in {0.1, 0.15, 0.3, 0.5, 1} {%
%%     \pgfmathtruncatemacro{\angle}{90 * \oi/4}
%%     \node [draw, right=of first, opacity=\o, rotate=\angle, anchor=north] (second\oi) {2};
%%     \coordinate (second_rotated_center) at (second\oi.center);
%%   }
%%   \begin{scope}[red, very thin]
%%     \node at (second0.north) {
%%       \begin{tikzpicture}[scale=0.02, very thin]
%%         \draw (-1, -1) -- (1,  1);
%%         \draw (-1,  1) -- (1, -1);
%%       \end{tikzpicture}
%%     };
%%     \begin{scope}[dash pattern=on 0.1mm off 0.1mm]
%%       \draw (second0.north) -- (second0.center);
%%       \draw (second0.north) -- (second_rotated_center);
%%     \end{scope}
%%     \draw ([yshift=-0.5mm] second0.north)
%%       -- ([yshift=-0.5mm, xshift=0.5mm] second0.north)
%%       -- ([xshift=0.5mm] second0.north);
%%     \draw [>=stealth,
%%            decoration={markings, mark=at position 1 with
%%              {\arrow[scale=0.5]{>}}}, 
%%            postaction={decorate}]
%%       (second0.center) to [bend right=45] (second_rotated_center);
%%   \end{scope}
%% \end{tikzpicture}~\\

\clearpage

	%% https://tex.stackexchange.com/questions/497485/create-hexagon-style
	\begin{tikzpicture}[]
		\def\rot{-60}    
		\node[regular polygon, regular polygon sides=6, shape aspect=0.5, minimum width=4cm, minimum height=1cm, draw,shape border rotate=\rot] (reg) {};
		
		\path [draw,name path=beam1] ($(reg.corner 3)!0.65!(reg.corner 4)$)coordinate(A)--($(reg.corner 5)!0.35!(reg.corner 6)$)coordinate(B);
		\begin{scope}[rotate=\rot]
			\path [name path=beam2](reg.side 4)--++(90:2);
			\path [name intersections={of=beam1 and beam2,by={x}}];
			\draw (reg.side 4)--(x);
			\filldraw [opacity=0.3,cyan](A)--(B)--(reg.corner 5)--(reg.corner 4)--cycle;
			\node[rotate=\rot] at (x)[anchor=north west]{label1};
			\node[rotate=\rot] at (x)[anchor=north east]{label2};
		\end{scope}
	\end{tikzpicture}
	~\\

~\\ %% \begin{minipage}[ht]{0.45\textwidth}
	%% https://tex.stackexchange.com/questions/345978/drawing-a-regular-hexagon
	\begin{tikzpicture}
	  [thick,decoration =
	    {markings,mark=at position 0.5 with {\arrow{stealth}}} ]
	  \newdimen\R
	  \R=2.5cm
	  \node {X};
	  \draw[-stealth'] (-150:{0.7*\R}) arc (-150:150:{0.7*\R});
	  \foreach \x/\l in
	    { 
	     60/a,
	     120/b,
	     180/c,
	     240/d,
	     300/e,
	     360/f
	    }
	    \draw[postaction={decorate}] ({\x-60}:\R) -- node[auto,swap]{\l} (\x:\R);
	  \foreach \x/\l/\p in
	    { 
	     60/{(2,3,1)}/above,
	     120/{(2,1,3)}/above,
	     180/{(1,2,3)}/left,
	     240/{(1,3,2)}/below,
	     300/{(3,1,2)}/below,
	     360/{(3,2,1)}/right
	    }
	    \node[inner sep=2pt,circle,draw,fill,label={\p:\l}] at (\x:\R) {};
	\end{tikzpicture}
~\\ %% \end{minipage} \hfill \begin{minipage}[ht]{0.45\textwidth}
	%% https://tex.stackexchange.com/questions/345978/drawing-a-regular-hexagon
	\begin{tikzpicture}
	   \newdimen\R
	   \R=2.5cm
	   \draw (0:\R) \foreach \x in {60,120,...,360} {  -- (\x:\R) };
	   \foreach \x/\l/\p in
	     { 
	      60/{(2,3,1)}/above,
	      120/{(2,1,3)}/above,
	      180/{(1,2,3)}/left,
	      240/{(1,3,2)}/below,
	      300/{(3,1,2)}/below,
	      360/{(3,2,1)}/right
	     }
	     \node[inner sep=1pt,circle,draw,fill,label={\p:\l}] at (\x:\R) {};
	\end{tikzpicture}
~\\ %% \end{minipage} 

\clearpage

%% https://tex.stackexchange.com/questions/474155/a0-hexagon-background-tikz
\def\hexapavage{--++(60:1)--+(120:1)++(0,0)--++(1,0)--++(60:1)--+(1,0)++(-120:1)--++(-60:1)}
\tikz\draw[blue](0,0)\hexapavage;

\begin{tikzpicture}
	%% Pavage hexagonal
	\foreach \j in {0,1,...,4} {
		\foreach \i in {0,1,...,4} {
			\draw[thick] (60:\j)++(120:\j)++
			(60:\i)++(-60:\i)++(\i,0)++(\i,0) \hexapavage ;
	}}
	\draw[blue,very thick](0,0)\hexapavage;
	
	% Pavage avec scale
	\begin{scope}[yshift=-6cm,scale=.6]
		 \foreach \j in {0,1,...,4} {
			\foreach \i in {0,1,...,4} {
				\draw[thick] (60:\j)++(120:\j)++
				(60:\i)++(-60:\i)++(\i,0)++(\i,0) \hexapavage ;
	}}
	\draw[blue,very thick](0,0)\hexapavage;
	\end{scope}
\end{tikzpicture} 


\end{document}
