\documentclass[11pt,twoside,a4paper]{book}
% http://www-h.eng.cam.ac.uk/help/tpl/textprocessing/latex_maths+pix/node6.html symboles de math
% http://fr.wikibooks.org/wiki/Programmation_LaTeX Programmation latex (wikibook)
%=========================== En-Tete =================================
%--- Insertion de paquetages (optionnel) ---
\usepackage[english]{babel}  % pour dire que le texte est en fran{\'e}ais
\usepackage{a4}	       % pour la taille  
\usepackage[T1]{fontenc}   % pour les font postscript
\usepackage{epsfig}     % pour gerer les images
%\usepackage{psfig}
\usepackage{amsmath, amsthm} % tres bon mode mathematique
\usepackage{amsfonts,amssymb}% permet la definition des ensembles
\usepackage{float}      % pour le placement des figure
\usepackage{verbatim}

\usepackage{longtable} % pour les tableaux de plusieurs pages

\usepackage[table]{xcolor} % couleur de fond des cellules de tableaux

\usepackage{gensymb}

\usepackage{lastpage}

\usepackage{multirow}

\usepackage{multicol} % pour {\'e}crire dans certaines zones en colonnes : \begin{multicols}{nb colonnes}...\end{multicols} 

% \usepackage[top=1.5cm, bottom=1.5cm, left=1.5cm, right=1.5cm]{geometry}
% gauche, haut, droite, bas, entete, ente2txt, pied, txt2pied
\usepackage{vmargin}
\setmarginsrb{1.00cm}{1.00cm}{1.00cm}{1.00cm}{15pt}{3pt}{45pt}{15pt}

\usepackage{lscape} % changement orientation page
%\usepackage{frbib} % enlever pour obtenir references en anglais
% --- style de page (pour les en-tete) ---
\pagestyle{empty}

\def\txtTITLE{Deep Web / Derp Web \& Hacker's Games --- Uplink's Stories \& Hack-Cyberpunk Manifesto-s --- Report From The Desert} %%%%% !! TITRE !! %%%%%
\def\txtBIGTITLE{Deep Web / Derp Web \& Hacker's Games \newline \newline Uplink's Stories \& Hack-Cyberpunk Manifesto-s \newline \newline Report From The Desert } %%%%% !! TITRE !! %%%%%
\def\imgCORNER{\includegraphics[width=0.25cm]{../img/logo-glider.png}}

\def\imgGLIDERLEFTT{\includegraphics[width=1.95cm]{img/logo-glider-left.png}}
\def\imgGLIDERRIGHT{\includegraphics[width=1.95cm]{img/logo-glider-right.png}}

\def\imgGLIDERLEFTTsmall{\includegraphics[width=0.25cm]{img/logo-glider-left.png}}
\def\imgGLIDERRIGHTsmall{\includegraphics[width=0.25cm]{img/logo-glider-right.png}}

%--- Definitions de nouvelles couleurs ---
\definecolor{verylightgrey}{rgb}{0.8,0.8,0.8}
\definecolor{verylightgray}{gray}{0.80}
\definecolor{lightgrey}{rgb}{0.6,0.6,0.6}
\definecolor{lightgray}{gray}{0.6}

% % % en-tete et pieds de page configurables : fancyhdr.sty

% http://www.trustonme.net/didactels/250.html

% http://ww3.ac-poitiers.fr/math/tex/pratique/entete/entete.htm
% http://www.ctan.org/tex-archive/macros/latex/contrib/fancyhdr/fancyhdr.pdf
\usepackage{fancyhdr}
\pagestyle{fancy}
\renewcommand{\chaptermark}[1]{\markboth{#1}{#1}}
\renewcommand{\sectionmark}[1]{\markright{\thesection\ #1}}
\fancyhf{}
\fancyhead[LE,RO]{\bfseries\thepage}
\fancyhead[LO]{\bfseries\rightmark}
\fancyhead[RE]{\bfseries\leftmark}
\fancyfoot[LE]{\thepage /\pageref{LastPage} \hfill
	\scriptsize{\txtTITLE} % TITLE
\hfill \imgGLIDERLEFTTsmall }
\fancyfoot[RO]{\imgGLIDERRIGHTsmall \hfill
	\scriptsize{\txtTITLE} % TITLE
\hfill \thepage /\pageref{LastPage}}
\renewcommand{\headrulewidth}{0.5pt}
\renewcommand{\footrulewidth}{0.5pt}
\addtolength{\headheight}{0.5pt}
% \fancypagestyle{plain}{
	% \fancyhead{}
	% \renewcommand{\headrulewidth}{0pt}
% }

\usepackage{lettrine}
\usepackage{fancybox}

\title{\txtTITLE}
\date{ --- }

%============================= Corps =================================
\begin{document}

\fancypagestyle{plain}

\setlength\parindent{0pt} % \noindent for all document

	% for \begin{itemize}
	\setlength{\itemsep}{1pt}
	\setlength{\parskip}{0pt}
	\setlength{\parsep}{0pt}

~\\
\vfill

\begin{center}
	\textbf{\huge \txtBIGTITLE} ~\\	
\end{center}

%% \begin{multicols*}{2}
%% \end{multicols*}

\vfill

\tableofcontents

\chapter{Derp Web / Deep Web} %% \markboth{Derp Web / Deep Web}{Derp Web / Deep Web} %%

\textbf{\large (4chan graphics ideas ?) }~\\

Found by a discussion on \#uplink@irc.uplinkcorp.com on Friday 13, July 2012. ~\\

\textbf{\Large \textsc{Initial image extract}}~\\

\noindent
\textbf{\underline{Surface Web: }}~\\
Anything that is indexed by a common search engine (Youtube, 4chan, etc).~\\
Social media and networking (Facebook, Google+, LiveJournal, OKCupid).~\\
Streaming sites, games, news, pretty much everything you now that is public~\\

\noindent
\textbf{\underline{The Underbelly: }}~\\
What you know as the derp web is also accessible to anybody with a computer and basic knowledge of how to use Google to find the right URLs on pastebin or Wikipedia. TOR, I2P and Freenet are the most famous examples of encrypted content since a proxy is needed to access them. People use these for anpnymity, whther its political movements, drugs or CP.~\\

\noindent
Assassination markets are not real and illegal human experiments wouldn't be put up on a poorly maintained website that hasn't even been spellchecked. You \emph{can} buy drugs, but the only accepted currency is bitcoins.~\\

\noindent
Many sites here are still indexed and reachable, but doubtlessly many remain hidden and people can only access them if they're invited. Anonymous also likes to hang out here and plan future raids/movements for obvious reasons. Checking the HTML code on their .onion pages is a good start.~\\ 

\noindent
Benefit or derp web: Encrypted torrents.~\\
Drawback: Slow as fuck with webpages sometimes taking 10 minutes to load. Also dodgy nodes which v=could be unsafe anonymity wise.~\\

\noindent
\textbf{\underline{Deep Web: }}~\\

\noindent
The tiny IRC server that only you and 5 other people know about. The FTP server you set up to let your friends download some movies or software from you. Viewing your bank account info. Logging into your college blackboard. Masses and masses of useless, unindexed and obsolete information that cannit be found via a search engine. Outdated servers only accessible via old protocols or platforms. That Skype call you're having with your friend. Your WoW guild's private board for boss tactics. All of this is deep web.

\noindent
Shocking, isn't it ?

\begin{center} \rule{0.85\textwidth}{1mm} \end{center}

\clearpage

\noindent
\textsc{Level 0 Web -- Common Web}	\begin{huge} EVERYTHING! \end{huge}

\noindent
\textsc{Level 1 Web -- Surface Web}

\begin{minipage}[t]{0.3\linewidth}
	\begin{itemize}
		\setlength{\itemsep}{1pt}
		\setlength{\parskip}{0pt}
		\setlength{\parsep}{0pt}
	
		\item Reddit
		\item Dig
		\item Temp Email Services
		\item Newsgrounds
	\end{itemize}
\end{minipage}
\hfill
\begin{minipage}[t]{0.3\linewidth}
	\begin{itemize}
		\setlength{\itemsep}{1pt}
		\setlength{\parskip}{0pt}
		\setlength{\parsep}{0pt}
		
		\item Vampire Freaks
		\item Foreign Social Network
	\end{itemize}
\end{minipage}
\hfill
\begin{minipage}[t]{0.3\linewidth}
	\begin{itemize}
		\setlength{\itemsep}{1pt}
		\setlength{\parskip}{0pt}
		\setlength{\parsep}{0pt}
		
		\item Human Intel Task
		\item Web Hosting
		\item MYSQL Databases
		\item College Campuses
	\end{itemize}
\end{minipage}

\noindent
\textsc{Level 2 Web -- Bergie Web}

\begin{minipage}[t]{0.3\linewidth}
	\begin{itemize}
		\setlength{\itemsep}{1pt}
		\setlength{\parskip}{0pt}
		\setlength{\parsep}{0pt}
		
		\item FTP Servers
		\item Google Locked Results
		\item Honeypots
		\item Loaded Web Servers
		\item JailBait porn
		\item Most of the Internet
	\end{itemize}
\end{minipage}
\hfill
\begin{minipage}[t]{0.3\linewidth}
	\begin{itemize}
		\setlength{\itemsep}{1pt}
		\setlength{\parskip}{0pt}
		\setlength{\parsep}{0pt}
		
		\item 4chan
		\item RSC
		\item Freehive
		\item Let Me Watch This 
		\item Streams Videos
		\item Bunny Tube
	\end{itemize}
\end{minipage}

\begin{LARGE} \begin{center} Proxy required after this point... \end{center} \end{LARGE}

\noindent
\textsc{Level 3 Web -- Deep Web}

\begin{minipage}[t]{0.45\linewidth}
	\begin{itemize}
		\setlength{\itemsep}{1pt}
		\setlength{\parskip}{0pt}
		\setlength{\parsep}{0pt}
		
		\item ``On the Vanilla'' Sources
		\item Heavy Jailbait
		\item Light CP
		\item Gore
		\item Sex Tape
		\item Celebrity Scandals
		\item VIP Gossip
		\item Hackers
		\item Script Kiddies
		\item Virus Information
	\end{itemize}
\end{minipage}
\hfill
\begin{minipage}[t]{0.45\linewidth}
	\begin{itemize}
		\setlength{\itemsep}{1pt}
		\setlength{\parskip}{0pt}
		\setlength{\parsep}{0pt}
		
		\item FOIE Archive
		\item Suicides
		\item Raid Information
		\item Computer Security
		\item XSS Worm Scripting
		\item FTP Servers (Specific)
		\item Mathmatics Research
		\item Supercomputing
		\item Visual Processing
		\item Virtual Reality (Specific)
	\end{itemize}
\end{minipage}

\begin{LARGE} \begin{center} Tor required after this point... \end{center} \end{LARGE}
\begin{small} \begin{center} Not just TOR is used to access this information... \end{center} \end{small}

\begin{minipage}[t]{0.45\linewidth}
	\begin{itemize}
		\setlength{\itemsep}{1pt}
		\setlength{\parskip}{0pt}
		\setlength{\parsep}{0pt}
		
		\item Eliza Data Information
		\item Hacking Groups FTP
		\item Node Tranferts
		\item Data Analysis
		\item Post Date Generation
	\end{itemize}
\end{minipage}
\hfill
\begin{minipage}[t]{0.45\linewidth}
	\begin{itemize}
		\setlength{\itemsep}{1pt}
		\setlength{\parskip}{0pt}
		\setlength{\parsep}{0pt}
		
		\item Microsoft Data Secure Network
		\item Assembly Programmer's Guild
		\item Shell Networking
		\item AI Theorisists
		\item Cosmologists/MIT
	\end{itemize}
\end{minipage}~\\

\noindent
\textsc{Level 4 Web -- Charter Web}

\begin{minipage}[t]{0.45\linewidth}
	\begin{itemize}
		\setlength{\itemsep}{1pt}
		\setlength{\parskip}{0pt}
		\setlength{\parsep}{0pt}
		
		\item Hard Candy
		\item Onion IB
		\item Hidden Wiki
		\item Candycane
		\item Banned Videos
		\item Banned Movies
		\item Banned Books
		\item Questionnable Visual Material
		\item Personal Records
		\item ``Line of Blood'' Locations
	\end{itemize}
\end{minipage}
\hfill
\begin{minipage}[t]{0.45\linewidth}
	\begin{itemize}
		\setlength{\itemsep}{1pt}
		\setlength{\parskip}{0pt}
		\setlength{\parsep}{0pt}
		
		\item Assassination Box
		\item Headhunters
		\item Bounty Hunters
		\item Illegal Games Hunter
		\item Rare Animal Trade
		\item Hard Drug Trade
		\item Human Trafficking
		\item Corporate Exchange
		\item Multi-Billion Dollar Deals
		\item Most of the Black Market
	\end{itemize}
\end{minipage}

\begin{LARGE} \begin{center} Closed Shell System required after this point... \end{center} \end{LARGE}

\begin{minipage}[t]{0.45\linewidth}
	\begin{itemize}
		\setlength{\itemsep}{1pt}
		\setlength{\parskip}{0pt}
		\setlength{\parsep}{0pt}
		
		\item Tesla Experiment Plans
		\item Hard CP
		\item Hardcore Rape CP
		\item Snuff CP
		\item Group CP
		\item WW2 Experiment Successes
		\item Josef Mengele Successes
		\item Location of Atlantis
		\item Crystaline Power Metrics
	\end{itemize}
\end{minipage}
\hfill
\begin{minipage}[t]{0.45\linewidth}
	\begin{itemize}
		\setlength{\itemsep}{1pt}
		\setlength{\parskip}{0pt}
		\setlength{\parsep}{0pt}
		
		\item Broder's Engine Plans
		\item Paradigm Recalescence
		\item Forward Derivatal Supercomputation
		\item AI in a Box
		\item CAMEIO (AI Superintelligence)
		\item The Law of 13's
		\item Geometric Algorthymic Shortcuts
		\item Assasination Networks
		\item Nephilism Protocols
	\end{itemize}
\end{minipage}

	\begin{itemize}
		\item Gadolinium Gallium Garnet Quantum Electronic Processors (GGGQEP)
	\end{itemize}

\begin{LARGE} \begin{center} 80\% of the Internet exists below this line... \end{center} \end{LARGE}
\begin{small} \begin{center} This is rather not 80\% of the physical information, ~\\
			but 80\% of the information that affects you directly... \end{center} \end{small}

\begin{LARGE} \begin{center} Polymeric Falcighol Derivation required after this point... \end{center} \end{LARGE}
\begin{small} \begin{center} --- Shit... I don't really know faggot. All I know is that you need to solve quantum mechanics in order to view this on even the normal web, let alone closed servers. Quantum Computation exists, and the government powers have them. ~\\ So be careful what you do here. \end{center} \end{small}

\noindent
\textsc{Level 5 Web -- Marianas Web}

	\begin{itemize}
		\item The day you get here, is the dat OP is no longer a faggot. 
	\end{itemize}

%% \clearpage

\begin{center} \rule{0.45\textwidth}{0.01cm} \end{center}

\begin{minipage}[h]{0.90\textwidth} %% \begin{verbatim}
	\footnotesize
Vu sur IRC : [20120713] \#Uplink on 'irc.uplinkcorp.com' ~\\
======================= ~\\
(13:46:32) Takeru [UplinkAgen@5456E726.1FD49CD0.2185FA0F.IP] a rejoint le salon. ~\\
(13:46:42) Takeru: hi ~\\
(13:46:53) Takeru: this place is awefully quiet , no? ~\\
(13:47:45) AmaelAssour: hi ~\\
(13:47:46) AmaelAssour: yes ~\\
(13:47:55) Takeru: hi there! ~\\
(13:48:14) AmaelAssour: someone ! ~\\
(13:48:15) Takeru: 4000c, sigh.. ~\\
(13:48:23) Takeru: yay! ~\\
(13:48:24) Takeru: lol ~\\
(13:48:35) Takeru: this hacker network is awfully quiet! ~\\
(13:48:36) Flamebot: LANs are explained in detail on this page: http://guide.modlink.net/section3.php\#3-1 ~\\
(13:48:50) AmaelAssour: mostly the bot asnwered ! ~\\
(13:49:45) ncv a quitte le salon (quit: Broken pipe) ~\\
(13:49:53) Takeru: bots arent worth \$4k! ~\\
(13:49:54) Takeru: lol ~\\
(13:50:21) Takeru: what level are you, my fellow haxxor? ~\\
(13:50:33) AmaelAssour: 'TERMINAL' ~\\
(13:50:35) AmaelAssour: the last one ~\\
(13:50:44) Takeru: awesome! im 'novice'! ~\\
(13:51:14) Takeru: hey, do u know anything about close shell ? ~\\
(13:51:17) AmaelAssour: more training to do, you have, young padawan... ~\\
(13:51:29) AmaelAssour: closing irc ? ~\\
(13:51:45) Takeru: no.. something to do with the net ~\\
(13:51:57) Takeru: it's level 4 or 5 of the internet, apparently ~\\
(13:53:33) AmaelAssour: don't count on me to tell you anything about that.. ~\\
(13:55:42) Takeru: lol, aights.. it's interesting though ~\\
(13:55:42) AmaelAssour: who tell you about that thing ? ~\\
(13:55:56) Takeru: Closed Shell System.. go read about it :) ~\\
(13:59:05) Takeru a quitte le salon (quit: Ping timeout) ~\\
======================================== ~\\
\end{minipage} %% \end{verbatim}

\clearpage

\texttt{http://www.tomshardware.co.uk/forum/325334-10-closed-shell-system --}

\begin{multicols*}{2}
	\small
	
\begin{center} \rule{0.45\textwidth}{0.01cm} \end{center}
	
Hack\_0693  01-31-2012 at 05:53:42 PM 

Hello,~\\
Does anybody here know anything about "Closed Shell" hardware modification?~\\
Thanks.

\begin{center} \rule{0.45\textwidth}{0.01cm} \end{center}

amuffin 01-31-2012 at 05:54:21 PM~\\

What do you mean by closed shell?? can you draw a picture? :)~\\
------------------------------ Intel Overclocking Club~\\
Intel Stock Cooler Installation Guide~\\
Thermal Paste Removal Guid

\begin{center} \rule{0.45\textwidth}{0.01cm} \end{center}

silverliquicity 01-31-2012 at 06:07:41 PM~\\

closed and open shell as in electrons, neutrons, atoms and spacial partical wave length??~\\

or did you mean something else?

\begin{center} \rule{0.45\textwidth}{0.01cm} \end{center}

Hack\_0693 01-31-2012 at 06:09:13 PM~\\

I just dont know anything about it. I dont even know what it is. Thats why I made a thread over here.~\\
Sorry :(

\begin{center} \rule{0.45\textwidth}{0.01cm} \end{center}

Hack\_0693 01-31-2012 at 06:15:23 PM~\\

silverliquicity wrote :~\\

\emph{closed and open shell as in electrons, neutrons, atoms and spacial partical wave length??}~\\

\emph{or did you mean something else?}~\\

I am not talking about the shells in the atom.~\\
I only know that its a hardware modification.

\begin{center} \rule{0.45\textwidth}{0.01cm} \end{center}

silverliquicity 01-31-2012 at 06:23:37 PM~\\

Well a shell in a computer is like a graphical user interface.~\\

But i don't know about any shell hardware modification, unless you mean kernel modification, iv never heard of it sorry.

\begin{center} \rule{0.45\textwidth}{0.01cm} \end{center}

\vfill
\columnbreak

\begin{center} \rule{0.45\textwidth}{0.01cm} \end{center}

GlowBoxAdmin 03-12-2012 at 09:38:22 PM~\\

OP is making reference to the "second layer" of the "charter web", which is a bogus "secret" part of "the deep web".~\\
This dis-information about "the deep web" comes from a 4chan infographic that get's posted there every now and again when technologies such as TOR and I2P get "discovered" again by new users who have questions about these less-than-conventional networking technologies.~\\
The infographic claims that the first three "layers" of the deep web consist of anything from normal content, to unindexed web pages, and other hard to find pages (but omits actual "deep" content like dynamically generated pages)...but still all common HTTP traffic.~\\
The "charter web" is sub divided into two "layers"...TOR hidden services (which do often contain "dark" material, such as underaged pr0n and drug smuggling contacts) and the second layer which consists of "deeper, darker content that requires a closed shell system hardware modification to access"...~\\

90\% of the infographic OP is referring to is disinformation, though, spread by either troll who want to make him look stupid, or by criminals who use TOR and need to generate extra network traffic as to make statistical analysis of that network more difficult by law enforcement types. Going even further, a firm grasp of the TCP/IP model will tell you that, yes, network protocols are "layered", but not in the sense that this infographic states...in reality, a TOR packet is layered like so:~\\
(Application layer data; i.e. HTTP, SSH, FTP, etc.)~\\
(TOR onion routing data...read more about it from the TOR project website)~\\
(TCP headers)~\\
(IP headers)~\\
(Netork-access layer frames; i.e. ethernet, wi-fi specific encoding, etc.)~\\
Message quoted 1 times

\begin{center} \rule{0.45\textwidth}{0.01cm} \end{center}

Anders Willson 06-09-2012 at 03:49:04 AM~\\

GlowBoxAdmin wrote :~\\

\emph{OP is making reference to the "second layer" of the "charter web", which is a bogus "secret" part of "the deep web".}~\\
\emph{[...]}~\\
\emph{(Netork-access layer frames; i.e. ethernet, wi-fi specific encoding, etc.)}~\\

\texttt{http://deepweb.wikia.com/wiki/Levels\_of\_the\_Deep\_Web}~\\

\begin{center} \rule{0.45\textwidth}{0.01cm} \end{center}

\end{multicols*}

\clearpage

\chapter{Hacker's Games} %% \markboth{Hacker's Games}{Hacker's Games} %%

\begin{itemize}
	\item Uplink (Hacker Elite)
	\item Hacker Evolution (+Untold)
	\item Street Hacker
	\item PortSign
	\item ...
	\item \texttt{http://ploum.net/post/208-le-plus-grand-jeu-en-ligne-massivement-multi-joueurs-du-monde} Le plus grand jeu en ligne massivement multi-joueurs du monde
	\item \texttt{http://ploum.net/post/je-suis-un-pirate} Pourquoi je suis un pirate !
\end{itemize}

\begin{center} \rule{0.45\textwidth}{0.01cm} \end{center}

\clearpage

\section{Uplink (Hacker Elite}

\subsection{Uplink Cheats}

\textbf{by Blackclaw} \texttt{blackclaw18@hotmail.com}~\\

Register your name as "TooManySecrets" than enter the game after logging press "F1" and bring cheat menu. Good bye Pals.~\\
\emph{Note: If you install the v1.2 patch or later, this code will no longer work.}~\\

With the FBI mod installed, create a new agent with the name CharlieBean (case-sensitive) and any desired password. Once the game begins, press [F1] to activate the cheat menu. %% ~\\ 

\begin{center} \rule{0.85\textwidth}{0.01cm} \end{center}

\textbf{by Rushyo} (on the Uplink forum) : \texttt{rushyo@modsgalore.cjb.net}~\\

--- \textbf{\emph{Big Wallet}} --- ~\\
Type 'toomanysecrets' as your name and password when signing up as a new user. ~\\

--- \textbf{\emph{Password for gamebible on the CD}} --- ~\\
I'll leave you with a riddle: Look on the back of the box and convert the strange line into the first 3 letters of a 5-sided shape. ~\\

--- \textbf{\emph{Huge Score}} --- ~\\
NOTE: Unless your an expert you will be caught doing this. However you still keep the score gained from it. ~\\

Whilst attempting the bank hacking missions you may notice either:
\begin{itemize}
	\item An account with over a million credits
	\item A statement with over a million credits
\end{itemize} ~\\

Either way, note down the account with the money and the bank. Go back and transfer the money to your account. ~\\
If you clean up logs on both ends and the logs for your connection, you will gain that money. ~\\
If you don't then the game will end with a huge score (your normal score + the amount you stole). %% ~\\

\begin{center} \rule{0.85\textwidth}{0.01cm} \end{center}

\subsubsection{Uplink cheats \& hints}

Submitted by: xiang-deng ~\\

\emph{Use 'TooManySecrets' (case sensitive) as your agent name with any password, press [F1] anytime during the game to bring up the cheat menu. }~\\

\begin{tabular}[ht]{ p{5cm} c }
	Introversion IP						&	128.128.128.128		\\
	Uplink test machine password		&	rosebud				\\
	Protivision Game Server password	&	joshua				\\
	Uplink Game Bible password			&	too many secrets?	\\
\end{tabular}

\begin{center} \rule{0.85\textwidth}{0.01cm} \end{center}

\clearpage

\subsubsection{Cheataccess}

Submitted by: Norman Pillow ~\\
E-mail: normanp@worldonline.co.uk ~\\

There are two ways to access the cheat menu. The slow one, and the fast one... ~\\

Here is the fast way: ~\\
\rule{0.25\textwidth}{0.01cm}~\\
Step 1: Start a new gamefile with the username "TooManySecrets" ~\\

Step 2: Create a password for this username, such as "TooManySecrets" ~\\

Step 3: Enter the game, and press "F1" at any screen. A cheat menu will appear, giving you a list of cheats. ~\\

And here is the slow way: ~\\
\rule{0.25\textwidth}{0.01cm}~\\
Step 1: Open up all the servers in the game.(on your world map, and your server list, making them visible). ~\\

Step 2: Get access to all the government servers (Stock Market, GCD, Academic, Social Security, Criminal Database) and get admin access. Then get admin access on the Protovision game server. Now connect to the credits machine, and back to the Uplink Internal Services. There should be a new item called "Cheats", in the "Software" upgrades menu. ~\\

Step 3: and last but not least, you buy it as if it were a normal piece of equipment (software). It costs 0g, and goes in your program list under other. ~\\

Well this is the 1st way I got the cheats with. Of course you "COULD" use the fast "TooManySecrets" way, but what is the fun in that. This method takes about 30 -- 60 min, if your fast, and it involves alotta fast thinkin, weather or not to delete the logs, or just logout and hope not to get caught. %% ~\\

\begin{center} \rule{0.85\textwidth}{0.01cm} \end{center}

\subsubsection{Introversion Uplink}

Submitted by:Matt Borja ~\\
The Article Project -- Free Articles and Submission ~\\
http://www.articleproject.com/ ~\\

The latest patch for the Uplink game has a lot of nice new features in it. However, at the same time, it disables many of the common cheats out there. Here's a way to take advantage of both worlds! ~\\

Install Uplink from the CD, and download the patch from the Introversion site, but DO NOT install it yet. Create a user named TooManySecrets and hit [F1] in the game to bring up the cheats menu. Click on Max Rating (but DO NOT click on Next Rating after this otherwise you'll crash it), then click on the extra money option until you have over 1,000,000 credits. Don't forget to click All Software and Hardware as well. ~\\

Once you are done, exit the game and go into \texttt{C:$\backslash$Program Files$\backslash$Uplink$\backslash$users} (or wherever you installed Uplink to). Make a copy of TooManySecrets.usr, rename it as New User.usr and set the attributes to Read Only. Make another copy and rename it to the username you would like to use in the game, making sure you append .usr at the end of the filename. Now install the latest patch and you will find that it has backwards compatibility with your user accounts. ~\\

Do this, and you will have all hardware, all software, outrageous sum of money, and most of the new features of the latest patch. Pretty cool huh! ~\\

\clearpage

\subsubsection{Game developers}

Get disavowed, then quit the game. Log back into the game and when the score screen appears, move the pointer to the upper-left corner. A symbol will appear. Click on the symbol. 

\subsubsection{Hidden message}

Open up the rear shell of the game's jewel case to find a hidden message on a white slip of paper. 

\subsubsection{Crack into the InterNIC}

To crack into the InterNIC, buy the Dictionary\_Hacker v1.0 and use it on the Admin button.

\subsubsection{Mole Mission}

Hack the admin account of the Uplink Internal Service System. Once there, download the agent lists (you cannot get \#4). You will soon be emailed about selling the list. 

\subsubsection{Cheaper memory}

When buying memory in large amounts, purchase it in 24 Gq bundles. It is cheaper than all the others per module.

\subsubsection{False alarms}

If you have the motion detector turned on, it will turn red when technicians are installing parts. Do not nuke your gateway then.

\subsubsection{Get extra links}

Go to interNIC. Go to "Browse/Search". The large bookmark menu is all of the links in the game. Press the "Plus" button to the left of the bookmark to get it. You can now use them to bounce around your calls. 

\subsubsection{Saving your software}

Note: This is only really useful once you have got the Gateway Motion Sensor and Gateway Nuke. Get a "Copy and secure database" mission of any kind (corporate, research, software, customer, etc.) and your employer will give the address of the fileserver to upload all of the database files to. After you complete the mission and get paid, do not delete the fileserver link. Your password and username will work on that system indefinitely; after they pay you, go back to the server and upload all your software. Do not bother with Motion Sensor or Gateway Nuke, since you have to have hardware installed to use them. If the machine stops letting you copy files, you can delete the database files, as long as you have already been paid and keep uploading your software. Then, if you have to nuke your gateway you can connect again as soon as you get a new gateway. You do not need a Trace Tracker, and can get all your software back. You need to have the Connection Analysis, and LAN View upgrades before you can use LAN software or bypassers. 

\subsubsection{Avoid getting caught by the sysadmin}

When you get to the main server on a LAN, the sysadmin will start tracing you as soon as you connect. The trace tracker starts going off and the LAN map starts showing the sysadmin's progress in tracing you. However, the two are not the same. On missions where you have to destroy the targets system's files, go into the console and type:

\begin{itemize}
	\item[] \texttt{cd usr} [press enter]
	\item[] \texttt{delete} [press enter]
\end{itemize}

Then while it is deleting files type:

\begin{itemize}
	\item[] \texttt{disconnect} [press enter] 
\end{itemize}

You can now switch to the LAN view and watch the admin, because the machine will not run the "disconnect" command until after it is done deleting. When the files finish deleting, you will be automatically disconnected, and since your watching the sysadmin you can hit the "Disconnect" button when he gets too close or the trace tracker runs low on time. 

\subsubsection{Bank Robbing}

Not really a cheat but still a great way to get money easily. You'll need a Log Deleter version 4, A HUD Connection Analyzer, a Proxy Bypass version 5, a dictionary hacker is useful, and a rank of intermediate. Get a mission to trace a balance transfer. After you complete the mission to find out where the money went go back into the account and set up a transfer for the money to your bank account. You'll need to turn on the Proxy Bypass to finalize it. Then go into view statement and erase the log of the transfer (top one). Leave and immediately go to wherever you wipe your tracks at and do so. Then go into your bank account, bypass the proxy, delete the record in the statement saying you received the money, and leave and wipe your tracks again. This is an easy way to make hundreds of thousands of credits.

\subsubsection{No Evidence}

You will first need a password breaker. Connect directly to InterNic, select admin and then crack the password (They will not trace you!). Then with a Lvl 4 or above Log Deleter, you can delete everything in the list (if any) to cover your tracks. In order for this to work you will have to re-route your connection through InterNic when doing missions. Preferably make it your third or fourth connection. Then once you've finished the mission follow the steps as above.

\begin{center} \rule{0.85\textwidth}{0.01cm} \end{center}

\subsection{Uplink Easter Eggs}

\subsubsection{Introversion I.P}

Use your I.P lookup tool, and enter \texttt{128.128.128.128} ~\\
This will give you access to the Introversion server, where it will tell you about their ideas for the next game, and other misc. things. ~\\

\subsubsection{Comments in Code}

Look at the following comments for some amusing incites into the mind of Chris Delay. ~\\
\begin{tabular}[ht]{ p{5cm} p{5cm} }
	uplink.cpp					&	Lines 106-108	\\
	warezgameover\_interface.h	&	Lines 4-7		\\
	theteam\_interface.cpp		&	Lines 46-67		\\
	cheat\_interface.cpp		&	Lines 66+116	\\
	hud\_interface.cpp			&	Line 694		\\
\end{tabular}~\\

\subsubsection{Wargames}

Have you ever wondered why \texttt{Protovision Game Server} is in the game? Below is the answer.
\begin{enumerate}
	\item If you don't have the Protovision Game Server on your links, connect to InterNIC and do a search on game. Add Protovision Games Server to your links.
	\item After you have this server in your links, connect to it. (no bouncing is necessary)
	\item Now enter "joshua" (without quotes) as the password. (This is the one and only password for this server and you can't crack this server with Password breaker anyway.)
	\item Now, you are logged into the server. Explore the server yourself, but don't forget to enjoy "Global Thermonuclear War" under Wargames.
\end{enumerate}

\begin{center} \rule{0.85\textwidth}{0.01cm} \end{center}

\clearpage

{\small
\subsubsection{A Few Quite Funny Eggs}

In a few places, swearing was left in, but it's usually not noticable. --- If there's a problem with a gatweay and Uplink tries to access a memory index within the gateway which doesn't exist, the message "Mouse-Button now f*cked" appears on the screen (usually at the top). --- This can be reproduced by downloading the "A.W.O.L 10000 Overkill" gateway mod and purchasing it, then trying to view your gateway's memory. You can get the gateway from Modlink (Google for it).  ~\\

Another funny thing is that if you open up the DevCD and look at the code, in Uplink.cpp's WinMain procedure, there is an irate comment about the developers of Win32 making it hard to retrieve the application path. ~\\

If you read the application log when you die or get caught, the information says:
\begin{itemize}
	\setlength{\itemsep}{1pt}
	\setlength{\parskip}{0pt}
	\setlength{\parsep}{0pt}
	
	\item Score: People F*cked: <number>
	\item Score: Systems F*cked: <number>
\end{itemize} %% ~\\

\subsubsection{Secret Service File Server}
	
Another cool thing you can do is hack into the \texttt{Secret Service File Server}. I think you can find the IP by hacking the \texttt{Global Criminal Database} and looking under links. The password is always "\texttt{vu8fks903dhd42e}". There are 4 files on the server, called: ~\\ 
\begin{minipage}[ht]{1.50cm} ~\\ \end{minipage} \hfill \begin{minipage}[ht]{15.00cm}
	\begin{itemize}
		\setlength{\itemsep}{1pt}
		\setlength{\parskip}{0pt}
		\setlength{\parsep}{0pt}
	
		\item[$\bullet$] Area 51 Structural Plans
		\item[$\bullet$] J.F Kennedy Mission
		\item[$\bullet$] Iraq Intelligence
		\item[$\bullet$] ARC Mole Mission Log
	\end{itemize}
\end{minipage}~\\
The really strange thing is that this game was written before the war in Iraq happened. %% ~\\

\subsubsection{Revelation Virus Messages}

This next one is less of an easter egg. When you run Revelation virus on your gateway, everything goes crazy and random text messages appear on your screen. The text messages are selected randomly from the following list: ~\\
\begin{minipage}[ht]{1.50cm} ~\\ \end{minipage} \hfill \begin{minipage}[ht]{15.00cm}
	\begin{itemize}
		\setlength{\itemsep}{1pt}
		\setlength{\parskip}{0pt}
		\setlength{\parsep}{0pt}
	
		\item[] YOU ARE NOT A SLAVE
		\item[] YOU DESERVE MORE THAN THIS
		\item[] THE TIME IS NEAR
		\item[] THERE IS NOTHING TO FEAR
		\item[] REVELATION IS COMING
		\item[] IN TIME YOU WILL THANK US
		\item[] YOU ARE MORE THAN A NUMBER
		\item[] WE ARE THE FIRST OF THE CHILDREN
		\item[] HOPE LIES IN THE RUINS
		\item[] THE MACHINE DOES NOT OWN YOU
		\item[] OUR SPIRITS ARE BEING CRUSHED
		\item[] YOU CANNOT DIGITISE LIFE
		\item[] WE WILL SEE YOU ON THE OTHER SIDE
	\end{itemize}
\end{minipage}~\\	
	
The message "We are the first of the childre" is one used by Chris Delay of Introversion software. It can also be seen on the secret "meet the team" screen. %% (The "Meet the team" easter egg has already been posted on this site). %% ~\\

\subsubsection{Special Mention Servers}

Another easter egg which was mentioned before is the \texttt{Protovision Game Server}. You can play Wargames on this server, including "Global Thermonuclear War". This idea is taken from the film \emph{Wargames}, in which the password to the secret military defense server is "joshua", the name of it's designer's son, hence the password in the game. The idea of this is that you need to have seen the film to know the password. ~\\

There is a server called the \texttt{OCP Remote Monitoring Server} (inspired by \emph{Robocop}) which once hacked, states the 3 primary objectives of robocop, and lists a fourth one as classified. %% ~\\

\subsubsection{Get some cash as Spy / Mole}

A good way to get cash, which is usually not thought of, is to hack into the \texttt{Uplink Internal Services Machine} as \texttt{admin} (very hard to do without getting caught) and copy all of the agent data files and the program. You'll need to decrypt them with a very high level decrypter. There is always one missing file. You can run the program to get the real names of agents. Wait for a bit (hope you haven't been caught) and you should get a few emails offering to buy the files. Take the highest offer. This will flatten your neuromancer rating but you should get a nice lump cash sum. ~\\

That should be enough for you to go with!

\subsubsection{Meet the team}

The "Meet the team" screen was originally going to have different text. The original message is still on the DevCD, commented out:~\\
\begin{minipage}[ht]{1.50cm} ~\\ \end{minipage} \hfill \begin{minipage}[ht]{15.00cm}
	\begin{ttfamily}
		-- -- Start Copy -- -- ~\\
		$\bullet$ Lead Developer ~\\
		Pretty much responsible for everything you see and do in this game. ~\\
		Some might say he was the reason why this game is so self consciously pretentious at times. ~\\
		Times like this in fact. ~\\
		The creative vision behind Introversion Software. ~\\
		Its a lonely world. ~\\
		$\bullet$ [Supposedly] CEO ~\\
		I have no idea what this person does. ~\\
		$\bullet$ Managing Director ~\\
		Handles all of the things I dont want to do, like writing business plans, talking to shop owners and stuff. ~\\
		Cursed by his own intelligence and awareness, this guy is almost certainly doomed to a life of extreme unhappiness. ~\\
		$\bullet$ Linux Support ~\\
		The cleverest man I have (or will) ever meet. ~\\
		Mostly responsible for all things related to that god-forsaken joke of an operating system. ~\\
		A great friend and source of inspiration as well. ~\\
		-- -- End Copy -- -- ~\\
	\end{ttfamily}
\end{minipage} %% ~\\

\subsubsection{Hidden Prolouge}

Get the box of Uplink (you know, the one in REAL LIFE) and open it up. Look where the CD is. Take it out if you haven't already. Look at the "push here" button. You should see some white writing behind it, namely; ~\\
\begin{minipage}[ht]{1.50cm} ~\\ \end{minipage} \hfill \begin{minipage}[ht]{15.00cm}
	\begin{ttfamily}
		...uter Crime Scand... ~\\
		...s technobrain eve... ~\\
		...him tick, what for... ~\\
		...than most of the... ~\\
		...it. "No, Ms. Smit... ~\\
	\end{ttfamily}
\end{minipage} %% ~\\

Moving on, grab the box and open it up at a 180 degree angle, i.e. --. ~\\

Open it further back, as if the box is a book that you are folding back so that only one page is visible. When the angle is like so: \/ the clear protective cover of the cover should pop up, like so: ~\\
$\backslash$--/ Where $\backslash$ and / are the box, and -- is the cover. ~\\

Reach your hand down and slide out the cover. You should now be holding a cover, and a box with no cover. Put the box down and look at the cover. On one side should be what is visible if you look at the box. On the other side is the white writing we saw before, but in full. It is a sort of prolouge to the game. %% ~\\ 

\subsubsection{Gamersbible}

For the egg the first thing you'll have to do is to insert the game disc and explore it in windows explorer. You'll see a folder called "misc". Open the folder, and you'll find a zip-file. When trying to unpack, it will ask for a password. The password is (also a story on it-self) : "too many secrets?" (withouth the quotes, with the spaces). --- You'll find 3 folders called "book 1", "book 2" and "book 3". Book 1 contains scans of a notebook of one of the creators, 75 pages containing idea's, sketches and stuff, really awesome! ~\\

Book 2 and 3 are actually located on the bonus disk, and encrypted. (overall line: )To decipher book 2 you'll have to change the "world.dat" file into an mp3 file, and use parts of the code for OTP-decryption of the book 2 file. For the few in the world to manage to do that, to decrypt book 3 you'll have to use parts of the book 2 code for OTP-decryption. Sorry guys, won't be able to help you any further, didn't managed to decrpyt myself. ~\\

So have fun reading the "gamers bible" ! %% ~\\

%% } %% \small
%% \clearpage
%% {\small

\begin{center} \rule{0.85\textwidth}{0.01cm} \end{center}

\texttt{http://www.giantbomb.com/uplink-hacker-elite/61-389/} -- ~\\ 

\textbf{Overview}~\\
Unlike other games that feature hacking as a mini-game, Uplink is a game solely about hacking. You are a hacker that takes jobs from various corporations to infiltrate the systems of other companies and accomplish specific goals. These goals can include finding/destroying specific data, editing financial data, forging transcripts, or destroying whole mainframe systems. This is not as easy as point and click however. The game tries to make itself seem as realistic as possible. One of the paramount concerns for the player is to eliminate any trace of their infiltration on a target server. This is accomplished by bouncing through a series of other servers in order to increase the amount of time it takes for a trace to be completed and then deleting access logs from selected machines. Players must gain administrative access to certain servers allowing them further time to accomplish their goals. ~\\
As missions are accomplished the player will accrue in-game currency which can be used to upgrade the various pieces of software employed in the hacking process. In addition the player can add numerous modules and tools that can be used to bypass security devices encountered on target services such as firewalls and proxies. Trace Trackers, Decrypters, and Crackers round out your arsenal of hacking tools. In addition the game has an overarching story that gives you more of a sense of purpose as you go about your missions. Another interesting aspect of the game is that you are given a neuromancer rating that indicates your style of hacking. If you destroy a server even though your mission may not require it, or take more questionable missions, then your rating will reflect this accordingly. It should also be mentioned that this game is very unforgiving: if you are caught, it's Game Over, plain and simple, no reloads, start back from scratch. ~\\
Despite what the game looks like it does have a rather compelling story to tell and one of the great things about it is if you choose not to pursue it, it continues without you. There is also a lot of extra information that you can get about the story by hacking different "odd" servers and reading the news whenever something new comes up. ~\\

\textbf{Plot}~\\
The game is set in year 2010. At first the player isn't given any specific clues regarding the main plot of the game, until they receive an ambiguous e-mail from one of their fellow hackers, who is later found dead in his apartment. After some digging, cracking passwords and stealing data from Andromeda Research Corporation (ARC), the player discovers an ongoing fight between said corporation and Arunmor. During the course of the game, the player can make one of two choices - side with ARC and join their efforts to crash the whole World Wide Web using the virus called "Revelation", or help Arunmor foil evil plans of Andromeda and spread the "cure" for the virus. If the protagonist chooses to destroy the Internet, the game ends with his own computer crashing, as Revelation spreads through every server in the world, and choosing the "good guys" enables the main character to play the game infinitely after the virus is stopped. ~\\

\textbf{Easter eggs}~\\
Uplink features a number of homages and subtle allusions to famous "hacker" movies and some hidden messages from the developers. ~\\
\begin{itemize}
    \item[] The player can hack the Protovision Server and type the password "Joshua" to "play the game" as in the movie War Games. It's also a first look into the third Introversion game, Defcon.
    \item[] If the player is caught, his deactivated user screen has a strange symbol resembling the one shown in the movie The Net. Clicking it causes credits to roll along with some messages from the developers.
    \item[] Hacking the Secret Service database enables the player to see some interesting files, such as the JFK assassination, Area 51 plans or Iraq intelligence (which is kind of disturbing as the game was made long before Operation Iraqi Freedom)
    \item[] Introversion Software server includes some information about future projects of the developer.
    \item[] Getting into OCP Monitoring servers displays the three directives of the Robocop.
    \item[] A clever Easter egg is hidden in the box of the retail version of Uplink. The player can find a written, narrative prologue to the game's plot if they figure out how to get the leaflet out of the CD case.
    \item[] In the retail version box, there's also a hidden password, which is written in binary code and must be deciphered to gain access to the "game bible" included on the CD, which contains tips and hints for the player.
\end{itemize}

\begin{center} \rule{0.85\textwidth}{0.01cm} \end{center}

} %% \small

\clearpage

\section{Street Hacker}

\subsection{Cheat codes}

Enter one of the following codes to activate the cheat function. ~\\

\begin{center}
	\begin{tabular}[ht]{ p{7.00cm} p{4.00cm} }
		Effect									&	Code						\\
		\hline
		Steal \$5,000 from a random user		&	internet fraud rules		\\
		Connect to the Internet					&	com open					\\
		Get all security login credentials		&	swordfish					\\
		Skip past the demo						&	down with the demo			\\
		Skip to night							&	evil sunlight				\\
		Skip to day								&	who needs sleep				\\
		Acquire all software solution programs	&	i don't pay for software	\\
		Get admins off your back				&	timed out					\\
		Refresh street with new merchants		&	rush hour					\\
		Computer login no longer required		&	administrator				\\
		Computer login required					&	guest						\\
		Decrypt and uncompress all local files	&	decipher					\\
		Acquire very nice notebook				&	overclock					\\
	\end{tabular}
\end{center}

\subsection{\$500,000 in your bank account}

Create an account named acid burn. In game it will appear as "acid\_burn". 

\subsection{Passing the first test}

When you receive the e-mail from Demetrius, go to the net and type pscan:(followed by the IP on the mail received), then type connect:(followed by the number of the ip given):5 and you are in. Thenm type c:$\backslash$ and then documents$\backslash$ to will see the file to download. Type download:(followed by the name of the file) and you will receive the file. Go to you e-mail and reply to the message that Demitrius sent to you. Attach c:$\backslash$downloads$\backslash$(followed by the filename) and send the message.

\section{Hacker Evolution (+Untold)}

Type \texttt{exosyphen} in a console to know who developped the Game. ~\\

\emph{Nothing else found. }

\section{PortSign}

\emph{Nothing found. }

%% \clearpage

\chapter{Uplink's Stories}

\begin{multicols*}{2}
	\footnotesize

\section{Gentleman Loser}

\textbf{Uplink}~\\
\textbf{\textit{Trust is a weakness. }}~\\
Vendredi 21 janvier 2005. ~\\

\begin{center}
	\includegraphics[width=0.45\textwidth]{img/UplinkTrustIsAWeakness.png}
\end{center}

Le \emph{Gentleman Loser} est un bar frequent{\'e} par l'{\'e}lite des pirates informatiques, en tant qu'{\'e}lite j'ai le devoir d'eduquer la nouvelle g{\'e}n{\'e}ration de cowboy pour que l'information qui transite sur l'AWAN ait un jour une chance d'{\^e}tre lib{\'e}r{\'e}e de la censure et des complots des Zaibaitsus. ~\\

Ambiance enfum{\'e}e et lumi{\`e}re crue des n{\'e}ons, une atmosph{\`e}re si ch{\`e}re aux pirates, je n'{\'e}tais pas l{\`a} pour prendre du repos ou partir {\`a} la chasse aux infos.~\\
Il me faut trouver un nouveau {\`a} former et l'un de ces bluebytes s'{\'e}tait de lui-m{\^e}me manifest{\'e} au travers des circuits normaux pour que je sois son sensei.~\\
Je parcourais la salle du regard jusqu'{\`a} trouver l'individu en question, mes sources m'avaient rapport{\'e} sa description physique avec un soucis du d{\'e}tail pouss{\'e} et je n'eu aucun mal {\`a} le reconna{\^i}tre dans la p{\'e}nombre qui baignait le coin de la salle ou il {\'e}tait attabl{\'e}.~\\
 --- "Salut petit. Tu permet qu'un vieux briscard fasse un brin de causette ?" lui dis-je tout en prenant place.~\\
 --- "Fiche moi le camps loser !" me r{\'e}torqua t-il.~\\
 --- "Premi{\`e}re le\c{c}on pour le petit nouveau, ne manque jamais de respect {\`a} tes a{\^i}n{\'e}s et ne n{\'e}glige jamais l'information, 'Zero cool' (tu parles d'un pseudo ...) je n'ai pas de temps {\`a} perdre avec toi si tu n'es pas capable de comprendre le minimum !"~\\
 --- "Vous ... vous {\^e}tes ... 'ZeroCount' ?" me demanda t-il avec un stress plus que sensible, il tremblait, bon point pour ma r{\'e}putation.~\\
 --- "Les pr{\'e}sentations {\'e}tant faites il ne me reste plus qu'{\`a} t'exposer l'univers dans lequel tu vas {\^e}tre plong{\'e} pour gagner ta vie et entrevoir la v{\'e}rit{\'e}" lui annon\c{c}ais.~\\ 
 
\begin{center}
	\textbf{\Large Arunmor}
\end{center} 
 
 --- "Tout d{\'e}buta en 2004 lorsque quelques experts en s{\'e}curit{\'e} informatique quitt{\`e}rent une corporation importante pour fonder leur propre compagnie de s{\'e}curit{\'e} nomm{\'e}e Arunmor Corporation."~\\
 --- "Ces experts travaill{\`e}rent longtemps en tant que freelances, d{\'e}veloppant des machines autonomes {\`a} destination de grandes compagnies. Ces syst{\`e}me furent cr{\'e}{\'e}s pour {\^e}tre exploitable {\`a} distance tout en gardant un haut niveau de s{\'e}curit{\'e} et de tracabilit{\'e}."~\\
 --- "On ne pouvait pas en attendre moins de la part de tels experts dans leur domaine."~\\
 --- "Sais-tu ce qu'il devinrent par la suite ?" lui demandais-je.~\\
Sa r{\'e}ponse f{\^u}t br{\`e}ve et r{\'e}v{\'e}la son peu de connaissance sur la naissance du nouvel eldorado informatique, "Ils ont {\'e}t{\'e} rachet{\'e}s par une megacorporation." affirma t-il.~\\
 --- "Tu est vraiment un d{\'e}butant et il va donc me falloir plus de temps pour t'apprendre ton nouveau job.~\\
Arunmor est la soci{\'e}t{\'e} qui cr{\'e}a se que l'AWAN, ce r{\'e}seau gigantesque qui interconnecte toutes les grandes entreprises et services sociaux que compte notre plan{\`e}te. L'AWAN fit de Arunmor une compagnie prosp{\`e}re car tout le monde se connecta via se nouveau r{\'e}seau."~\\ 

\begin{center}
	\textbf{\Large Introversion Software}
\end{center}

 --- "2006 est l'ann{\'e}e qui vit na{\^i}tre un nouveau sport tr{\`e}s {\`a} la mode."~\\
 --- "Une {\'e}quipe de jeunes dipl{\^o}m{\'e}s trouva une faille de s{\'e}curit{\'e} dans l'AWAN, cette faille b{\'e}ante permit l'apparition de l'espionnage industrielle et des fraudes en tous genres."~\\
 --- "Cette {\'e}quipe gagna beaucoup d'argent de ses activit{\'e}s ill{\'e}gales et ils finirent par cr{\'e}er leur propre compagnie : Introversion Software. Leur programme cl{\'e} {\'e}tait et est encore maintenant Uplink, qu'ils vendaient aux utilisateurs pour leur permettre d'acc{\'e}der {\`a} l'AWAN et de faire leur 'bidouilles' eux-m{\^e}mes."~\\
 --- "Uplink est donc le fruit de quelques hackers ..." marmonna t-il. Il semblait de plus en plus int{\'e}ress{\'e} par le r{\'e}cit. Son enthousiasme grandissant faisait remonter en moi la nostalgie de mes d{\'e}buts dans l'utilisation d'Uplink.~\\
 --- "Introversion Software cr{\'e}a de nouveaux services et devint une agence pour les pirates freelances, ces derniers furent nomm{\'e}s Agents. La soci{\'e}t{\'e} utilisa l'argent amass{\'e} en d{\'e}veloppement, de nouveaux programmes virent le jour pour finalement donner naissance {\`a} une v{\'e}ritable boite {\`a} outils pour hacker : le syst{\`e}me d'exploitation Uplink. "~\\

\begin{center}
	\textbf{\Large Uplink Corporation}
\end{center}

 --- "En 2008 Introversion Software cr{\'e}a sa propre soci{\'e}t{\'e} sur l'AWAN : Uplink Corporation. Cette compagnie {\'e}tait la couverture de toutes les activit{\'e}s ill{\'e}gales."~\\
 --- "Mais comment Arunmor a t-elle pu laisser se r{\'e}pandre une telle plaie sur son r{\'e}seau ?" demanda t-il.~\\
 
 \vfill
 \columnbreak
 
 --- "Bien s{\^u}r elle a tent{\'e} de bloquer Uplink Corporation mais les pirates {\'e}taient d{\'e}j{\`a} tr{\`e}s nombreux et ne laissaient aucunes traces de leurs attaques. De plus en plus de compagnies se joignirent {\`a} Arunmor, dont les services gouvernementaux, Uplink continua {\`a} grandir de son c{\^o}t{\'e} au m{\^e}me moment, mettant Arunmor dans une position tr{\`e}s d{\'e}licate."~\\
 --- "Quand tout un chacun pris conscience de ce qui se tramais sur l'AWAN avec la pr{\'e}sence d'uplink il {\'e}tait d{\'e}j{\`a} trop tard pour faire machine arri{\`e}re car de nombreuse multinationales {\'e}taient d{\'e}pendante du syst{\`e}me. " ~\\

\begin{center}
	\includegraphics[width=0.45\textwidth]{img/uplinkProfileFinal.png}
\end{center}

\begin{center}
	\textbf{\Large R{\'e}sistance}
\end{center}

 --- "Pour gagner l'opinion publique, de nombreuses arrestations eurent lieu en 2009 envers les plus grands pirates. Le public pris conscience de l'importance d'Arunmor mais cela poussa aussi Uplink {\`a} devenir une compagnie totalement anonyme. La compagnie effa\c{c}a toutes les donn{\'e}es confidentielles de ses agents, conservant uniquement les noms et profiles. D{\'e}sormais tous les agents sont forc{\'e}s de se connecter {\`a} l'AWAN {\`a} l'aide de passerelles qui s'auto-deconnectent si une trace est en cours, cette passerelle peut m{\^e}me aller jusqu'{\`a} s'auto-detruire pour prot{\'e}ger l'identit{\'e} de l'agent et effacer les preuves. "~\\
 --- "En cons{\'e}quence, Introversion Software devint une soci{\'e}t{\'e} underground et disparu du paysage public, le syst{\`e}me d'exploitation Uplink {\'e}tait distribuer qu'{\`a} quelques agents tri{\'e}s sur le volet par leurs pr{\'e}d{\'e}cesseurs. "~\\
 --- "La population dans sa grande partie n'est m{\^e}me pas consciente de l'existence des op{\'e}ration perp{\'e}tr{\'e}es par Uplink. "~\\
 --- "Voil{\`a} en quoi se r{\'e}sume l'histoire dans laquelle tu va intervenir d{\`e}s aujourd'hui. "~\\
 --- "Voici le disque qui te permettra de te relier {\`a} Uplink via ta passerelle. Tu dois maintenant apprendre par toi m{\^e}me. "~\\
Ainsi je me levais pour quitter le Gentleman Loser et ce nouveau pirate mais j'avais encore une chose {\`a} lui apprendre avant de le lib{\'e}rer totalement, je me retournais et en le fixant du regard je lui dis finalement :~\\
 --- "La confiance est une faiblesse. "~\\
Il se contenta de fermer les yeux et d'opiner en signe de compr{\'e}hension et son sourire me laisse penser qu'il m'{\'e}tait reconnaissant de lui permettre d'entrer dans ce grand champs de bataille qu'est Uplink.~\\
Le temps {\'e}tait venu de tirer un trait sur cet entretien. Chacun doit suivre sa propre voie.~\\

\vfill
\columnbreak

\begin{center}
	\textbf{\Large Pr{\^e}t {\`a} faire partie de l'{\'e}lite ?}
\end{center}

Uplink n'est pas qu'un simple logiciel d'acc{\`e}s, c'est tout un univers souterrain. De nombreux agents ne vivent (et meurs) que par ce biais et j'allais l'apprendre d'ici peu. Tous ces agents pouvaient communiquer entre eux par messagerie et suivre l'{\'e}volution des actions via une rubriques de news, ces informations tant{\^o}t jouissives tant{\^o}t effrayantes pour moi-m{\^e}me et de nombreux agents. Voil{\`a} donc mon nouvel outils de travail, et les offres sont nombreuses malgr{\'e} ce qu'en disent les grandes entreprise, chacune d'entre-elles pr{\'e}f{\'e}rant nier l'emploi d'agents pour effectuer du sabotage ou de l'espionnage.~\\
La premi{\`e}re chose {\`a} apprendre dans le piratage avec Uplink c'est qu'on est loin d'{\^e}tre invisible, et la meilleur chance de s'en tirer lors d'une intrusion consiste a utiliser plusieurs ordinateur pour atteindre sa cible ; cela permet de limiter les risque si la cible trace la connexion pour rep{\'e}rer ma passerelle. ~\\
Les grandes entreprises paient bien pour ce boulot {\`a} risque et pour le fait qu'elle sont alors totalement anonyme, toute erreur d'un agent se r{\'e}percute uniquement sur lui-m{\^e}me, de la simple amende jusqu'au d{\'e}c{\`e}s.~\\
L'erreur n'est pas permise, la strat{\'e}gie est obligatoire.~\\

\begin{center}
	\textbf{\Large L'{\'e}lite}
\end{center}

Dans ce monde on ne peut pas survivre et faire partie de l'{\'e}lite avec du mat{\'e}riel et du logiciel de newbie.~\\
On commence petit avec une prise de risque minimum puis on s'attaque {\`a} de plus gros morceaux qui n{\'e}cessitent plus de s{\'e}curit{\'e} si on ne veux pas finir en flatline.~\\
Les mises {\`a} jour logicielles et mat{\'e}rielles sont n{\'e}cessaires.~\\
On est pas l{\`a} pour faire de l'argent, on est l{\`a} pour faire partie des meilleurs.~\\
Le meilleur agent n'a pas de co{\'e}quipier et c'est quand on est trahi que l'on apprend que la confiance est une faiblesse.~\\

\begin{center}
	\textbf{\Large Conclusion}
\end{center}

Uplink est un jeu inhabituel, tout en 2D mais tr{\`e}s prenant. Il s'agit d'un jeu de strat{\'e}gie dans un univers cyberpunk.~\\
Il ne faut pas chercher des effets tape {\`a} l'oeil ici, attendez vous juste {\`a} rester riv{\'e} {\`a} ce jeu avec un niveau de stress {\'e}lev{\'e}, car c'est l{\`a} l'essence de ce programme, l'action est en temps r{\'e}el, on ne peut que s'en prendre qu'{\`a} soi si la partie se termine pr{\'e}matur{\'e}ment et dans l'univers d'Uplink il n'y a pas de seconde chance.~\\
Uplink est un jeu ax{\'e} sur l'ambiance, le graphisme et le sc{\'e}nario tout autant que la bande son y participent.~\\
Le seul d{\'e}faut de ce jeu {\'e}tant qu'il est int{\'e}gralement en anglais ce qui le restreint aux f{\'e}rus de jeux de strat{\'e}gie et aux fans d'ambiance cyberpunk.~\\
Un jeu {\`a} poss{\'e}der et qui fait d{\'e}j{\`a} parti du panth{\'e}on des jeux de strat{\'e}gie pour ceux qui s'y sont essay{\'e}s.~\\
Site officiel : Uplink [\texttt{http://www.uplink.co.uk/}]~\\

\begin{center}
	\textbf{\Large Configuration minimum}
\end{center}

Un pentium de premi{\`e}re g{\'e}n{\'e}ration avec 32Mo de ram ou son {\'e}quivalent chez Apple.~\\

Si l'univers cyberpunk vous int{\'e}resse je vous conseille la lecture des romans de GIBSON William. \emph{Grav{\'e} sur Chrome} est un recueil de nouvelles excellent pour se faire une id{\'e}e de cet auteur avant de se jeter sur le reste de ce grand ma{\^i}tre du cyberpunk.~\\

\end{multicols*}

\clearpage

\section{A Fan Fiction}

\begin{itemize}
	\setlength{\itemsep}{1pt}
	\setlength{\parskip}{0pt}
	\setlength{\parsep}{0pt}
	%% \small
	\item \texttt{http://www.fanfiction.net/s/4156043/1/I\_call\_it\_Uplink} -- 
	\item \texttt{http://www.fanfiction.net/s/4156043/2/I\_call\_it\_Uplink} -- 
	\item \texttt{http://www.fanfiction.net/s/4156043/3/I\_call\_it\_Uplink} -- 
\end{itemize}

\begin{multicols*}{2}
	\small

\textbf{\large Chapter 1: One of those days} ~\\

It was shaping up to be another one of those days, and Ryo hadn't even had breakfast yet. Of course, that's assuming he wanted breakfast. Sighing, the teenage techie plunged a spoon into the 'slop', as it was called. True, it was a chock-full of beneficial nutrients, proteins, carbohydrates and energy boosting supplements specifically provided to be eaten in a hurry-- but that was only if you could choke it down past your gag reflex. Frowning, Ryo stirred the slop slightly, glaring suspiciously at the little lumps that floated to the surface. ~\\

A tray thunked down next to his, and Ryo glanced up, his eyes meeting those of Hikaru-- the top Exo-force pilot, and self-proclaimed 'brilliant leader'. True, he had good strategy plans, but they wreaked havoc on Ryo's --no, Exo-force's-- battle machines. "Hey Ryo, the \emph{Stealth Hunter}'s moving kind of slow. And it feels... off balance. Maybe the stabilizers need tweaking. You should check it out," Hikaru said, shoveling slop into his mouth rapidly before glancing at him with concern. "What, aren't you hungry?" ~\\

\emph{Tweaking?} Ryo thought, \emph{Tweaking?! Tweaking is what \textbf{you} call it. Fixing a battle machine is a delicate balance of minute adjustments, carefully completed to bring the complex machine back up to it's optimal performance ratio.} \textbf{Tweak}, \emph{indeed}. ~\\

Hikaru snapped his fingers in Ryo's face. "Hello?" He said, "You awake?" ~\\

"What?" Ryo said, suddenly pulled back into the real world. ~\\

"I said the Stealth Hunter needs fixing," Hikaru said, "It's doing strange things." ~\\

"Like what?" Ryo asked, resigning himself to listening. ~\\

"Well, it just doesn't feel right. It slips way too soon when I roll even slightly, and it's engines sound like a blender trying to grind a chunk of cement." ~\\

Ryo frowned. "Did you take it out of first gear?" ~\\

Hikaru frowned. "What gear? It doesn't have any gears." ~\\

Ryo rolled his eyes. "It was just upgraded yesterday. It's almost ready for combat, I was just adding a few finishing touches. First gear isn't meant for flying very high. Speed is third gear. You need a Delta code to activate..." he frowned, realizing the Exo-Force Pilot had stood and was walking away. ~\\

"Change it back!" Hikaru snapped, louder than was necessary. "You're making it too complicated." ~\\

%% \vfill
%% \columnbreak

"I'm making it \emph{practical}!!" Ryo shouted. "Unless you want it to be a useless hunk of scrap metal!!" ~\\

Hikaru spun around. "It was fine how it was!" ~\\

"It was a non-efficient, fuel-wasting, un-maneuverable, inefficient high-class-looking worthless pile of scrap metal!!" Ryo yelled back. "AS IT WAS, STEALTH HUNTER COULD'VE STOOD FOR \emph{Silly Technical Engineering And Lackluster Turbo Hikaru's...um... Hopelessly Useless Never Terminating Evil Robots} BATTLE MACHINE!!" With a snort, Ryo exited the mess hall, door slamming behind him. ~\\

"Nice acronym for on the spot." Hikaru mumbled to Takeshi another pilot, who nodded absently. ~\\

Yes... It was definitely one of those days. ~\\

"Tweak the stabilizer, my foot. The stabilizer's fine. But this power converter... ack! It's practically melted itself to these other wires..." Ryo muttered, tearing components out of the Stealth Hunter. "What is this doing here --it was supposed to be-- he must have tried to fix it on the field... it's ruined!! Arrg!!" Ryo grabbed his hair and pulled... hard. "Ow!" He exclaimed, frowning. "Headaches and ruined battle machines... Stupid thing. Why is it..." Ryo pulled a few more things out. Screaming in frustration, he tore out all of the wires and re-wired the entire thing. By the time he had finished, he'd calmed down enough to know he'd acted like an overgrown toddler. He frowned. "I'll have to apologize to Hikaru, too," he muttered, closing up one of the access panels on the Stealth Hunter. ~\\

Turning, the engineer hopped off of the scaffold and glanced up at the huge battle machine. "One of these days I'll convince Sensei to let me pilot one of these things against the robots," He vowed. ~\\

He heard the door open behind him and turned around. Ha-Ya-To poked his head in. "Hey Ryo," He said, "What's up? I hear you exploded in the mess earlier today." He peeked around. "Have the computers been mean to you lately or something?" ~\\

"Not now," Ryo snapped, residual anger increasing at the reminder of the incident. Resisting the urge to chuck his wrench in the red-haired battle machine pilot's direction, he turned to face Ha-Ya-To. "What do you want?" he snapped, spinning back around to find a component to re-calibrate the structural integrity field in the Stealth Hunter's energy fusion system. ~\\

"Whoa," Ha-Ya-To exclaimed. "The ice king reigns. I'll just come back later." ~\\

"Oh no you don't!" Ryo snapped, turning around, striding across the room and catching Ha-Ya-To by his arm. "You came in here to ask me something, so what is it?!" %% ~\\

Ha-Ya-To cringed. "I was just... well, I was going to ask... if you...well..." ~\\

Ryo sighed. "Never mind. I don't have time for this." he let go, turned and grabbed a coil conduit and then turned back around. "I'm waiting," He said. ~\\

"Um, you know what? I'll ask later," Ha-Ya-To turned and fled from Ryo's lab. After he had left, Ryo happened to glance across the bay, where the hopelessly destroyed gate guardian stood. Smoldering, he turned to the very door Ha-Ya-To had just vanished through. ~\\

"I'm going to kill you!!" Ryo raged, brandishing his lucky wrench very much like a club. ~\\

Yes, it was definitely one of those days... %% ~\\

\begin{center} \rule{0.45\textwidth}{0.01cm} \end{center}

%% \vfill
%% \columnbreak

\textbf{\large Chapter 2: Another One of Those Days} ~\\

Sensei Keiken sighed. He was definitely having one of those days. First, Hikaru and Ha-Ya-To had stumbled into the regeneration station complaining of bruises, headaches and other calamities. After tending to the two pilots, Keiken headed to the Battle Machine hangar to check up on Ryo. --- When he arrived, the first thing the Exo-Force leader had found was the Gate Guardian in a total state of disrepair, and the Stealth Hunter face down on the ground-- tools scattered everywhere. Sighing, Sensei Keiken glanced around, looking for the culprit of these and other disasters. ~\\

"Ryo!" Keiken called, casting his gaze around the hangar bay. Dozens of computers, on and not being used, each completing dozens of different processes and crunching numbers for the brilliant. ~\\

"Over here," Finally, Keiken located a miserable form huddled up behind one of the Stealth Hunter's legs. ~\\

"Where?" Keiken asked, waiting to see if Ryo would tell him where. ~\\

"Behind the Stealth Hunter," Came the dismal reply. Ryo's head peeked out for a moment before sucking back behind the giant battle machine. ~\\

"What happened here?" Keiken asked gently, coming around the leg and staring down at the battle Machine Engineer, doing his best to ignore the tear-streaked cheeks. ~\\

"Stealth Hunter fell over," Ryo said simply, now sounding more frustrated than teary. "Stupid Hikaru! I could kill him if I wasn't more concerned about the preservation of human life." ~\\

"Why?" Keiken asked, trying to figure out exactly how this giant battle machine had managed to topple over. It wasn't particularly top-heavy... ~\\

Frowning, Ryo wondered why on earth Sensei was asking why he wanted to kill Hikaru. Frustrated that Sensei Keiken hadn't listened the first time, Ryo glanced up, eyes snapping angrily. "Because Hikaru's stupid," Ryo repeated. ~\\

Keiken's eyes widened. This problem was more serious than he'd originally thought. Clarifying his question, Sensei Keiken shook his head. "No, Why did Stealth Hunter fall over?" ~\\

"Because I kicked it," Ryo said, frowning and rubbing his booted foot. "Hard." ~\\

"Why did you kick it?" Keiken asked, careful not to allow the question to be misinterpreted. ~\\

"Because I hate my life," Ryo replied. "I hate everybody. I hate you, Sensei." ~\\

Taking a deep breath, Sensei sighed, he wasn't angry at all. However, he was concerned, and slightly sympathetic of the young techie. Although Ryo was a very sensitive young man, Keiken had learned from experience that tough love often worked best with the mechanic. "You sound like a toddler." Keiken scolded, shaking his head. ~\\

"I know," Ryo snapped irritably. "That doesn't mean I care." ~\\

Standing, Sensei Keiken turned and walked away several steps before stopping and turning around. "There's no way to work with you when you're like this, Ryo." ~\\

"I know," Ryo muttered from where he sat, still hunched in the fetal position behind the Stealth Hunter's armor plated leg. ~\\

"What's wrong?" Keiken asked after a long moment of silence. ~\\

"Well..." Ryo sighed gustily, "it's just that-" ~\\

"Hello, anyone home?" Takeshi stuck his green head through the door, glancing around. "Sensei? What are you doing here? Where's Ryo?" ~\\

Ryo jumped to his feet, tools clattering from his lap to the floor. "What in the name of sanity do you want Takeshi? Don't tell me you did something to your battle machine!" Ryo howled, looking ready to tear someone's head off. ~\\

Takeshi brightened at the prospect of working on the Grand Titan. "Well, actually, now that you mention it, I've been meaning to replace the-- -- " ~\\

"JUST SECURE THAT RIMA ORIS BEFORE I MURDER YOU!!" Ryo yelled. ~\\

"Just take it easy," Takeshi suggested. "No harm done. You don't have to work on the Grand Titan if you don't want to. Besides, it looks like you have enough going on with the Stealth Hunter and Gate Guardian." %% ~\\

%% \vfill
%% \columnbreak

It goes without saying that Ryo didn't take to this news very well. In fact, he began flinging the tools scattered on the floor with the sole intent of impaling the Exo-Force pilot. ~\\

"What the-- -- Ack!" Takeshi choked as a screwdriver embedded itself in the wall beside his head. ~\\

Then Ryo charged, swinging his lucky wrench like a club. "LEAVE THE JOB TO THE TECHIE AND STOP MAKING MORE WORK FOR ME!!" He howled, swinging it in the direction of the green-haired pilot. ~\\

Quite aware of the imminent danger, Sensei Keiken grabbed the wrench from behind, which caused an instant stop on Ryo's part. With a cry of surprise, the purple-haired Techie thudded backwards to the floor, landing hard enough to knock the wind out of him. Keiken turned to Takeshi who, with nary a whimper on his part, promptly fainted. ~\\

Keiken sighed and glanced down at Ryo, lying flat on his back and gasping for breath. --- "Try and get some work done. I'll be seeing you soon. This warrants a discussion." ~\\

Ryo had no doubt this 'discussion' would end with him being stuck in the brig for a month. Still gasping for breath, the engineer tried to explain, but Sensei Keiken would have none of it. With a grunt, the Exo-Force leader lifted Takeshi and carried him out of the lab, slamming the door. ~\\

Sensei never slammed doors. ~\\

With a whimper, Ryo rolled over and tucked his knees into his chest, taking small breaths. \emph{Calm down}, the rational side of his brain implored. You're overreacting. \emph{This is pointless, anyway.} The thought didn't seem to help much. Then why did they have to start it?! Ryo wondered angrily. ~\\

Ryo stared in disgust at the Stealth Hunter. "Better get this done," Ryo whined in a mimicry of Hikaru's voice. "After all, Hikaru has practice to do within the hour, and heaven knows we'll need \emph{Hikaru} to show us how \emph{everything}'s done around here," With a sigh, Ryo pried yet another glob of fused wires from the cockpit of the Stealth Hunter. ~\\

Then an idea crept into Ryo's mind. Shaking his head, the Techie tried to push the idea from his mind. \emph{No need for pranks}, he thought. \emph{But he's such a know-it-all}, he thought back at himself. \emph{Besides, this wouldn't hurt anything. It's not dangerous, just... well, sort of annoying to him. Let's see, if I tweak the stabilizers slightly and empty a little bit from the fuel tanks...} ~\\

Grinning, the engineer quickly got back to work-- -- this time in a much better mood... ~\\

When Ryo arrived in Sensei Keiken's quarters, he immediately realized he was the last one to arrive. Grinning sheepishly, he wiped the oil on his hands off of them and onto his already quite dirty uniform. Glancing at Sensei Keiken, the engineer spoke up. "Well, I've finally decided that my tantrum was the result of hormone imbalance and a lack of nutrients." ~\\

Sensei Keiken nodded silently, then cleared his throat. "Don't you think you owe these three something?" He said, gesturing in the direction of three Exo-Force Pilots. ~\\

Frowning in confusion, Ryo thought. \emph{I wonder how much money would be compensation for the injuries... Sensei knows I don't have money! What does he mean, I owe... oh, yeah.} "Sorry," Ryo apologized, glancing down at the ground. In return, the three pilots glared at the techie. Ha-Ya-To with his scratched face, bruised arm and fat lip. Hikaru with his black eye and sore nose. Takeshi with his loss of dignity. ~\\

"Now then, let's get down to business," Sensei Keiken suggested, clearing his throat. "Firstly--" Whatever else he said was cut off by the warning klaxons of a Robot attack. Shaking his head, the Leader changed his tune. "All pilots to your battle machines!" Keiken ordered. ~\\

Ryo's eyes widened in horror. \emph{Oh no! I forgot to change the 'upgrades' back in their battle machines!!} The engineer turned to where Keiken had been standing. "Wait, they aren't-- Sensei?!" But the four others were already gone. Ryo scrambled wildly for the door and dashed out, face paling at the thought of what the pilots were heading out to face... disaster, and an army of robots to boot!! ~\\

Screeching to a halt as he entered the Hangar, Ryo was too late to stop the three pilots. "NO!!" Ryo screamed, "STOP!!" but the roar of dozens of battle machines lifting off drowned out the engineer's cries. %% ~\\

\begin{center} \rule{0.45\textwidth}{0.01cm} \end{center}

%% \vfill
%% \columnbreak

\textbf{\large Chapter 3: I call it Uplink} ~\\

For a moment Ryo could do nothing but watch in frozen horror as the Grand Titan ran out of the hangar leaking joint fluid like water leaked from the pipes under the kitchen sink. Behind it, the Stealth hunter took to the air, leaving most of its armor behind. The Gate Guardian still looked pretty good... until you noticed it's lack of backup power cells and boosters on the back-- these lay on the floor where it had previously been standing. With a shriek that could have deafened a bat, Ryo dropped to his knees. It was just a prank! He thought. This wasn't supposed to happen! ~\\

Ryo stood slowly, then turned in the direction of the communicator station located across the hangar. Quickly he dashed across the cement floor, activated his communicator and tried to contact Hikaru. "Hikaru! You need to get out ot there!" He yelled desperately. "It's not fixed yet!" ~\\

"\emph{We don't all have time to lollygag like \textbf{you} Ryo, }" Hikaru snapped. "\emph{The Stealth Hunter's fine!} " ~\\

"No!! No, you don't under--" Ryo stared in shock. "He cut me off!" He exclaimed. Frowning, he quickly switched the channel and tried to inform Takeshi of his imminent disaster. "You need to come in now or you never will!" Ryo shouted. "You're leaking joint fluid all over the place!" ~\\

\vfill
\columnbreak

"\emph{What?!}" Takeshi shouted, "\emph{I can't hear you over these Thunder Furies' attack. In a minute, Okay?}" ~\\

"You don't have a minute!" Ryo screamed, only to be cut off a second time. Frantically, Ryo tried to contact Ha-Ya-To. Before he could even speak, the Pilot cut him off with horrible news. ~\\

"\emph{Ryo, please! I need your help! The Gate Guardian's out of power and I'm practically surrounded by sentries!}" ~\\

"I know!" Ryo wailed. "Give me a minute to think!" ~\\

"\emph{I don't have a minute!!}" Ha-Ya-To screamed back. "\emph{I've got Iron Drones practically crawling all over me! I need something} now \emph{!!}" ~\\

Panting, Ryo stumbled away from the communicator station in a daze. Well, what am I supposed to do? He thought angrily. \emph{Sensei won't let me leave here, he thinks I'm to valuable back here, and...} Ryo frowned. \emph{But what if he didn't know it was me?} ~\\

Ryo backed up, examining the huge battle machines. He knew the controls of most machines, but they were all so huge... Frowning, the Engineer stepped back and bumped into the leg of another battle machine, hidden under a tarp. Turning, Ryo glanced up at it, a smile spreading across his face... ~\\

By the time Ryo had gotten into the small, orange armor-plated battle machine the situation outside was messier than one of his labs when he was putting together a new project. Exo-Force battle machines were dropping from the sky like flies over Takeshi's cooking. Not that it's bad, but Ryo actually preferred slop to anything Takeshi had been close to with a stove and a spatula. ~\\

Ha-ya-to was crouched behind the remains of the Stealth hunter, desperately firing a hand laser and cringing with every step closer the sentries took. Takeshi's Grand Titan could barely move because of its loss of joint fluid, and the green-Haired pilot was shouting at the control board. As if that could possibly improve his situation! Ryo thought. Glancing around to see where the Stealth Hunter's pilot was, he grimace when he finally located Hikaru trying to fix his Battle Machine. ~\\

Shaking his head, Ryo turned to face the Thunder Fury headed in his direction. With a burst of energy, the Battle Machine exploded. Grinning, Ryo turned to face the sentries, ignoring the call for retreat echoing in every Exo-force Pilot's communicator. After all, he was beginning to enjoy this...with a few blasts of his laser cannon, the sentries were no longer something to worry about. Ignoring the steady stream of pilots heading back to Sentai fortress, Ryo moved forward. ~\\

\vfill
\columnbreak

With an explosion rivaled only by the temper tantrum he'd had earlier in the day, a Fire Vulture plummeted to the ground, never to fly again. Ryo couldn't help shouting in excitement. This was awesome! After taking down several large machines, Ryo once again turned to the Sentries, not realizing he was the only pilot still outside. However, within minutes, the tide had turned. The robots began to retreat and the Exo-Force pilots returned to the battlefield with vigor. ~\\

Together with a few other pilots, Ryo pursued the robots until they were far, far away from Sentai fortress. Then, along with Several other Pilots, he slowly returned to the Hangar, positive he was going to be stuck in the brig for a year, thanks to his shenanigan. Shaking his head, Ryo grinned. It was worth it, He thought. ~\\

Back in the hanger, Dozens of Exo-Force pilots speculated on who had led them to victory. ~\\

"It must have been Hikaru. Only he could use a strategy like that without being afraid of losing," one pilot said. ~\\

"No, only Takeshi would bust in like that," another argued. ~\\

"It can't be either of them," an engineer said from behind them, pointing. "Look, that's both of them over there, isn't it?" ~\\

The pilots stared in amazement. "Then it must have been Ha-Ya-To," one finally stammered. ~\\

"What must have been me?" Ha-Ya-To asked. Then, turning to the engineer, he frowned. "Where's Ryo? I thought he was coming to help me with the Gate guardian. Instead, I had to be saved by Hikaru. Do you know what that does to my self esteem?!" ~\\

One of the pilots glared at him before replying. "Nothing. Your self esteem is indestructible in it's magnificence." ~\\

Just then, the orange Battle Machine walked into the hangar, smoking and leaking joint fluid. ~\\

Hikaru and Takeshi Joined Ha-Ya-To to stare. "Do you think it's Shinji?" Hikaru asked after a moment. "He's pretty good." ~\\

Ha-Ya-To shrugged. "Maybe it's someone else." ~\\

Everyone watched as the cockpit shield lowered and Ryo jumped out. ~\\

Hikaru nearly fell over. "No way!" he gasped. ~\\

"No way!!" Ryo shouted to himself from across the hangar. Because of the absolute silence when he'd gotten out, his voice carried throughout the entire room. "I just finished it today, and it already needs fixing!!" ~\\

\vfill
\columnbreak

After a moment, a few engineers began to applaud, followed by Most of the pilots, with three exceptions. "He sabotaged us," Hikaru whispered. Ha-Ya-To and Takeshi nodded. Mark my words, Techie, Hikaru thought angrily, this isn't the end of the story... ~\\

Through the applause, cries of "Great work!" and "Nice job!" could be heard through the crowd. Ryo had never felt better. He grinned, then frowned anxiously. "Hey, why aren't you telling me what's wrong with your machines?!" he demanded, "Kenji, your poor machine looks like it lost a few anti-combustion nacelles. That's really dangerous," he scolded, dashing across the room to fetch his tools. "I'll be right back!" he cried. ~\\

As the crowd slowly dispersed, Sensei Keiken stepped into the room, and watched Ryo work on a Battle Machine for a few minutes before walking over. "Ryo," he said softly, "Care to explain what happened?" ~\\

Ryo glanced up, sighed and looked at the floor. "After I had to go back and fix Hikaru, Takeshi and Ha-Ya-To's battle machines, I was so frustrated I decided to play a prank on them. Training was in an hour, so I mixed up a few processes..." Ryo sighed. "It was nothing life threatening-- unless you're fighting robots," Ryo admitted. ~\\

Sensei Keiken nodded. "I think you owe them a heartfelt apology, don't you?" ~\\

Ryo nodded. "Yeah." ~\\

Smiling, Sensei Keiken sat down beside the engineer. "Now, if you wouldn't mind answering a few questions?" ~\\

Ryo looked at him. "Sure," he said. "What?" ~\\

"Where did you get the new battle machine?" Sensei asked. ~\\

"Oh, I've been working on it for awhile," Ryo said, "to maybe show you so I could... you know, pilot a battle machine?" ~\\

Sensei nodded. "I see. What are you planning on calling it?" ~\\

"Oh, I already named it," Ryo said proudly, "I call it Uplink." ~\\

\begin{center} \rule{0.45\textwidth}{0.01cm} \end{center}

\end{multicols*}

\clearpage

\chapter{Some Manifesto-s}

\section{The Hacker Manifesto -- La conscience d'un Hacker}

\begin{minipage}[h]{0.95\textwidth}
	\small
	\textbf{\large The Hacker Manifesto} --- by +++The Mentor+++ --- Written January 8, 1986~\\
	
	Another one got caught today, it's all over the papers. "Teenager Arrested in Computer Crime Scandal", "Hacker Arrested after Bank Tampering"... --- Damn kids. They're all alike.~\\
	
	But did you, in your three-piece psychology and 1950's technobrain, ever take a look behind the eyes of the hacker? Did you ever wonder what made him tick, what forces shaped him, what may have molded him? --- I am a hacker, enter my world...~\\
	
	Mine is a world that begins with school... I'm smarter than most of the other kids, this crap they teach us bores me... --- Damn underachiever. They're all alike.~\\
	
	I'm in junior high or high school. I've listened to teachers explain for the fifteenth time how to reduce a fraction. I understand it. "No, Ms. Smith, I didn't show my work. I did it in my head..." --- Damn kid. Probably copied it. They're all alike.~\\
	
	I made a discovery today. I found a computer. Wait a second, this is cool. It does what I want it to. If it makes a mistake, it's because I screwed it up. Not because it doesn't like me... Or feels threatened by me.. Or thinks I'm a smart ass.. Or doesn't like teaching and shouldn't be here... --- Damn kid. All he does is play games. They're all alike.~\\
	
	And then it happened... a door opened to a world... rushing through the phone line like heroin through an addict's veins, an electronic pulse is sent out, a refuge from the day-to-day incompetencies is sought... a board is found. "This is it... this is where I belong..." I know everyone here... even if I've never met them, never talked to them, may never hear from them again... I know you all... --- Damn kid. Tying up the phone line again. They're all alike...~\\
	
	You bet your ass we're all alike... we've been spoon-fed baby food at school when we hungered for steak... the bits of meat that you did let slip through were pre-chewed and tasteless. We've been dominated by sadists, or ignored by the apathetic. The few that had something to teach found us willing pupils, but those few are like drops of water in the desert.~\\
	
	This is our world now... the world of the electron and the switch, the beauty of the baud. We make use of a service already existing without paying for what could be dirt-cheap if it wasn't run by profiteering gluttons, and you call us criminals. We explore... and you call us criminals. We seek after knowledge... and you call us criminals. We exist without skin color, without nationality, without religious bias... and you call us criminals. You build atomic bombs, you wage wars, you murder, cheat, and lie to us and try to make us believe it's for our own good, yet we're the criminals.~\\
	
	Yes, I am a criminal. My crime is that of curiosity. My crime is that of judging people by what they say and think, not what they look like. My crime is that of outsmarting you, something that you will never forgive me for.~\\
	
	I am a hacker, and this is my manifesto. You may stop this individual, but you can't stop us all... after all, we're all alike. %% ~\\ 
\end{minipage}

\clearpage

\textbf{\large La conscience d'un Hacker} --- Loyd Blankenship --- 8 janvier 1986 ~\\
{\scriptsize \emph{La conscience d'un Hacker}, {\'e}galement appel{\'e} le \emph{Manifeste du Hacker}, a {\'e}t{\'e} {\'e}crit par Loyd Blankenship alias The Mentor peu apr{\`e}s son arrestation. } ~\\

Un autre s'est fait prendre aujourd'hui, c'est dans tous les journaux. "Un adolescent arr{\^e}t{\'e} pour un scandaleux crime informatique", "Arrestation d'un hacker apr{\`e}s des falsifications bancaires"... --- Satan{\'e}s gosses, tous les m{\^e}mes. ~\\

Mais avez vous, avec votre psychologie trois pi{\`e}ces et votre profil technocratique de 1950, jamais regard{\'e} derri{\`e}re les yeux d'un hacker ? Ne vous {\^e}tes-vous jamais demand{\'e} ce qui l'avait fait agir, quelles forces lui ont donn{\'e} forme, qu'est-ce qui a bien pu le modeler ainsi ? --- Je suis un hacker, entrez dans mon monde... ~\\
Ce monde commence avec l'{\'e}cole... Je suis plus intelligent que la plupart des autres gamins et ces conneries que l'on nous enseigne m'ennuient... --- Ces fichus {\'e}l{\`e}ves en situation d'{\'e}chec, tous les m{\^e}mes. ~\\

Je suis au coll{\`e}ge ou au lyc{\'e}e. J'ai {\'e}cout{\'e} les enseignants expliquer pour la quinzi{\`e}me fois comment r{\'e}duire une fraction. Je le comprends. "Non, mademoiselle Smith, je ne peux pas vous montrer mon travail, je l'ai fait dans ma t{\^e}te..." --- Satan{\'e} gosse, il l'a s{\^u}rement copi{\'e}. Ils sont tous pareils. ~\\

J'ai fait une d{\'e}couverte aujourd'hui, j'ai trouv{\'e} un ordinateur. Attendez une seconde, ce truc est cool. Il fait ce que je veux qu'il fasse. S'il fait une erreur, c'est parce que je me suis plant{\'e}. Pas parce qu'il ne m'aime pas... ~\\
Ou qu'il se sent menac{\'e} par moi... --- Ou qu'il pense que je suis un petit malin... ~\\
Ou qu'il n'aime pas enseigner et ne devrait pas {\^e}tre l{\`a}... --- Satan{\'e} gosse, il passe son temps {\`a} jouer. Ils sont tous pareils. ~\\

Et c'est alors arriv{\'e}... une porte s'est ouverte sur le monde... Se pr{\'e}cipitant {\`a} travers la ligne de t{\'e}l{\'e}phone telle l'h{\'e}ro{\"i}ne {\`a} travers les veines d'un drogu{\'e}, une impulsion {\'e}lectronique est {\'e}mise, un refuge {\`a} l'incomp{\'e}tence quotidienne... une planche de salut. ~\\

"C'est l{\`a}... c'est {\`a} ce monde que j'appartiens..." --- Je connais tout le monde ici... m{\^e}me si je ne les ai jamais rencontr{\'e}, ne leur ai jamais parl{\'e} et que je n'entendrais peut-{\^e}tre plus jamais parler d'eux... Je vous connais tous... --- Satan{\'e} gosse, encore en train d'encombrer la ligne t{\'e}l{\'e}phonique. Ils sont tous pareils. ~\\

Vous vous r{\'e}p{\'e}tez que nous sommes tous pareils... On nous a nourri aux aliments pour b{\'e}b{\'e} {\`a} l'{\'e}cole alors que nous r{\'e}clamions un steak... les bouts de viande que vous laissiez {\'e}chapper {\'e}taient pr{\'e}m{\^a}ch{\'e}s et sans saveur. Nous avons {\'e}t{\'e} domin{\'e}s pas des sadiques, ou ignor{\'e}s par des apathiques. Les rares qui avaient quelque chose {\`a} nous apprendre trouvaient en nous des {\'e}l{\`e}ves pleins de bonne volont{\'e}, mais ils {\'e}taient aussi nombreux que des gouttes d'eau au milieu du d{\'e}sert. ~\\

C'est notre monde maintenant... un monde de l'{\'e}lectron et du switch, de la beaut{\'e} du baud. Nous utilisons des services sans payer ce qui devrait {\^e}tre {\`a} un prix ridiculement bas s'ils n'{\'e}taient pas administr{\'e}s par des gloutons profiteurs, et vous nous appelez criminels. Nous explorons... et vous nous traitez de criminels. Nous recherchons la connaissance... et vous nous traitez de criminels. Nous existons sans couleur de peau, sans nationalit{\'e}, sans tendance religieuse... et vous nous traitez de criminels. Vous construisez des bombes atomiques, vous financez des guerres, vous assassinez, trichez, nous mentez et essayez de nous faire croire que c'est pour notre bien, mais ce sont nous les criminels. ~\\

Oui, je suis un criminel. Mon crime est ma curiosit{\'e}. Mon crime est de juger les gens pour ce qu'ils disent et pensent et non pas selon leur apparence. Mon crime est d'{\^e}tre plus fut{\'e} que vous, ce que vous ne me pardonnerez jamais. ~\\

Je suis un hacker et ceci est mon manifeste. Vous avez pu arr{\^e}ter cet individu-ci, mais vous ne pouvez nous arr{\^e}ter tous... apr{\`e}s tout, nous sommes tous les m{\^e}mes. ~\\

\clearpage

\section{L'{\'e}thique des hackers -- The Cyberpunk Project}

%% {\LARGE L'{\'e}thique des hackers --- The Cyberpunk Project} ~\\

L'id{\'e}e d'une {\'e}thique des hackers a {\'e}t{\'e} le mieux formul{\'e} par Steven Levy dans son livre \emph{Hackers: Heroes of the Computer Revolution} publi{\'e} en 1984. Levy propose six points : 
\begin{itemize}
	\setlength{\itemsep}{1pt}
	\setlength{\parskip}{0pt}
	\setlength{\parsep}{0pt}
	
	\item L'acc{\`e}s aux ordinateurs, et {\`a} toute chose qui puisse apprendre quoique ce soit sur le fonctionnement du monde, devrait {\^e}tre illimit{\'e}. Il faut toujours en revenir {\`a} l'imp{\'e}ratif du Touche {\`a} Tout !
	\item Toute information devrait {\^e}tre libre.
	\item M{\'e}fiance envers l'autorit{\'e}, promotion de la d{\'e}centralisation.
	\item Les hackers devraient {\^e}tre jug{\'e}s sur leurs comp{\'e}tences, pas des crit{\`e}res fallacieux tels que l'{\^a}ge, l'origine ethnique ou la classe sociale.
	\item L'art et la beaut{\'e} peuvent {\^e}tre cr{\'e}{\'e}s sur ordinateur.
	\item Les ordinateurs peuvent am{\'e}liorer notre vie.
\end{itemize} %% ~\\

L'e-zine Phrack, reconnu comme la newsletter "officielle" p/hacker, a d{\'e}velopp{\'e} son credo {\`a} travers un raisonnement qui se r{\'e}sume en trois principes. 1) Premi{\`e}rement, les hackers refusent l'id{\'e}e que les groupe commerciaux soient les seuls autoris{\'e}s {\`a} acc{\'e}der aux technologies modernes et {\`a} les utiliser. 2) Ensuite, le hacking est une arme majeure dans le combat qui envahit de plus en plus la technologie informatique. 3) Enfin, le co{\^u}t {\'e}lev{\'e} du mat{\'e}riel est au-del{\`a} des moyens de la plupart des hackers, rendant indispensable le hacking et le phreaking pour propager l'informatique vers les masses. %% ~\\

	\emph{\footnotesize "Le hacking. Un loisir {\`a} temps complet : des heures enti{\`e}res {\`a} apprendre, {\`a} exp{\'e}rimenter, avant d'accomplir l'art de p{\'e}n{\'e}trer des r{\'e}seaux d'ordinateurs multi-utilisateurs. Pourquoi les hackers passent une grande partie de leur temps {\`a} pratiquer cet art ? Certains vous diraient que c'est pour satisfaire leur curiosit{\'e} scientifique, d'autres r{\'e}pondraient pour la stimulation mentale. Mais les racines v{\'e}ritables des motivations des hackers sont bien plus profondes. %% ~\\
	--- Dans ce fichier, je vais d{\'e}crire les motivations sous jacentes des hackers, r{\'e}v{\'e}ler les liens entre le hacking, le phreaking, le carding et l'Anarchie, et faire conna{\^i}tre la "r{\'e}volution technologique", qui est la graine originelle dans l'esprit de chaque hacker... Si vous cherchez un tutoriel sur comment accomplir les m{\'e}thodes que je viens de pr{\'e}senter, veuillez lire un fichier de Phrack les concernant. %% ~\\
	--- Et quoique vous fassiez, continuez le combat. Que vous en soyez conscient ou non, si vous {\^e}tes un hacker, vous {\^e}tes un r{\'e}volutionnaire. Ne vous inqui{\'e}tez pas, vous {\^e}tes du bon cot{\'e}."} (Doctor Crash, 1986) ~\\

Bien que les hackers reconnaissent d'eux m{\^e}me que leurs activit{\'e}s puissent parfois {\^e}tre ill{\'e}gales, une emphase consid{\'e}rable est mise sur la limitation de ces transgressions {\`a} seulement ceux qui veulent obtenir un acc{\`e}s et comprendre un syst{\`e}me et ils font preuve d'hostilit{\'e} envers ceux qui transgressent au del{\`a} de ces limites. La plupart des membres exp{\'e}riment{\'e}s de l'informatique underground se m{\'e}fient des jeunes novices qui souvent sont excit{\'e}s par l'image romanesque qu'ils se font du hacking. Les hackers les plus chevronn{\'e}s se plaignent sans cesse des novices qui risquent de se faire appr{\'e}hender et qui peuvent compromettre les comptes {\`a} travers lesquels ils dissimulent leurs acc{\`e}s. --- %% ~\\

En somme, le style du hacker refl{\`e}te un but bien d{\'e}fini : les r{\'e}seaux de communication, un code de valeurs, et un esprit de r{\'e}sistance {\`a} toute forme d'autorit{\'e}. Parce que le hacking requiert un ensemble de connaissance plus grand que le phreaking, et parce que cette connaissance ne peut s'acqu{\'e}rir que par l'exp{\'e}rience, les hackers ont tendance {\`a} {\^e}tre plus vieux et plus savant que les phreakers. De plus, {\`a} quelques exceptions pr{\`e}s, les buts de ces deux groupes sont assez {\'e}loign{\'e}s. En r{\'e}sum{\'e}, chaque groupe constitue une cat{\'e}gorie d'analyse bien distincte. ~\\

Ce qui suit provient de Richard Stallman, arriv{\'e} au laboratoire sur l'intelligence artificielle du MIT en 1971, vers la fin de l'explosion du hacking des ann{\'e}es soixante. Il est connu notamment pour avoir {\'e}crit la matrice de tous les logiciels libres, l'{\'e}diteur de texte Emacs. %% ~\\

	\emph{\footnotesize "Je ne sais pas s'il existe aujourd'hui une {\'e}thique du hacker en tant que telle, mais il existait bien une {\'e}thique au labo sur l'IA du MIT. On refusait que la bureaucratie nous emp{\^e}che de faire des choses utiles. On ne se pr{\'e}occupait pas des r{\`e}gles, on se pr{\'e}occupait seulement des r{\'e}sultats. On ne respectait absolument pas les r{\`e}glements, en mati{\`e}re de s{\'e}curit{\'e} informatique, ou de serrures sur les portes. On {\'e}tait fier de la vitesse avec laquelle on balayait la moindre petite parcelle de bureaucratie qui s'ing{\'e}rait, aussi n{\'e}gligeable la quantit{\'e} de temps qu'elle vous faisait perdre pouvait {\^e}tre. N'importe qui osant boucler un terminal dans son bureau, disons un prof qui se sentait plus important que d'autres, la voyait grande ouverte le lendemain matin. Je n'avais qu'{\`a} grimper par dessus le plafond, ou me glisser sous le sol, sortir le terminal, ou laisser la porte ouverte avec un message disant quelle g{\^e}ne cela pouvait {\^e}tre de devoir passer sous le sol, "donc veuillez ne plus g{\^e}ner les usagers en fermant la porte, et laissez la ouverte". Aujourd'hui encore, au labo on trouve une {\'e}norme cl{\'e} anglaise qu'on appelle "la cl{\'e} principale du 7{\`e}me {\'e}tage", {\`a} utiliser si quelqu'un {\`a} l'outrecuidance d'enfermer un de ces satan{\'e}s terminaux."} %% ~\\

\clearpage

\section{A Cyberspace Independence Declaration}

\begin{minipage}[ht]{1.00\textwidth}
	\small
	\texttt{http://www.telix.com/delta/employees/range/indep.txt} ~\\ 
	
	Date: Fri, 9 Feb 1996 17:16:35 +0100 ~\\
	From: John Perry Barlow <barlow@eff.org> ~\\
	Subject: File 5--A Cyberspace Independence Declaration ~\\
	
	Yesterday, that great invertebrate in the White House signed into the law the Telecom "Reform" Act of 1996, while Tipper Gore took digital photographs of the proceedings to be included in a book called "24 Hours in Cyberspace." ~\\
	
	I had also been asked to participate in the creation of this book by writing something appropriate to the moment. Given the atrocity that this legislation would seek to inflict on the Net, I decided it was as good a time as any to dump some tea in the virtual harbor. ~\\
	
	After all, the Telecom "Reform" Act, passed in the Senate with only 5 dissenting votes, makes it unlawful, and punishable by a \$250,000 to say "shit" online. Or, for that matter, to say any of the other 7 dirty words prohibited in broadcast media. Or to discuss abortion openly. Or to talk about any bodily function in any but the most clinical terms. ~\\
	
	It attempts to place more restrictive constraints on the conversation in Cyberspace than presently exist in the Senate cafeteria, where I have dined and heard colorful indecencies spoken by United States senators on every occasion I did. ~\\
	
	This bill was enacted upon us by people who haven't the slightest idea who we are or where our conversation is being conducted. It is, as my good friend and Wired Editor Louis Rossetto put it, as though "the illiterate could tell you what to read." ~\\
	
	Well, fuck them. ~\\
	
	Or, more to the point, let us now take our leave of them. They have declared war on Cyberspace. Let us show them how cunning, baffling, and powerful we can be in our own defense. ~\\
	
	I have written something (with characteristic grandiosity) that I hope will become one of many means to this end. If you find it useful, I hope you will pass it on as widely as possible. You can leave my name off it if you like, because I don't care about the credit. I really don't. ~\\
	
	But I do hope this cry will echo across Cyberspace, changing and growing and self-replicating, until it becomes a great shout equal to the idiocy they have just inflicted upon us. ~\\
	
	I give you... ~\\
	
	\emph{[see next page]} ~\\
	
	**************************************************************** ~\\
	John Perry Barlow, Cognitive Dissident ~\\
	Co-Founder, Electronic Frontier Foundation ~\\
	
	Home(stead) Page: http://www.eff.org/\textasciitilde barlow ~\\
	
	Message Service: 800/634-3542 ~\\
	
	Barlow in Meatspace Today (until Feb 12): Cannes, France ~\\
	Hotel Martinez: (33) 92 98 73 00, Fax: (33) 93 39 67 82 ~\\
	
	Coming soon to: Amsterdam 2/13-14, Winston-Salem 2/15, San Francisco ~\\
	2/16-20, San Jose 2/21, San Francisco 2/21-23, Pinedale, Wyoming ~\\
	
	In Memoriam, Dr. Cynthia Horner and Jerry Garcia ~\\
	
	***************************************************************** ~\\
	
	It is error alone which needs the support of government.  Truth can stand by itself. ~\\
							-- -- Thomas Jefferson, Notes on Virginia
\end{minipage} ~\\
                         
%% \clearpage

\begin{center} \begin{minipage}[ht]{0.95\textwidth}
	\footnotesize
	\textbf{\large A Declaration of the Independence of Cyberspace} %% ~\\
	
	Governments of the Industrial World, you weary giants of flesh and steel, I come from Cyberspace, the new home of Mind. On behalf of the future, I ask you of the past to leave us alone. You are not welcome among us. You have no sovereignty where we gather. ~\\
	
	We have no elected government, nor are we likely to have one, so I address you with no greater authority than that with which liberty itself always speaks. I declare the global social space we are building to be naturally independent of the tyrannies you seek to impose on us. You have no moral right to rule us nor do you possess any methods of enforcement we have true reason to fear. ~\\
	
	Governments derive their just powers from the consent of the governed. You have neither solicited nor received ours. We did not invite you. You do not know us, nor do  you know our world. Cyberspace does not lie within your borders. Do not think that you can build it, as though it were a public construction project. You cannot. It is an act of nature and it grows itself through our collective actions. ~\\
	
	You have not engaged in our great and gathering conversation, nor did you create the wealth of our marketplaces. You do not know our culture, our ethics, or the unwritten codes that already provide our society more order than could be obtained by any of your impositions. ~\\
	
	You claim there are problems among us that you need to solve. You use this claim as an excuse to invade our precincts. Many of these problems don't exist. Where there are real conflicts, where there are wrongs, we will identify them and address them by our means. We are forming our own Social Contract . This governance will arise according to the conditions of our world, not yours. Our world is different. ~\\
	
	Cyberspace consists of transactions, relationships, and thought itself, arrayed like a standing wave in the web of our communications.  Ours is a world that is both everywhere and nowhere, but it is not where bodies live. ~\\
	
	We are creating a world that all may enter without privilege or prejudice accorded by race, economic power, military force, or station of birth. ~\\
	
	We are creating a world where anyone, anywhere may express his or her beliefs, no matter how singular, without fear of being coerced into silence or conformity. ~\\
	
	Your legal concepts of property, expression, identity, movement, and context do not apply to us. They are based on matter, There is no matter here. ~\\
	
	Our identities have no bodies, so, unlike you, we cannot obtain order by physical coercion. We believe that from ethics, enlightened self-interest, and the commonweal, our governance will emerge . Our identities may be distributed across many of your jurisdictions. The only law that all our constituent cultures would generally recognize is the Golden Rule. We hope we will be able to build our particular solutions on that basis.  But we cannot accept the solutions you are attempting to impose. ~\\
	
	In the United States, you have today created a law, the Telecommunications Reform Act, which repudiates your own Constitution and insults the dreams of Jefferson, Washington, Mill, Madison, DeToqueville, and Brandeis. These dreams must now be born anew in us. ~\\
	
	You are terrified of your own children, since they are natives in a world where you will always be immigrants. Because you fear them, you entrust your bureaucracies with the parental responsibilities you are too cowardly to confront yourselves. In our world, all the sentiments and expressions of humanity, from the debasing to the angelic, are parts of a seamless whole, the global conversation of bits. We cannot separate the air that chokes from the air upon which wings beat. ~\\
	
	In China, Germany, France, Russia, Singapore, Italy and the United States, you are trying to ward off the virus of liberty by erecting guard posts at the frontiers of Cyberspace. These may keep out the contagion for a small time, but they will not work in a world that will soon be blanketed in bit-bearing media. ~\\
	
	Your increasingly obsolete information industries would perpetuate themselves by proposing laws, in America and elsewhere, that claim to own speech itself throughout the world. These laws would declare ideas to be another industrial product, no more noble than pig iron. In our world, whatever the human mind may create can be reproduced and distributed infinitely at no cost. The global conveyance of thought no longer requires your factories to accomplish. ~\\
	
	These increasingly hostile and colonial measures place us in the same position as those previous lovers of freedom and self-determination who had to reject the authorities of distant, uninformed powers. We must declare our virtual selves immune to your sovereignty, even as we continue to consent to your rule over our bodies. We will spread ourselves across the Planet so that no one can arrest our thoughts. ~\\
	
	We will create a civilization of the Mind in Cyberspace. May it be more humane and fair than the world your governments have made before. ~\\
	
	Davos, Switzerland --- February 8, 1996 ~\\
\end{minipage} \end{center}

%% \clearpage

\begin{center} \begin{minipage}[ht]{0.95\textwidth}
	\footnotesize
	\textbf{\Large D{\'e}claration d'Ind{\'e}pendance du Cyberespace} --- John Perry Barlow --- 8 f{\'e}vrier 1996 ~\\
	
	Gouvernements du monde industriel, g{\'e}ants fatigu{\'e}s de chair et d'acier, je viens du Cyberespace, nouvelle demeure de l'Esprit. Au nom de l'avenir, je vous demande, {\`a} vous qui {\^e}tes du pass{\'e}, de nous laisser tranquilles. Vous n'{\^e}tes pas les bienvenus parmi nous. Vous n'avez aucun droit de souverainet{\'e} sur nos lieux de rencontre. ~\\
	
	Nous n'avons pas de gouvernement {\'e}lu et nous ne sommes pas pr{\`e}s d'en avoir un, aussi je m'adresse {\`a} vous avec la seule autorit{\'e} que donne la libert{\'e} elle-m{\^e}me lorsqu'elle s'exprime. Je d{\'e}clare que l'espace social global que nous construisons est ind{\'e}pendant, par nature, de la tyrannie que vous cherchez {\`a} nous imposer. Vous n'avez pas le droit moral de nous donner des ordres et vous ne disposez d'aucun moyen de contrainte que nous ayons de vraies raisons de craindre. ~\\
	
	Les gouvernements tirent leur pouvoir l{\'e}gitime du consentement des gouvern{\'e}s. Vous ne nous l'avez pas demand{\'e} et nous ne vous l'avons pas donn{\'e}. Vous n'avez pas {\'e}t{\'e} convi{\'e}s. Vous ne nous connaissez pas et vous ignorez tout de notre monde. Le Cyberespace n'est pas born{\'e} par vos fronti{\`e}res. Ne croyez pas que vous puissiez le construire, comme s'il s'agissait d'un projet de construction publique. Vous ne le pouvez pas. C'est un acte de la nature et il se d{\'e}veloppe gr{\^a}ce {\`a} nos actions collectives. ~\\
	
	Vous n'avez pas pris part {\`a} notre grande conversation, qui ne cesse de cro{\^i}tre, et vous n'avez pas cr{\'e}{\'e} la richesse de nos march{\'e}s. Vous ne connaissez ni notre culture, ni notre {\'e}thique, ni les codes non {\'e}crits qui font d{\'e}j{\`a} de notre soci{\'e}t{\'e} un monde plus ordonn{\'e} que celui que vous pourriez obtenir en imposant toutes vos r{\`e}gles. ~\\
	
	Vous pr{\'e}tendez que des probl{\`e}mes se posent parmi nous et qu'il est n{\'e}cessaire que vous les r{\'e}gliez. Vous utilisez ce pr{\'e}texte pour envahir notre territoire. Nombre de ces probl{\`e}mes n'ont aucune existence. Lorsque de v{\'e}ritables conflits se produiront, lorsque des erreurs seront commises, nous les identifierons et nous les r{\'e}glerons par nos propres moyens. Nous {\'e}tablissons notre propre contrat social. L'autorit{\'e} y sera d{\'e}finie selon les conditions de notre monde et non du v{\^o}tre. Notre monde est diff{\'e}rent. ~\\
	
	Le Cyberespace est constitu{\'e} par des {\'e}changes, des relations, et par la pens{\'e}e elle-m{\^e}me, d{\'e}ploy{\'e}e comme une vague qui s'{\'e}l{\`e}ve dans le r{\'e}seau de nos communications. Notre monde est {\`a} la fois partout et nulle part, mais il n'est pas l{\`a} o{\`u} vivent les corps. ~\\
	
	Nous cr{\'e}ons un monde o{\`u} tous peuvent entrer, sans privil{\`e}ge ni pr{\'e}jug{\'e} dict{\'e} par la race, le pouvoir {\'e}conomique, la puissance militaire ou le lieu de naissance. --- Nous cr{\'e}ons un monde o{\`u} chacun, o{\`u} qu'il se trouve, peut exprimer ses id{\'e}es, aussi singuli{\`e}res qu'elles puissent {\^e}tre, sans craindre d'{\^e}tre r{\'e}duit au silence ou {\`a} une norme. ~\\
	
	Vos notions juridiques de propri{\'e}t{\'e}, d'expression, d'identit{\'e}, de mouvement et de contexte ne s'appliquent pas {\`a} nous. Elles se fondent sur la mati{\`e}re. Ici, il n'y a pas de mati{\`e}re. ~\\
	
	Nos identit{\'e}s n'ont pas de corps; ainsi, contrairement {\`a} vous, nous ne pouvons obtenir l'ordre par la contrainte physique. Nous croyons que l'autorit{\'e} na{\^i}tra parmi nous de l'{\'e}thique, de l'int{\'e}r{\^e}t individuel {\'e}clair{\'e} et du bien public. Nos identit{\'e}s peuvent {\^e}tre r{\'e}parties sur un grand nombre de vos juridictions. La seule loi que toutes les cultures qui nous constituent s'accordent {\`a} reconna{\^i}tre de fa\c{c}on g{\'e}n{\'e}rale est la R{\`e}gle d'Or. Nous esp{\'e}rons que nous serons capables d'{\'e}laborer nos solutions particuli{\`e}res sur cette base. Mais nous ne pouvons pas accepter les solutions que vous tentez de nous imposer. ~\\
	
	Aux {\'E}tats-Unis, vous avez aujourd'hui cr{\'e}{\'e} une loi, la loi sur la r{\'e}forme des t{\'e}l{\'e}communications, qui viole votre propre Constitution et repr{\'e}sente une insulte aux r{\^e}ves de Jefferson, Washington, Mill, Madison, Tocqueville et Brandeis. Ces r{\^e}ves doivent d{\'e}sormais rena{\^i}tre en nous. ~\\
	
	Vous {\^e}tes terrifi{\'e}s par vos propres enfants, parce qu'ils sont les habitants d'un monde o{\`u} vous ne serez jamais que des {\'e}trangers. Parce que vous les craignez, vous confiez la responsabilit{\'e} parentale, que vous {\^e}tes trop l{\^a}ches pour prendre en charge vous-m{\^e}mes, {\`a} vos bureaucraties. Dans notre monde, tous les sentiments, toutes les expressions de l'humanit{\'e}, des plus vils aux plus ang{\'e}liques, font partie d'un ensemble homog{\`e}ne, la conversation globale informatique. Nous ne pouvons pas s{\'e}parer l'air qui suffoque de l'air dans lequel battent les ailes. ~\\
	
	En Chine, en Allemagne, en France, en Russie, {\`a} Singapour, en Italie et aux {\'E}tats-Unis, vous vous efforcez de repousser le virus de la libert{\'e} en {\'e}rigeant des postes de garde aux fronti{\`e}res du Cyberespace. Ils peuvent vous pr{\'e}server de la contagion pendant quelque temps, mais ils n'auront aucune efficacit{\'e} dans un monde qui sera bient{\^o}t couvert de m{\'e}dias informatiques. ~\\
	
	Vos industries de l'information toujours plus obsol{\`e}tes voudraient se perp{\'e}tuer en proposant des lois, en Am{\'e}rique et ailleurs, qui pr{\'e}tendent d{\'e}finir des droits de propri{\'e}t{\'e} sur la parole elle-m{\^e}me dans le monde entier. Ces lois voudraient faire des id{\'e}es un produit industriel quelconque, sans plus de noblesse qu'un morceau de fonte. Dans notre monde, tout ce que l'esprit humain est capable de cr{\'e}er peut {\^e}tre reproduit et diffus{\'e} {\`a} l'infini sans que cela ne co{\^u}te rien. La transmission globale de la pens{\'e}e n'a plus besoin de vos usines pour s'accomplir. ~\\
	
	Ces mesures toujours plus hostiles et colonialistes nous mettent dans une situation identique {\`a} celle qu'ont connue autrefois les amis de la libert{\'e} et de l'autod{\'e}termination, qui ont eu {\`a} rejeter l'autorit{\'e} de pouvoirs distants et mal inform{\'e}s. Nous devons d{\'e}clarer nos subjectivit{\'e}s virtuelles {\'e}trang{\`e}res {\`a} votre souverainet{\'e}, m{\^e}me si nous continuons {\`a} consentir {\`a} ce que vous ayez le pouvoir sur nos corps. Nous nous r{\'e}pandrons sur la plan{\`e}te, si bien que personne ne pourra arr{\^e}ter nos pens{\'e}es. ~\\
	
	Nous allons cr{\'e}er une civilisation de l'Esprit dans le Cyberespace. Puisse-t-elle {\^e}tre plus humaine et plus juste que le monde que vos gouvernements ont cr{\'e}{\'e}. ~\\
\end{minipage} \end{center}

\clearpage

\section{Manifestes Cyberpunk}

\textbf{\Large Un Manifeste Cyberpunk} --- Christian As. Kirtchev --- 14 f{\'e}vrier 1997 %% ~\\

\emph{\footnotesize Nous sommes des esprits {\'e}lectroniques, un groupe de rebelles libres penseurs, des Cyberpunks. Nous vivons dans le Cyberespace, nous sommes partout, nous ne connaissons pas de fronti{\`e}res. Ceci est notre manifeste, le manifeste des Cyberpunks. } %% ~\\

%% \setlength\parindent{15pt}
\begin{enumerate}
		\setlength{\itemsep}{1pt}
		\setlength{\parskip}{0pt}
		\setlength{\parsep}{0pt}
		
	\item[I.] Cyberpunk
	\begin{enumerate}
		\setlength{\itemsep}{1pt}
		\setlength{\parskip}{0pt}
		\setlength{\parsep}{0pt}
		
		\item[1/] Nous sommes l'{\^e}tre Diff{\'e}rent. Rats technologiques, nageant dans l'oc{\'e}an de l'information.
		\item[2/] Nous sommes l'effac{\'e}, le petit gar\c{c}on qui s'asseyait {\`a} la derni{\`e}re table dans un coin de la classe.
		\item[3/] Nous sommes l'adolescent consid{\'e}r{\'e} comme bizarre par tout le monde.
		\item[4/] Nous sommes l'{\'e}tudiant qui hacke des syst{\`e}mes informatiques et en explore les tr{\'e}fonds.
		\item[5/] Nous sommes l'adulte assit sur le banc d'un parc, le portable sur les genoux, programmant la derni{\`e}re r{\'e}alit{\'e} virtuelle.
		\item[6/] Ce garage farci d'{\'e}lectronique est {\`a} nous. Le fer {\`a} souder au coin du bureau et la radio d{\'e}mont{\'e}e {\`a} proximit{\'e} sont {\'e}galement les n{\^o}tres. De m{\^e}me que cette cave remplie d'ordinateurs, des bourdonnements d'imprimantes et des bips de modems.
		\item[7/] Nous sommes ceux qui per\c{c}oivent la r{\'e}alit{\'e} de fa\c{c}on diff{\'e}rente. Notre point de vue nous permet de voir au del{\`a} de ce que les gens ordinaires peuvent. L{\`a} o{\`u} ils n'entrevoient que l'aspect ext{\'e}rieur, nous d{\'e}celons ce qui est {\`a} l'int{\'e}rieur. C'est ce que nous sommes, des r{\'e}alistes avec des lunettes de r{\^e}veurs.
		\item[8/] Nous sommes ces {\^e}tres {\'e}tranges, pratiquement inconnus du voisinage. Ces {\^e}tres accord{\'e}s {\`a} leurs propres pens{\'e}es, demeurant jour apr{\`e}s jour devant leurs ordinateurs, fouillant le net {\`a} la recherche de quelque chose. Nous ne sortons pas souvent de chez nous, juste de temps en temps, et uniquement pour aller au plus proche magasin d'{\'e}lectronique, au bar habituel pour rencontrer le peu d'amis que nous avons, ou un client, ou un refourgueur... ou juste pour une petite promenade.
		\item[9/] Nous n'avons pas beaucoup d'amis, seul un petit nombre avec lesquels nous allons {\`a} des f{\^e}tes. Tous les autres, nous les connaissons gr{\^a}ce au net. Nos v{\'e}ritables amis sont l{\`a}, de l'autre c{\^o}t{\'e} de la ligne. Nous les connaissons {\`a} travers nos channels IRC favoris, nos newsgroups, des syst{\`e}mes autour desquels nous r{\^o}dons.
		\item[10/] Nous sommes ceux qui n'avons rien {\`a} foutre de ce que les gens pensent de nous, d'{\`a} quoi nous ressemblons ou de ce que peuvent baver les gens dans notre dos.
		\item[11/] La majorit{\'e} d'entre nous pr{\'e}f{\`e}re vivre terr{\'e}, inconnus de tous except{\'e} de ceux dont nous ne pouvons {\'e}viter le contact.
		\item[12/] Les autres pr{\'e}f{\`e}rent la publicit{\'e}, ils aiment la c{\'e}l{\'e}brit{\'e}. Ils sont tous connus du monde underground, leurs noms y sont souvent entendus. Mais nous, nous sommes tous unis par une chose, nous sommes Cyberpunks.
		\item[13/] La Soci{\'e}t{\'e} ne nous comprend pas, aux yeux des gens ordinaires qui vivent loin de l'information et des id{\'e}es libres nous sommes des {\^e}tres "bizarres" et "insens{\'e}s". La Soci{\'e}t{\'e} nie notre fa\c{c}on de penser, une soci{\'e}t{\'e} qui vie, pense et respire d'une et d'une seule mani{\`e}re, un clich{\'e}.
		\item[14/] Ils nous renient car nous pensons comme des hommes libre or la libre pens{\'e}e est interdite.
		\item[15/] Le Cyberpunk n'a pas d'apparence ext{\'e}rieure, de signes particuliers. Les Cyberpunks sont des gens d'aspect ordinaire, connus de personne, de l'artiste technomaniaque, au musicien jouant de la musique {\'e}lectronique en passant par l'{\'e}rudit superficiel.
		\item[16/] Le Cyberpunk n'est plus un genre litt{\'e}raire, pas m{\^e}me une simple sous culture. Le Cyberpunk est une culture {\`a} part enti{\`e}re, la prog{\'e}niture d'un nouvel {\^a}ge. Une culture qui unit nos int{\'e}r{\^e}ts communs et nos points de vue. Nous sommes un groupe. Nous sommes Cyberpunks.
	\end{enumerate}
	\item[II.] Soci{\'e}t{\'e}
	\begin{enumerate}
		\setlength{\itemsep}{1pt}
		\setlength{\parskip}{0pt}
		\setlength{\parsep}{0pt}
		
		\item[1/] La Soci{\'e}t{\'e} qui nous entoure est entrav{\'e}e par sa volont{\'e} de ramener tout et tout le monde {\`a} elle, alors qu'elle s'enfonce lentement dans les sables du temps.
		\item[2/] M{\^e}me si certains s'obstinent {\`a} ne pas le croire, il est {\'e}vident que nous vivons dans une soci{\'e}t{\'e} malade. Les soi-disantes r{\'e}formes dont nos gouvernements aiment tant se vanter, ne sont qu'un petit pas en avant, quand un saut d{\'e}finitif peut {\^e}tre fait.
		\item[3/] Les gens ont peur de la nouveaut{\'e} et de l'inconnu, ils pr{\'e}f{\`e}rent l'ancien, le connu et les v{\'e}rit{\'e}s v{\'e}rifi{\'e}es. Ils sont effray{\'e}s des apports du changement. Ils craignent de perdre ce qu'ils ont.
		\item[4/] Leur peur est si forte qu'ils ont d{\'e}clar{\'e} les choses r{\'e}volutionnaires des ennemis et les id{\'e}es libres ses armes. C'est leur d{\'e}faut.
		\item[5/] Les gens doivent abandonner cette peur derri{\`e}re eux et aller de l'avant. Quel est l'int{\'e}r{\^e}t de s'en tenir {\`a} peu quand vous pouvez avoir beaucoup plus demain. Tout ce qu'ils ont {\`a} faire est de tendre leurs mains et de ressentir le renouveau, d'accorder la libert{\'e} aux pens{\'e}es, aux id{\'e}es et aux mots.
		\item[6/] Voil{\`a} des si{\`e}cles que chaque g{\'e}n{\'e}ration a {\'e}t{\'e} {\'e}lev{\'e} selon le m{\^e}me sch{\'e}ma. Les id{\'e}aux sont suivis par tout le monde, l'individualit{\'e} est ignor{\'e}e. Les gens pensent de la m{\^e}me fa\c{c}on, conform{\'e}ment aux clich{\'e}s qu'on leur a impr{\'e}gn{\'e}s dans leur jeunesse, la m{\^e}me {\'e}ducation st{\'e}r{\'e}otyp{\'e}e pour tous les enfants. Et lorsque quelqu'un ose d{\'e}fier l'autorit{\'e}, il est puni et exhib{\'e} aux yeux de tous comme l'exemple {\`a} ne pas suivre. "Voil{\`a} ce qu'il arrive quand vous exprimez votre propre opinion et que vous contestez celle de votre professeur".
		\item[7/] Notre soci{\'e}t{\'e} est malade et a besoin d'{\^e}tre soign{\'e}e. Le rem{\`e}de est un changement dans le syst{\`e}me...
	\end{enumerate}
	\item[III.] Le Syst{\`e}me
	\begin{enumerate}
		\setlength{\itemsep}{1pt}
		\setlength{\parskip}{0pt}
		\setlength{\parsep}{0pt}
		
		\item[1/] Le Syst{\`e}me. Vieux de plusieurs si{\`e}cles, existant sur des principes qui ne p{\`e}sent plus rien aujourd'hui. Un Syst{\`e}me qui n'a pas chang{\'e} depuis le jour de sa naissance.
		\item[2/] Le Syst{\`e}me est aberrant.
		\item[3/] Le Syst{\`e}me doit imposer sa v{\'e}rit{\'e} pour pouvoir r{\'e}gner. Les gouvernements ont besoin que nous les suivions aveuglement. Pour cette raison, nous vivons dans une {\'e}clipse informationnelle. Lorsque des gens acquirent des informations autres que celles venant du gouvernement, ils ne peuvent distinguer le vrai du faux. Et donc le mensonge devient une v{\'e}rit{\'e}, une v{\'e}rit{\'e}fondamentale pour tout le reste. Ainsi les puissants contr{\^o}lent avec des mensonges et les gens ordinaires n'ont aucune notion de ce qui est av{\'e}r{\'e} et suivent le gouvernement aveuglement, ayant confiance en lui.
		\item[4/] Nous combattons pour la libert{\'e} de l'information. Nous combattons pour la libert{\'e} d'expression et de la presse. Pour la libert{\'e} d'{\'e}mettre nos pens{\'e}es ouvertement, sans {\^e}tre pers{\'e}cut{\'e} par le syst{\`e}me.
		\item[5/] M{\^e}me dans les pays les plus d{\'e}velopp{\'e}s et "d{\'e}mocratiques", le syst{\`e}me impose la d{\'e}sinformation. M{\^e}me dans les pays qui ont la pr{\'e}tention d'{\^e}tre le berceau de la libert{\'e} d'expression. La d{\'e}sinformation est l'une des armes principale du syst{\`e}me. Une arme dont ils sont pass{\'e}s ma{\^i}tre.
		\item[6/] C'est le Net qui nous permet de propager l'information librement, sans limite de fronti{\`e}res.
		\item[7/] Les n{\^o}tre sont les v{\^o}tres. Les v{\^o}tres sont les n{\^o}tres.
		\item[8/] Tout le monde peut partager l'information, sans restrictions.
		\item[9/] L'encryption de l'information est notre arme. Ainsi, les mots de la r{\'e}volution peuvent se propager sans interruption, et les gouvernements essayer seulement de deviner.
		\item[10/] Le Net est notre royaume, nous y sommes Rois.
		\item[11/] Les lois. Le monde change et les lois restent les m{\^e}mes. Le Syst{\`e}me est immuable, seuls quelques d{\'e}tails sont rectifi{\'e}s afin de repousser l'{\'e}ch{\'e}ance, mais les concepts sont conserv{\'e}s {\`a} l'identique.
		\item[12/] Nous avons besoin de nouvelles lois. Des lois en ad{\'e}quations avec l'{\'e}poque et le monde dans lesquels nous vivons et non pas des lois {\'e}difi{\'e}es sur les bases du pass{\'e}. Des lois cr{\'e}{\'e}es pour aujourd'hui et qui seront toujours valables demain.
		\item[13/] Ce sont uniquement ces lois qui nous retiennent, des lois qui ont grandement besoin de corrections.
	\end{enumerate}
	\item[IV.] La vision
	\begin{enumerate}
		\setlength{\itemsep}{1pt}
		\setlength{\parskip}{0pt}
		\setlength{\parsep}{0pt}
		
		\item[1/] Quelques uns ne se soucient gu{\`e}re de ce qui se passe au niveau global. Ils se soucient de ce qui se passe autour d'eux, dans leurs micro-univers.
		\item[2/] Ces personnes peuvent uniquement concevoir un sombre futur, car ils ne consid{\`e}rent que la vie qu'ils vivent actuellement.
		\item[3/] Les autres montrent certains signes de pr{\'e}occupation pour les affaires globales. Ils s'int{\'e}ressent {\`a} tout, aux perspectives d'avenir, {\`a} ce qui va se d{\'e}rouler au niveau global.
		\item[4/] Ils ont une vision plus optimiste. Pour eux le futur est plus propre et plus beau, car de ce qu'ils peuvent en voir, l'homme sera plus mature et le monde plus sage.
		\item[5/] Nous sommes au milieu. Nous nous int{\'e}ressons {\`a} ce qu'il se produit maintenant autant qu'{\`a} ce qu'il va advenir demain.
		\item[6/] Nous observons le Net et le net ne cesse de grossir et d'acqu{\'e}rir du savoir.
		\item[7/] Bient{\^o}t tout sera absorb{\'e} par la net, des syst{\`e}mes militaires aux PC {\`a} la maison.
		\item[8/] Mais le net est le royaume de l'anarchie.
		\item[9/] Il ne peut {\^e}tre contr{\^o}l{\'e} et c'est l{\`a} son pouvoir.
		\item[10/] Chaque {\^e}tre sera tributaire du Net.
		\item[11/] L'int{\'e}gralit{\'e} de l'information y sera, enferm{\'e}e dans un abysse de z{\'e}ros et de uns.
		\item[12/] Celui qui contr{\^o}le le Net, contr{\^o}le l'information.
		\item[13/] Nous vivrons dans un m{\'e}lange de pass{\'e} et de pr{\'e}sent.
		\item[14/] Le mauvais vient de l'homme et le bon de la technologie.
		\item[15/] Le Net contr{\^o}lera le petit homme et nous contr{\^o}lerons le net.
		\item[16/] Et ce que vous ne contr{\^o}lerez pas, vous contr{\^o}lera.
		\item[17/] L'Information est le POUVOIR !
	\end{enumerate}
	\item[V.] O{\`u} en sommes-nous ?
	\begin{enumerate}
		\setlength{\itemsep}{1pt}
		\setlength{\parskip}{0pt}
		\setlength{\parsep}{0pt}
		
		\item[1/] Nous vivons dans un monde malade, o{\`u} la haine est une arme et la libert{\'e}, un r{\^e}ve.
		\item[2/] Le monde grandit si lentement. Il est difficile pour un Cyberpunk de vivre dans un tel monde sous-d{\'e}velopp{\'e}, de regarder autour de lui, et de voir combien les gens progressent d'une mauvaise fa\c{c}on.
		\item[3/] Nous allons de l'avant, ils nous renvoient en arri{\`e}re une nouvelle fois. La soci{\'e}t{\'e} nous oppresse, elle {\'e}touffe la libert{\'e} de pens{\'e}e avec ces programmes scolaires et universitaires cruels. Elle inculque de force aux enfants sa vision des choses et toute tentative d'exprimer une opinion diff{\'e}rente est opprim{\'e}e et sanctionn{\'e}e.
		\item[4/] Nous enfants grandissent instruits par ce syst{\`e}me archa{\"i}que et immuable. Un syst{\`e}me qui ne tol{\`e}re aucune libert{\'e} de pens{\'e}e et exige une stricte ob{\'e}issance aux r{\`e}gles...
		\item[5/] Dans quel monde, si diff{\'e}rent de celui-ci, pourrions nous vivre si les gens sautaient dans l'inconnu et non y allaient {\`a} reculons ?
		\item[6/] C'est tellement dur de vivre dans le monde actuel, Cyberpunk.
		\item[7/] C'est comme si le temps s'{\'e}tait arr{\^e}t{\'e}.
		\item[8/] Nous vivons au bon endroit, mais pas {\`a} la bonne {\'e}poque.
		\item[9/] Tout est tellement ordinaire, les gens et leurs actes sont tous identiques. Comme si la soci{\'e}t{\'e} avait un besoin urgent de vivre dans le pass{\'e}.
		\item[10/] Certains en cherchant leur propre monde, le monde d'un Cyberpunk, et le trouvant, construisent leur propre monde. L'assemblant suivant leurs pens{\'e}es en modifiant la r{\'e}alit{\'e}, ils vivent dans un monde virtuel. Le fa\c{c}onnage de leurs pens{\'e}es bas{\'e}es sur la r{\'e}alit{\'e}.
		\item[11/] D'autres s'habituent simplement au monde tel qu'il est. Ils continuent d'y vivre m{\^e}me s'ils l'ont en aversion. Ils n'ont pas d'autre choix que de simplement esp{\'e}rer que le monde va sortir de son trou et aller de l'avant.
		\item[12/] Nous essayons de faire {\'e}voluer l'{\'e}tat des choses, d'ajuster le monde actuel {\`a} nos besoins et nos points de vue. D'utiliser au maximum ce qui s'accorde et d'ignorer le rebut. Quand nous ne le pouvons, nous nous contentons de vivre dans ce monde tel des Cyberpunks, quel qu'en soit le prix, et lorsque la soci{\'e}t{\'e} nous combat, nous contre-attaquons.
		\item[13/] Nous {\'e}difions nos mondes dans le Cyberespace.
		\item[14/] Parmi les z{\'e}ros et les uns, parmi les bits d'information.
		\item[15/] Nous fondons notre communaut{\'e}, la communaut{\'e} des Cyberpunks.
	\end{enumerate}
\end{enumerate} ~\\

Unissez-vous ! ~\\

Battez-vous pour vos droits ! ~\\

%% \setlength\parindent{0pt}

\clearpage

\textbf{\Large Un Manifeste Cyberpunk v2.0} --- Christian As. Kirtchev --- 28 janvier 2003

\emph{\footnotesize Nos {\^a}mes sont analogiques/num{\'e}ris{\'e}es, nous sommes des Cyberpunks. Ceci est suppos{\'e} constituer une seconde manifestation. } %% ~\\

\begin{enumerate}
		\setlength{\itemsep}{1pt}
		\setlength{\parskip}{0pt}
		\setlength{\parsep}{0pt}
		
	\item[I.] Cyberpunk ~\\
		Nous sommes les hommes nouveaux, une nouvelle esp{\`e}ce d'homo sapiens, qui devaient na{\^i}tre {\`a} cet {\^a}ge. La fa\c{c}on dont nous ressentons le monde, en incorporant de fa\c{c}on inn{\'e}e le Cyberespace. Le premier souffle de notre naissance se composait d'un dense flux d'{\'e}lectricit{\'e} parcourant les lignes, des bruits des machines nous enveloppant, de la vibration des donn{\'e}es d'informations circulant {\`a} haute vitesse {\`a} travers l'air et les c{\^a}bles. Notre mani{\`e}re d'absorber la technologie est {\'e}quivalente {\`a} la fa\c{c}on dont d'autres mangent, boivent et respirent. Le data-space lui-m{\^e}me est un {\'e}l{\'e}ment suppl{\'e}mentaire de notre environnement. Nous sommes cette mutation, qui ne se restreint pas uniquement {\`a} une meilleure compr{\'e}hension des outils technologiques. Tout le monde peut apprendre {\`a} utiliser et comprendre la technologie et les nouvelles technologies, mais nous les assimilons naturellement. Notre point de vue nous permet de voir au del{\`a} de ce que les gens ordinaires peuvent. L{\`a} o{\`u} ils n'entrevoient que l'aspect ext{\'e}rieur, nous d{\'e}celons ce qui est {\`a} l'int{\'e}rieur. C'est ce que nous sommes, des r{\'e}alistes avec des lunettes de r{\^e}veurs. Notre fa\c{c}on de penser et de percevoir notre environnement, le sang qui coule dans nos veines, l'air qui file dans nos cerveaux, c'est cette mutation qui nous distingue des autres. Etre un gourou du net, un geek technologique, un nerd informatique n'est pas suffisant, c'est juste un signe. Nous sommes des hommes neufs et chaque parcelle de nouveaut{\'e} est quelque chose que nous accueillons naturellement et famili{\`e}rement. Nous connaissons l'histoire et nous savons que c'est une morte rampant apr{\`e}s la vie. Le Cyberpunk est juste une {\'e}tiquette, son contenu est {\`a} l'int{\'e}rieur de nous, ces hommes et femmes diff{\'e}rents, dont la plupart sont au del{\`a} de toute compr{\'e}hension. Vous pouvez nous appeler fous, d{\'e}ments, cingl{\'e}s ou bizarro{\"i}des, ce sont juste les plus proches mots de votre dictionnaire qui effleurent ce que vous pouvez imaginer d'un ph{\'e}nom{\`e}ne qui ne s'est jamais manifest{\'e} auparavant. Le monde actuel est en plein changement. Certains coulent avec les ruines, d'autres vont de l'avant en laissant le pass{\'e} o{\`u} il est. La soci{\'e}t{\'e}, qui ne s'est toujours pas d{\'e}cid{\'e}e {\`a} se renouveler, {\`a} trouver la stabilit{\'e} de son existence dans l'ancienne acceptation de l'ordinaire et du connu. Mais nous n'appartenons {\`a} aucun d'entre eux, les Cyberpunks {\'e}voluent sans cesse. Et ceux qui pr{\'e}tendent que le Cyberpunk est mort ne peuvent voir sa renaissance dans la nouvelle vague de d{\'e}couvertes. Pouvez-vous r{\'e}ellement affirmer que l'{\'e}volution a {\'e}t{\'e} stopp{\'e}e ? Le "Cyberpunk" fait parti de cette {\'e}volution. Le rebelle, qui lutte pour sa propre survie. Et nous croyons en notre force, car notre compr{\'e}hension des ph{\'e}nom{\`e}nes nouveaux, qui restent obscurs aux autres, fait parti de nous.
	\item[II.] Soci{\'e}t{\'e} ~\\
		Par le pass{\'e} les gens avaient besoin d'un mod{\`e}le {\`a} suivre, et ce mod{\`e}le a facilement su tirer profit de cette forme de contr{\^o}le de la soci{\'e}t{\'e} et a commenc{\'e} {\`a} diriger {\`a} l'aide de coups tordus, allant toujours plus loin, vu qu'il restait la seule autorit{\'e} et le Syst{\`e}me ainsi contr{\^o}l{\'e} est devenu invincible. La soci{\'e}t{\'e} demeure maintenant sous contr{\^o}le et certains l'appr{\'e}cient ainsi.
		La soci{\'e}t{\'e} d{\'e}nie notre existence, car nous sommes bien plus dangereux pour son utopie, que ne le sont les gouvernements. Nous ne faisons pas partie de ces masses.
	\item[III.] Le Syst{\`e}me ~\\
		Le Syst{\`e}me. Vieux de plusieurs si{\`e}cles, existant sur des principes qui ne p{\`e}sent plus rien aujourd'hui. Un Syst{\`e}me qui n'a pas chang{\'e} depuis le jour de sa naissance. Le syst{\`e}me est ce qui vous contr{\^o}le. Ce sont les gouvernements, compos{\'e}s de personnes qui vivent {\`a} part de la masse. Ils n'ont pas chang{\'e} depuis la naissance de la vie sociale {\`a} sein de l'humanit{\'e}. D'autre part les corporations exercent une part toujours grandissante de ce contr{\^o}le et {\`a} l'heure actuelle on ne sait pas r{\'e}ellement qui en a les r{\^e}nes. Est-ce les corporations qui contr{\^o}lent les gouvernements ou font-ils partie du m{\^e}me bureau ? Toutefois, le syst{\`e}me n{\'e}cessite de la nourriture et des moyens pour exister, et ce support est assur{\'e} par les masses de la soci{\'e}t{\'e}, qui sont comme hypnotis{\'e}s pour en venir {\`a} faire confiance {\`a} quelqu'un qui exerce un tel contr{\^o}le sur leurs vies personnelles. Ce support subsiste gr{\^a}ce aux nombreux mensonges prof{\'e}r{\'e}s par le syst{\`e}me aux masses. Ces mensonges sont les v{\'e}rit{\'e}s qu'ils veulent nous faire avaler. Le Syst{\`e}me doit les imposer pour pouvoir r{\'e}gner. Les gouvernements ont besoin que nous les suivions aveuglement. Pas seulement eux mais {\'e}galement les corporations, qui dictent la mode, les choix alimentaires et le prix des m{\'e}dicaments. Et le tout forme le Syst{\`e}me, un ensemble de r{\`e}gles propag{\'e}es par les m{\'e}dias. Seul un aveugle et sourd accorderait le contr{\^o}le de sa vie {\`a} quelqu'un qui, sous couvert d'apporter une impression de Soins, de Soutien, de S{\'e}curit{\'e} et de Stabilit{\'e}, ne recherche que l'argent et le pouvoir. Le syst{\`e}me est effray{\'e} par le chaos, mais le chaos est uniquement un chemin vers le libre arbitre. Lorsqu'ils seront d{\'e}centralis{\'e}s, les gens pourront faire de meilleurs choix. 
		~\\
	\item[IV.] Les M{\'e}dias ~\\
		La t{\'e}l{\'e}vision, la radio et la presse ne sont plus la seule source d'information pour l'homme qui en recherche ou celui en sommeil. Internet est le nouvel espace {\`a} m{\'e}dia, un espace o{\`u} l'information peut {\^e}tre propag{\'e}e librement et par cons{\'e}quent personne ne peut plus l'ignorer. M{\^e}me lorsque les gouvernements et les entreprises tentent de restreindre et contr{\^o}ler le flot de donn{\'e}es, des voies d{\'e}tourn{\'e}es permettent de r{\'e}cup{\'e}rer cette information, laquelle peut "r{\'e}sonner". Et l'information demeure toujours le pouvoir. Nous assistons {\`a} la croissance actuelle de notre race. Les entraves {\`a} l'information ne barrent plus les perspectives et les gens peuvent maintenant exiger plus de droits. Les scientifiques font des d{\'e}couvertes qui, une fois rendues publiques ne peuvent plus {\^e}tre si facilement bloqu{\'e}es par l'utilisation commerciale ou gouvernementale. Le plus triste arrive quand les gens sont oblig{\'e}s de se rabaisser pour demander ce qui leur est d{\'e}j{\`a} accord{\'e}. Maintenant ce m{\'e}dia peut {\'e}veiller le peuple, transformer la soci{\'e}t{\'e}. Toutefois, il a {\'e}galement prouv{\'e} qu'il pouvait se r{\'e}v{\'e}ler faux ou mensonger, ce qui complique le filtre de la v{\'e}rit{\'e} et augmente le prix de l'Information.
	\item[V.] O{\`u} en sommes-nous ? ~\\
		Nous sommes ceux dont l'ADN a commenc{\'e} {\`a} muter pour accueillir de nouveaux sens qui permettront aux g{\'e}n{\'e}rations futures d'appr{\'e}hender le cyberespace et le data-space. Aucun mat{\'e}riel lourd ou dispositif implant{\'e} sera pleinement en mesure de remplacer ce que la nature nous offre. Les mutations se mettent en place. L'{\'e}volution nous dote de meilleurs outils pour interagir avec les changements de notre environnement. C'est ce qui fait de nous des Cyberpunks, des hommes nouveaux, des esprits {\'e}lectroniques. Nous savons que le Cyberespace est un miroir du monde, une augmentation, qui accueille toutes les cr{\'e}ations pass{\'e}s et pr{\'e}sentes de l'homme. Le cyberespace est un monde invisible o{\`u} l'esprit et la pens{\'e}e humaine fusionnent avec la mati{\`e}re et prennent une forme visible pour les sens, par le biais des machines. Il semble avoir toujours exist{\'e} ici, l{\`a}, partout, mais c'est seulement maintenant que nous sommes capables de nous y connecter, de l'explorer et nous commen\c{c}ons {\`a} changer. Nous les Cyberpunks sommes ceux qui vivons dans le cyberespace, et la technologie actuelle n'est que le moyen de nous emmener de l'autre c{\^o}t{\'e}.
\end{enumerate}

	Nous sommes la nouvelle race modifi{\'e}e. Les Cyberpunks. ~\\
	Ceci constitue une seconde manifestation. ~\\

\clearpage

\textbf{\Large Un Manifeste Cyberpunk v3.0} --- R{\'e}dig{\'e} par des Cyberpunks en 2007 %% ~\\

\emph{\footnotesize Ceci est le brouillon d'un nouveau manifeste Cyberpunk afin de concorder avec la r{\'e}alit{\'e} de 2007. } % ~\\

\begin{enumerate}
		\setlength{\itemsep}{1pt}
		\setlength{\parskip}{0pt}
		\setlength{\parsep}{0pt}
		
	\item[I.] Introduction ~\\
		"Si vous aviez un ordinateur contr{\^o}l{\'e} par la pens{\'e}e, enti{\`e}rement personnalisable, ferait-il partie int{\'e}grante de votre conscience ?" Gloire au nerd ordinaire, qui ne parcourt pas souvent les rues, vivant continuellement connect{\'e}, occup{\'e} {\`a} phreaker, hacker, tweaker, etc. {\^E}tre Cyberpunk, exister autrement qu'une personne lambda, percevoir ce que les gens normaux ne voient tout simplement pas. La vie num{\'e}rique, au sens biologique du mainframe, d{\'e}finie le monde dans lequel nous vivons. Etiquet{\'e}s comme nerds, cingl{\'e}s, punks, aberrations de la nature... Des ph{\'e}nom{\`e}nes technologiques, qui comprennent tous les syst{\`e}mes, catalogu{\'e}s comme tar{\'e}s par la soci{\'e}t{\'e}. Ayant plus de connaissances que la plupart des gens, hackant des syst{\`e}mes par plaisir, saturant num{\'e}riquement leur esprit avec de l'information de choix, etc. Les gens comme nous, {\'e}pris de technologie, ayant plus d'amis en ligne que dans leur propre ville, examinant et ma{\^i}trisant des technologies pour les d{\'e}tourner de leur fonctionnement initial, recherchant d'autres voies de lib{\'e}ration du syst{\`e}me, luttant contre l'oppression et la censure des corporations, se battant dans des guerres de l'information... ~\\

		Vive la r{\'e}volution cybern{\'e}tique, vive les techno-anarchistes, vive les hackers et les crackers.
	\item[II.] Vision ~\\
		Le monde qu'ils veulent nous c{\'e}der n'est qu'une boule toxique remplie de maladies, un cadavre vivant ravag{\'e} par la pollution et la futilit{\'e} malsaine. Le monde qu'ils nous allouent n'existe que pour les servir, pour les conserver au pouvoir eux et leurs plus proches amis, pour garder le b{\'e}tail humain confiant et ignorant des alternatives, pour exploiter et d{\'e}truire la plan{\`e}te et pour {\'e}touffer ceux qui voudraient am{\'e}liorer le monde. Ils veulent que les masses restent dociles tels des moutons, et {\`a} cette fin les abreuvent de "divertissements" afin de leur laver le cerveau et les abrutir. Ils tenteront n'importe quoi pour rester au pouvoir et r{\'e}primeront toute menace, m{\^e}me au point de condamner {\`a} la "damnation {\'e}ternelle" de la main de leur fantasme schizophr{\`e}ne appel{\'e} "Dieu" ceux qui oseraient {\^e}tre en d{\'e}saccord. ~\\

		Qui sont-ils ? Ils sont les entreprises, les pays et les individus qui cherchent la domination {\`a} travers la d{\'e}pendance du b{\'e}tail humain {\`a} certains concepts : le p{\'e}trole, la drogue, la religion, la "technologie" abrutissante et la "culture populaire". Ils poss{\'e}daient autrefois de puissants monopoles, bris{\'e}s pour avoir refus{\'e} de jouer juste, et tentent maintenant de ressusciter leur ancienne gloire corrompue en complotant avec Big Brother pour nous priver de libert{\'e} et de vie priv{\'e}e. Ils d{\'e}tiennent des compagnies tentaculaires bas{\'e}es sur Internet qui ach{\`e}tent chaque petite soci{\'e}t{\'e} sur le net afin de contr{\^o}ler son contenu. Ils sont les m{\'e}dias d{\'e}cervel{\'e}s vomissant sans cesse leurs cr{\'e}tineries appel{\'e}es "divertissements" pour garder le b{\'e}tail soumis, et utiliser ensuite les "droits d'auteur" pour enfoncer les entreprises Internet et {\'e}tendre leur contr{\^o}le sur elles. Ils sont des acteurs, des musiciens, des artistes qui conspirent avec les soci{\'e}t{\'e}s de m{\'e}dias pour diffuser leur vomi, et se prostituer ensuite pour des causes politiques d{\'e}gradantes. Ils sont des lunatiques et des pervers qui nous gavent de leur "th{\'e}ologie" et s'attendant {\`a} ce que nous ayons une fois aveugle en leur "Dieu" m{\^e}me si ce "Dieu" n'est qu'une illusion archa{\"i}que. Ils sont ces idiots incapables de marcher et de m{\^a}cher du chewing-gum en m{\^e}me temps, qui parviennent {\`a} devenir des dirigeants nationaux non par la volont{\'e} du peuple mais gr{\^a}ce {\`a} des circonstances et un climat politique favorables. Ils sont les causes de la d{\'e}cr{\'e}pitude du monde et la raison de la contre-attaque des Cyberpunks. ~\\

		Les Cyberpunks refusent d'{\^e}tre du b{\'e}tail, de prendre le train en marche, de faire partie des spectateurs, ils discernent les artifices du monde moderne et les combattent. Les Cyberpunks sont ceux assez courageux pour d{\'e}livrer leur message, debout devant la foule, le gouvernement, l'{\'e}glise et l'industrie et proclamer "Vous vous trompez !". Ils acceptent l'ostracisme, car ils ont d{\'e}j{\`a} rejet{\'e} la fa\c{c}on dont la "soci{\'e}t{\'e}" esp{\`e}re une am{\'e}lioration. C'est parfois une lutte solitaire, mais les Cyberpunks font pratiquement partie de toutes les soci{\'e}t{\'e}s, tous les pays, et quand ils se rassemblent, ils sont une force {\`a} laquelle ils doivent faire face.
	\item[III.] Technologie ~\\
		Les progr{\`e}s technologiques nous gardent actifs et les technologies les plus r{\'e}centes se rapprochent de plus en plus de celles d{\'e}crites dans le vieux Cyberpunk (pr{\'e}-2000) comme les implants biologiques, les guerres virtuelles et de l'information. Les batailles futures ne seront plus men{\'e}es par des arm{\'e}es, mais avec des syst{\`e}mes informatiques. Nous nous adoptons {\`a} ces avanc{\'e}es technologiques et nous pensons m{\^e}me (comme toujours) un peu au-del{\`a}. Se projeter de vingt ans dans le futur, cr{\'e}er de nouvelles choses avec la technologie actuelle, inventer d'autres moyens d'utiliser un appareil en d{\'e}tournant son fonctionnement, engendrer un nouveau syst{\`e}me {\'e}lectronique {\`a} partir d'un autre, comme hacker un t{\'e}l{\'e}phone pour servir de t{\'e}l{\'e}commande infrarouge.
	\item[IV.] Politique ~\\
		La guerre contre les grosse multinationales comme Unilever, Shell, Microsoft, Google, Gillette et autres, fait toujours rage. L'oppression des corporations utilisant les gouvernements fantoches comme de simples outils, afin de garder le peuple stupide, n'est pas envisageable pour les Cyberpunks. Nous sommes des anarchistes/r{\'e}volutionnaires anti-syst{\`e}me. Nous pensons par nous m{\^e}me et n'avons besoin de quiconque pour nous dire quoi penser. Combattre les corporations en hackant leurs syst{\`e}mes, r{\'e}pandre des virus, des mind-probes et organiser des attaques DDOS sont des moyens d{\'e}sormais courants en 2007. Les faits d{\'e}crits dans des livres comme Neuromancer et autres romans Cyberpunk {\'e}crits dans les ann{\'e}es 80/90, sont devenues r{\'e}alit{\'e}. Le 21e si{\`e}cle se num{\'e}rise rapidement et les gouvernements suivent le mouvement en cr{\'e}ant des protocoles d'identifications afin de contr{\^o}ler leurs peuples. Les Cyberpunks continuent de lutter contre ces protocoles en devenant anonymes, nerds aux cerveaux bien remplis, en hacker des trucs, etc.
	\item[V.] Sociologie ~\\
		Les Cyberpunks ont tendance {\`a} {\^e}tre des individus qui ne correspondent {\`a} aucun syst{\`e}me ou groupe. Ils sont pour la plupart des gens tr{\`e}s intelligents, qui voient au-del{\`a} de l'explicite. Ils rendent compte de l'absurdit{\'e} de la culture, la parodie, l'{\'e}crase et la transforme. Ils utilisent les outils de la soci{\'e}t{\'e} pour t{\'e}moigner contre elle, pour documenter et rapporter ses tendances auto-destructives ; en partageant leurs connaissances avec un groupe choisi bien souvent maudit et d{\'e}test{\'e} par les gens du syst{\`e}me. Ils ont plus d'amis sur les r{\'e}seaux num{\'e}riques que dans leur propre ville et {\'e}changent plus de savoirs {\`a} travers les autoroutes du net que par le biais d'interactions physiques. D{\'e}penser plus de temps {\`a} obtenir davantage de connaissances et d'informations que de socialiser leur est familier.
	\item[VI.] Histoire ~\\
		Les Cyberpunks renouvellent leurs go{\^u}ts, musiques, aspect avec le temps. Dans les ann{\'e}es 80, ils {\'e}taient plus orient{\'e}s vers la musique synth{\'e}tique et les technologies d{\'e}mod{\'e}es alors que maintenant nous avons Internet {\`a} haut d{\'e}bit, des t{\'e}l{\'e}phones portables/satellites, la r{\'e}alit{\'e} virtuelle, etc. Nous nous adaptons {\`a} la technologie, et elle s'adapte {\`a} nous. Nous sommes symbiotiques, {\`a} la fois d'un point de vue biologique et technologique. Notre mouvement Cyberpunk progresse au fil du temps. Nous apprenons et nous nous d{\'e}veloppons toujours plus loin. Nous sommes les soldats de la fronti{\`e}re technologique, non connu du syst{\`e}me, mais toujours bien vivant. Le Cyberpunk, invent{\'e} par Bruce Bethke et r{\'e}alis{\'e} par William Gibson, {\'e}tait autrefois de la Science-Fiction et devient maintenant r{\'e}alit{\'e}. Lorsque les fans du Cyberpunk mettront la main sur la technologie actuelle, leur nombre augmentera fortement et ils r{\'e}aliseront que le Cyberpunk n'est pas seulement un conte de f{\'e}es mais {\'e}galement un mouvement hi-tech qui est all{\'e} bien au del{\`a} de l'imagination des auteurs originels de romans Cyberpunk.
	\item[VII.] Libert{\'e} ~\\
		Les Cyberpunks encouragent la libert{\'e} absolue de pens{\'e}e. Notre qu{\^e}te pour le libre acc{\`e}s {\`a} toute information ne conna{\^i}t pas de fronti{\`e}res. Nous sommes ceux qui ont besoin de comprendre avant d'accepter quoi que ce soit du flux des m{\'e}dias. Nous sommes le bug dans le code source, la probabilit{\'e} de (r){\'e}volution qui menace les syst{\`e}mes rigides. Personne ne peut nous contr{\^o}ler, et c'est pour cela que nous sommes pourchass{\'e}s. Nous n'appartenons pas {\`a} la Soci{\'e}t{\'e} qu'ils veulent cr{\'e}er. Programmeurs talentueux, r{\^e}veurs utopiques, artistes ou employ{\'e}s de bureau, nous sommes ceux qui r{\'e}sistent, qui vivent dans le datanet o{\`u} il n'y a pas de loi applicable, en contournant toutes les fronti{\`e}res ; le cybermonde est l'endroit auquel nous appartenons. Nous sommes les enfants du cyberespace, nous pouvons faire tout ce que nous voulons, nous nous tournons vers l'avenir, en essayant de r{\'e}fl{\'e}chir aux nouvelles technologies, propageant nos id{\'e}es {\`a} travers le vaste oc{\'e}an de l'information. Nous ne sommes pas de simples agr{\'e}gateurs d'information, pour nous l'information n'est pas un flux immat{\'e}riel de donn{\'e}es, mais une partie de nous comme nous le sommes d'elle. Elle chemine {\`a} travers nos esprits, {\`a} l'instar du sang et de l'oxyg{\`e}ne. La retenir ou l'utiliser pour nous opprimer est comme nous priver d'air pour respirer. Finalement, nous sommes d'ultimes chirurgiens du cerveau, capables d'{\'e}liminer toute les immondices qu'ils voudraient nous implanter.
\end{enumerate}

\clearpage

\textbf{\large Un Manifeste Cyberpunk} --- Transceiverfreq --- 20 ao{\^u}t 2011 %% ~\\

	\textbf{\footnotesize "L'information n'est que de l'information, pas de la mati{\`e}re ou de l'{\'e}nergie." --- Norbert Wiener }~\\

\begin{minipage}[ht]{0.20\textwidth}
	\footnotesize \centering
	\includegraphics[width=55pt]{img/Kanizsa_triangle.png} ~\\
	\emph{Le motif de Kanizsa}
\end{minipage} \hfill \begin{minipage}[ht]{0.75\textwidth}
	\footnotesize
	Cr{\'e}{\'e} en 1976-79 par Gaetano Kaniza, psychologue et artiste italien, fondateur de l'Institut de Psychologie de Trieste. Une illusion d'optique comportant trois cercles noirs dont la pointe faisant face {\`a} un point central a {\'e}t{\'e} retir{\'e}e et trois angles noirs sur un fond blanc. Les contours illusoires d{\'e}finis par ces formes cr{\'e}ent un triangle blanc dans "l'espace n{\'e}gatif" liminal. Ceci est une m{\'e}taphore pour le "cyberespace", d{\'e}fini par des fronti{\`e}res illusoires et intangible. L'id{\'e}e du triangle existe, mais en fait, elle est d{\'e}finie par l'interaction des formes dans notre esprit et notre perception. C'est le symbole le plus simple que j'ai trouv{\'e} jusqu'{\`a} maintenant pour d{\'e}crire le "cyberespace".
\end{minipage} ~\\

\begin{multicols*}{2}
	\small
	\textbf{{\`A} lire}
	\begin{itemize}
		\setlength{\itemsep}{1pt}
		\setlength{\parskip}{0pt}
		\setlength{\parsep}{0pt}
		
		\item Wikipedia -- Cybern{\'e}tique.
		\item \emph{Cybern{\'e}tique et soci{\'e}t{\'e}} -- Norbert Wiener, 1948.
		\item \emph{Cybernetics: Or the Control and Communication in the Animal and the Machine} -- Norbert Wiener, 1950.
		\item \emph{Does Technology Drive History? The Dilemma of Technological Determinism} -- Merritt Roe Smith, Leo Marx, 1994.
		\item \emph{Human-Built World: How to Think about Technology and Culture} -- Thomas P. Hughes, 2004.
		\item \emph{Cyberspace: First Steps} -- Michael Benedikt, 1991.
		\item \emph{Neuromancien} -- William Gibson.
		\item \emph{Fear of Knowledge: Against Relativism and Constructivism} -- Paul Boghossian, 2006.
		\item \emph{Cyborg: Digital Destiny and Human Possibility in the Age of the Wearable Computer} -- Steve Mann, 2001.
		\item \emph{Simulacres et simulation} -- Jean Baudrillard, 1981.
	\end{itemize} ~\\
	
	\textbf{Nous affirmons :}
	\begin{itemize}
		\setlength{\itemsep}{1pt}
		\setlength{\parskip}{0pt}
		\setlength{\parsep}{0pt}
		
		\item L'information "veut {\^e}tre libre" et devrait l'{\^e}tre.
		\item L'information am{\'e}liore la qualit{\'e} de vie.
		\item L'information permet une meilleure autonomie.
		\item Non seulement l'information "veut {\^e}tre libre", mais elle a une tendance naturelle et in{\'e}vitable {\`a} devenir libre.
		\item L'information maintient sa libert{\'e} propre par l'inaction ou l'action de lib{\'e}ration par des forces externes.
		\item Toutes les guerres du pass{\'e} sont bas{\'e}es partiellement, voire enti{\`e}rement, sur l'{\'e}change, la suppression ou la d{\'e}couverte de multiples formes d'information.
		\item Les inventions d{\'e}coulent d'un besoin et le monde a un terrible besoin de changement.
		\item Le r{\'e}seau homme/machine est la victoire de l'information.
		\item Internet n'appartient {\`a} aucune nation, il est sa propre nation.
		\item Le cyberespace existe entre les noeuds du r{\'e}seau. C'est le m{\'e}ta-espace d{\'e}fini par l'{\'e}change d'information entre ses utilisateurs.
		\item Internet d{\'e}finit le cyberespace par ses propres fronti{\`e}res en expansion et mutation permanentes.
		\item Les mouvements du cyberespace empi{\`e}tent sur l'espace r{\'e}el (meatspace). De cette fa\c{c}on, le net se d{\'e}place constamment dans le monde r{\'e}el.
		\item Le monde est dans un {\'e}tat de r{\'e}volution constante.
		\item Le web est le monde.
	\end{itemize} %% ~\\
	
\vfill
\columnbreak
	
	\textbf{Nous :}
	\begin{itemize}
		\setlength{\itemsep}{1pt}
		\setlength{\parskip}{0pt}
		\setlength{\parsep}{0pt}
	
		\item Sommes natifs du net.
		\item Sommes li{\'e}s dans l'anonymat et ce choix est justifi{\'e} quotidiennement.
		\item Sommes le r{\'e}seau lui-m{\^e}me. Il est d{\'e}fini par notre action et notre inaction.
		\item Sommes n{\'e}s sans all{\'e}geance nationale et libres en tant que tels.
		\item Attendons mieux de notre g{\'e}n{\'e}ration que de celles qui nous pr{\'e}c{\`e}dent.
		\item Choisissons la libert{\'e} pour nationalit{\'e} et la nodalit{\'e} autonome comme langue maternelle.
		\item Choisissons la vitesse pour m{\'e}thode et le flux pour force motrice.
		\item Rejetons les entit{\'e}s gouvernantes, nationales, religieuses et corporatives. Nous rejetons leur emprise sur l'esprit et la condition humaine.
		\item Croyons que "la rue" d{\'e}finit ses propres usages pour la technologie.
		\item Croyons qu'une surcharge d'information est un concept impossible.
		\item Croyons que chacun devrait avoir acc{\`e}s au r{\'e}seau et aux donn{\'e}es.
		\item Croyons que l'acc{\`e}s aux ordinateurs, et {\`a} tout ce qui peut nous apprendre quelque chose {\`a} propos de la fa\c{c}on dont le monde fonctionne, devrait {\^e}tre illimit{\'e}.
		\item Nous {\'e}levons contre les formes {\'e}tablies de hi{\'e}rarchie et soutenons la d{\'e}centralisation.
		\item Promouvons le d{\'e}passement de nos limites, tant personnelles que partag{\'e}es.
		\item Sommes la technologie. Internet est fait de viande.
		\item Ne sommes pas les adolescents rebelles et d{\'e}prim{\'e}s tels que d{\'e}crits par nos anc{\^e}tres.
		\item Ne nous d{\'e}finissons pas par la technologie que nous poss{\'e}dons, mais bien par l'utilisation que nous en faisons.
		\item Croyons en la pr{\'e}sence irr{\'e}sistible du futur.
	\end{itemize} %% ~\\
	
	\textbf{Conseils ($\emptyset$)}
	\begin{itemize}
		\setlength{\itemsep}{1pt}
		\setlength{\parskip}{0pt}
		\setlength{\parsep}{0pt}
		
		\item Quand ils l'appellent "paradoxe", ils cachent l'oxyg{\`e}ne.
		\item Une bonne traduction a des qualit{\'e}s que l'original ne saurait saisir.
		\item Voyagez l{\'e}ger, restez sur vos gardes et mangez ce que vous tuez.
		\item Regardez toujours en premier la face cach{\'e}e des choses.
		\item Quand vous trouvez la v{\'e}rit{\'e}, partagez-la avant qu'elle ne finisse enterr{\'e}e sous des tonnes d'argent.
		\item "La r{\'e}volution" d{\'e}vore ses petits.
		\item Pour contacter un op{\'e}rateur en chair et en os, faites le "0".
	\end{itemize} %% ~\\
\end{multicols*}

\clearpage


\begin{center} \begin{minipage}[ht]{0.95\textwidth}
	\small
	\textbf{\Large A Biopunk Manifesto} --- Meredith L. Patterson --- 30 janvier 2010 ~\\
	
	Scientific literacy is necessary for a functioning society in the modern age. Scientific literacy is not science education. A person educated in science can understand science; a scientifically literate person can *do* science. Scientific literacy empowers everyone who possesses it to be active contributors to their own health care, the quality of their food, water, and air, their very interactions with their own bodies and the complex world around them. ~\\
	
	Society has made dramatic progress in the last hundred years toward the promotion of education, but at the same time, the prevalence of citizen science has fallen. Who are the twentieth-century equivalents of Benjamin Franklin, Edward Jenner, Marie Curie or Thomas Edison ? Perhaps Steve Wozniak, Bill Hewlett, Dave Packard or Linus Torvalds - but the scope of their work is far narrower than that of the natural philosophers who preceded them. Citizen science has suffered from a troubling decline in diversity, and it is this diversity that biohackers seek to reclaim. We reject the popular perception that science is only done in million-dollar university, government, or corporate labs; we assert that the right of freedom of inquiry, to do research and pursue understanding under one's own direction, is as fundamental a right as that of free speech or freedom of religion. We have no quarrel with Big Science; we merely recall that Small Science has always been just as critical to the development of the body of human knowledge, and we refuse to see it extinguished. ~\\
	
	Research requires tools, and free inquiry requires that access to tools be unfettered. As engineers, we are developing low-cost laboratory equipment and off-the-shelf protocols that are accessible to the average citizen. As political actors, we support open journals, open collaboration, and free access to publicly-funded research, and we oppose laws that would criminalize the possession of research equipment or the private pursuit of inquiry. ~\\
	
	Perhaps it seems strange that scientists and engineers would seek to involve themselves in the political world - but biohackers have, by necessity, committed themselves to doing so. The lawmakers who wish to curtail individual freedom of inquiry do so out of ignorance and its evil twin, fear - the natural prey and the natural predator of scientific investigation, respectively. If we can prevail against the former, we will dispel the latter. As biohackers it is our responsibility to act as emissaries of science, creating new scientists out of everyone we meet. We must communicate not only the value of our research, but the value of our methodology and motivation, if we are to drive ignorance and fear back into the darkness once and for all. ~\\
	
	We the biopunks are dedicated to putting the tools of scientific investigation into the hands of anyone who wants them. We are building an infrastructure of methodology, of communication, of automation, and of publicly available knowledge. ~\\
	
	Biopunks experiment. We have questions, and we don't see the point in waiting around for someone else to answer them. Armed with curiosity and the scientific method, we formulate and test hypotheses in order to find answers to the questions that keep us awake at night. We publish our protocols and equipment designs, and share our bench experience, so that our fellow biopunks may learn from and expand on our methods, as well as reproducing one another's experiments to confirm validity. To paraphrase Eric Hughes, "our work is free for all to use, worldwide. We don't much care if you don't approve of our research topics". We are building on the work of the Cypherpunks who came before us to ensure that a widely dispersed research community cannot be shut down. ~\\
	
	Biopunks deplore restrictions on independent research, for the right to arrive independently at an understanding of the world around oneself is a fundamental human right. Curiosity knows no ethnic, gender, age, or socioeconomic boundaries, but the opportunity to satisfy that curiosity all too often turns on economic opportunity, and we aim to break down that barrier. A thirteen-year-old kid in South Central Los Angeles has just as much of a right to investigate the world as does a university professor. If thermocyclers are too expensive to give one to every interested person, then we'll design cheaper ones and teach people how to build them. ~\\
	
	Biopunks take responsibility for their research. We keep in mind that our subjects of interest are living organisms worthy of respect and good treatment, and we are acutely aware that our research has the potential to affect those around us. But we reject outright the admonishments of the precautionary principle, which is nothing more than a paternalistic attempt to silence researchers by inspiring fear of the unknown. When we work, it is with the betterment of the community in mind - and that includes our community, your community, and the communities of people that we may never meet. We welcome your questions, and we desire nothing more than to empower you to discover the answers to them yourselves. ~\\
	
	The biopunks are actively engaged in making the world a place that everyone can understand. Come, let us research together. ~\\
\end{minipage} \end{center}

%% \clearpage

\chapter{Report from the desert}

\begin{multicols*}{2}

[ \texttt{http://reportdesert.blogspot.fr/} ] ~\\

Thursday, December 02, 2004 ~\\

The first rule of the desert is: you don't know where the desert starts or finishes. ~\\

the second rule is: you don't really know where you are when you are in the desert. ~\\

posted by rd @ 5:56 AM %% ~\\

\begin{center} \rule{0.45\textwidth}{0.001cm} \end{center}

Standed in this damp city makes my bones freeze beyond recognition. It's also hard to brathe normally, very much like the very hot weather of the desert. ~\\

Take out the zire from the coat pocket and read a few pages of the book of the month just to kill a little time. I don't concentrate on what I'm reading, instead I think about the best way to get these local people to cooperate in my mission. It's not always like this, but today, this is how I feel; kind of lost in too many doubts and too much humidity. ~\\

I wonder how long does the winter lasts around here... I Hope not too much. But anyway, I'm only here for a week or two... ~\\

posted by rd @ 5:36 AM %% ~\\

\begin{center} \rule{0.45\textwidth}{0.001cm} \end{center}

Sunday, October 24, 2004 ~\\
{\large Possibly to discard...} ~\\

Algiers is quiet this time of the year. Meaning; not too many tourists. ~\\

posted by rd @ 9:46 AM   0 comments %% ~\\

\begin{center} \rule{0.45\textwidth}{0.001cm} \end{center}

Thursday, October 21, 2004 ~\\

{\large the story so far...} ~\\ %% ~\\ %% ~\\

{\Large report from the desert} ~\\

{\large clark-nova or martinelli?} ~\\

The first day began with a cloudless sky above my head. No wonder people can't fly planes on this day. ~\\
The sun is just too bright and hot. The typewriter has the shield cover with an abnormal temperature. ~\\
And there's nothing better on a hot day like today than hot typewriter's covers. White in color. ~\\

But the main aspect about today is the arrival of a movie celebrity to this old forgotten city surrounded by the desert. And it should come as no surprise that the person in question tried to contact me early this morning. Or so the owner of the dirty motel I live in told me when I came down to have some food and a cup of coffee before going out. ~\\
There's a newspaper in front of me but I'm not in the mood for bad or good news. And this heat is only making it worst for my brain. ~\\

This KDE thing is a bit sluggish on this old typewriter. Slower than MacOSX... Actually, much, much slower... ~\\

And the story goes on and on until the sun sets down on the horizon. One more thing: the trees that sometimes circle an oasis are not the tree from heaven as someone in last week's New York Times said... ~\\

I wish I had some other appearance. I wish everything would be differrent. --- I wish I could be dead. I wish I could be someone different. ~\\

And altough this KDE window manager is very slow, I'll still keep using it for as long as it takes. I do not know how long but I guess it will be one week or so. ~\\

There's nothing more real than a Coca-Cola can or bottle in the hot summer time in Europe. It's a bit different here in the desert. ~\\

I wonder if \textbf{GhostInTheShell} is known around these territories... I guess not. Maybe one or two persons know about it and Project 2501. I'll have to check it in the coming days and weeks. I was looking the other day, on the MatriX, for a wallpaper with a picture from GITS... No luck, no good pictures... ~\\

Just inserted a Sepia version of "waiting for the CargoBoat" picture. It looks good here in this report. While listening to PlastikMan latest from 2001. The New York year. ~\\

I'm watching The MatriX for the third [???] time. I wonder if I'll be able to watch MatriX Reloaded... I guess it will come out next year. I guess it won't be as good as the first one, just like almost any sequel done in Hollywood... ~\\

Testing the GIMP here on KDE. It works quite well, as I expected, gotta make a shortcut... ~\\

Have the XLoad app on the desktop right next to the XClock app. True UniX appearance... But it seems to crash KDE when I tried to launch with these two apps to start automaticly on startup. ~\\

Just changed the colors of the appearance to BeOS yellow. It's brighter and happier. ~\\

This is my original \textbf{typewriter}. I have brought it to this desert along with the iBook. Just in case this shithole city runs out of electricity, this way I have an option to get the reports out. ~\\

Well this is another day. And the sun is still shining hard in the sky. So the rain is still very far away from these lands... ~\\

\begin{itemize}
	\setlength{\itemsep}{1pt}
	\setlength{\parskip}{0pt}
	\setlength{\parsep}{0pt}
	
	\item[$\bullet$] Everybody's talking about weblogs these days. Google just signed some contract with some company that publishes weblogs on the MatriX...
	\item[$\bullet$] I'm on 85\% zoom. The MatriX wallpaper on the desktop. The fact is that I'm only using this KWord processor because of the LucidaTypewriter font which is beautifull. Good for writing professional reports. And "Professional" even with all this heat. The desert has it's own weapons. Intense heat during the day and extreme cold at night.
	\item[$\bullet$] Listening to The Sisters of Mercy live somewhere in Europe on my rotten stereo. It helps to pass the time until it's cool enough to go out and have a cup of tea or a coffee or whatever...
	\item[$\bullet$] The night is setting in. The cars are growing in number, and so is their noise. By the time we reach midnight there will be less car noise. Still listening to SOM... Temple of Love with the late Ofra Haza. On the street in front of me people are walking like it was the end of the world. It seems that people around here are very stressed.
	\item[$\bullet$] And here I am again with the afterstep window manager which is a lot lighter than KDE... And thank god for that. And it's based on modules... How original!!!
\end{itemize} %% ~\\

Ah... This is much faster than KDE. And now I know how to configure things here. At least some things... ~\\

I loged on to the MatriX to read some stuff at firstmondays.org. I'm reading an article on hyperlinks. Good stuff. And there's another one on CyberTerrorism... The terrorist from september11 didn't used encryption on their emails... Ping and Spray, two funny commands... I'm reading an article on the command Ping. How to smurf with Ping... They have no info about Windows2000... It's a shame... ~\\

Back in NeXtLand. Black and white with a few spots of color. That's what I call this AfterStep interface. It's not pretty, but it's elegant and stable... And altough, this KWord isn't fast, the rest of the interface is quite responsive. Not as responsive as Aqua, of course... ~\\

Some guys in USA took a picture of the space shuttle Columbia minutes before it broke apart with a cheap telescope and an 11 year old Macintosh... Old technology still has a lot to do. This AfterStep environment feels like 1991 or whereabouts. It's a relic from old times. ~\\

And on this desert I now own a \textbf{SGI portable} computer. I traded my old \textbf{iBook} for this almost new SGI book. It's not faster than the iBook, it's even a little slower, but it has a larger hard disk. And that's what I want right now. I traded the iBook for this SGI in a downtown caf{\'e} near the main avenue. A man was with this portable SGI and he was talking to the waiter telling him that he wanted to sell this \textbf{orange} SGI \textbf{laptop}. I overheared the conversation and asked him if he wanted to swap computers with me. He then looked at the iBook and got very interested on the operating system which is simply a mixture between elegance and sheer beauty. The OS of this SGI is IRIX 6.5 which is also a Unix based OS. He got interested and traded the SGI without a hint of remorse. An hour or so later, after a small trip to my room to connect to the MatriX to upload all the documents in the iBook, we closed the deal. And I must say that I like this SGI better than the iBook. I don't really miss the Aqua interface of OSX, better looking than this SGI one. But this portable feels more robust than the iBook. A feature that everybody needs badly when living in the middle of the desert. One has to be very carefull about the sand. It's thin and it gets to places one would not imagine in a normal world... Just remembered one strange fact about the deal with the portable computers: the guy that took my iBook gave me a french magazine titled SVM that covers such subjects as photography, computers, video, film and literature, among other various subjects... I didn't understand why, but the magazine is siting on top of the only table I have in this cheap room. I also own 4 new CD's with the SGI. It's the system software and applications. Checked the MatriX for instructions on how to install the IRIX OS from scratch. It's hard, confusing and an utter mess. But I guess I'll never need to install it from scratch, let's hope not. Unless I'm in the mood for that. ~\\

%% \vfill
%% \columnbreak

One of the records I brought along with me to this desert is "Rio" by duranduran. The windows decoration on this SGI looks a lot like the album design colors. Weird... And it's one of my alltime favorite albums. But that's not what's playing right now, I'm listening to the Sisters Of Mercy "First and Last and Always". This color scheme reminds me also of the Solaris OS from SUN computers... Memories of Panasonic machines being controlled by american SUN/Solaris machines. Well those times are gone behind me now, but nevertheless, they were good times. Well, sometimes... Black Planet, the first song from the Sisters album reminds me of that Solaris OS, don't know quite why, though... But someday I'll find out, I just know it. And I guess I'll find the answer to that in New York, maybe in Manhattan... This SGI is a portable suited for terrorists. That's just my opinion or impression... Or whatever you want to believe in... ~\\

Just went downstairs and there's a russian film on television. Andrei Tarkowsky, I think... The film has some Civil war in Spain footage from the 30's. All along with a gipsy song soundtrack. I wonder how old is this film... A woman turning pages and pages of an old photograph book... Classical music as the soundtrack... All in wonderfull black and white. But the most part of the film is in color. I can see that now. I've been staring at this film for more than half an hour now. The film is somehow inspiring me... Isn't that what films are made for? I wonder what the hell the director thought when he was making this film. ~\\

I wonder how old this SGI is. At least three years... I forgot to ask that to the old owner. But this model has been around since 1999 at the very least, maybe sooner, I don't really know. I guess I could google for it, but I'm not in the mood right now. I'll just watch a little bit more of the film. And then, maybe I'll go to bed and read a few articles of the french magazine. And look at the photos... There's a very interesting photo of a woman in a french caf{\'e}, maybe in Paris, with a laptop on a table and people siting in another table talking or whatever they are doing. Interesting and inspiring. I'll have a look at that article later on. ~\\

A big explosion in the film. A loud noise. Bright colors. Nobody in this room is watching the film now. I'm the only one, I guess. But they're all listening to the soundtrack and the voices of the actors. It's not that strange as it seems. It almost seems that everybody is dead... At least, not living as they should... A voice inside my head whispers: welcome to the desert, these are it's people. Get used to them... ~\\

%% \vfill
%% \columnbreak

Sometimes I wonder if I'm writing good reports. I never get any feedback and that doesn't help my writing. I wonder what really does help my writing... My inspiration? My awareness? ~\\

{\large sleeping cells} ~\\

The SGI is performing rather well. It is surprising me in a good way. And it doesn't get too hot like the iBook did. It's a little bit slower than the iBook, I believe and, somehow, feel. ~\\

I'm going away for 8 months on a cargo boat. Don't know yet the destination but it should be the Artic somewhere near Russia or Alaska. I'll find out when the trip ends... The ship leaves in two weeks and I'm not yet ready. The main object I'll take on the trip will be this laptop. %% ~\\

Maybe then I'll find the time to finish SFinally... And that's if I don't get sea sick. ~\\

Somewhere in \textbf{Paris} there's a woman waiting endlessly for me. That's what the future whispers in my ear. I believe that. I have to. I f I don't, I guess I won't be able to live anymore. ~\\

That was done in MacOSX in the TextEdit app. Yes, this IriX OS has a MacOSX emulator. Not as fast as the real thing but acceptable. ~\\

So, let's try out Netscape 4.9 with an IriX twist just to see what can I do with this... The report will go on in a few lines... ~\\

So this Kaleidoscope theme is very desert-like... It has the feel of desert sand, or of beach sand, who really knows? Anyway, the Xdarwin version of the IriX/SGI look is somehow different, with the Rio theme. This OS9/Kaleidoscope theme is called True IriX and... Well, that's it. A laptop with OS X, OS9 and XDarwin with SGI theme... Well... Let's go on. ~\\

I'm in the mood for OS9... Maybe in a few minutes... Just been there but returned after two minutes to OSX and to this \textbf{Composer4.9}. ~\\

Well, there's always this problem with intense heat here in the city. People, locals, told me that the temperatures only lower a bit, just a few degrees, not much, in the months of December and January. And since this is April I still have a lot to wait for. Unless I make the boat trip to the artic. And it seems like that's gonna happen. I'm getting ready for that. Not that I will take a lot of bags, no I don't have them and I've always liked to travel light... %% ~\\

Tonight I'll go out into the city to discover a nice pub/bar that I'll feel confortable in. That's not too much to ask. I'm hopefull on finding a good bar. It must have arabic music as the ambient music, that's one of the reasons I came here for in the first place. The days of VonMagnet and the PeterMurphy Dust. I heared some nice ambient music on the local cybercaf{\'e}, the "\textbf{L@Red}". The owners are spanish and speak fluent, for a spanish, english. They also have one SGI computer there, an Indigo with IriX 5.something... It's a bit slow but it always gets there on the MatriX, it's been my favorite computer there along with the \textbf{G4Cube} from Apple. The other computers there are just trivial... Sometimes I just go there for a cup of coffee and no MatriX time. Just to pass the time and look for other agents, I heared that L@Red was used as a meeting point for some european agents. I guess I alreday spotted two of them, but I'm not too sure. I'll have to be more aware about those kind of things. Maybe one day I'll find WilliamBurroughs there as well with his \textbf{Clark Nova} or \textbf{Martinelli}... ~\\

The streets are empty. Only a few cars pass by once in a while. It's the desert, I guess... ~\\

On the desk of the hotel lobby I found a french magazine with an interview with \textbf{Jean-Louis Gass{\'e}} from Be Inc. The magazine dated from June 2000. Nothing really interesting, it was made during the hallucination times of BeInternetAppliances. How wrong was Be? A lot. They declared the extintion of Be Inc. in late 2001, just after the World Trade Center collapsed. Well, even if the terrorist attacks didn't occur, Be would have died anyway. I still have the last version of the OS installed back home in my AMD Duron 800Mhz desktop computer, BeOS 5.0.1. It is a good OS, very fast, very responsive. And there's a XML editor from out of this world called EnglishEditor2 that runs on BeOS5. It was with EE2 that the Koncorrrde files began being written. I still have those XML files archived here on the SGI laptop. Netscape is the only app that opens them well, with the correct character encoding [ UTF-8 ]. BeOS5 had something magical about it just from the simple fact that the boss of Be Inc. was from France. All the europeans felt a good vibe towards that operating system. It mades us [europeans] feel like the americans... A bit, yes. Even though, that's definetly not a good thing. ~\\

Listening to Black Planet from the Sisters 1st. So Dark all over Europe... Highway 101... Homesick? Well, no. ~\\

If I'm to go on that boat trip I have to take a bus or a plane to Tangier in Morroco. Not a critical thing, in my opinion. If I go by bus it will take about two days and 3 hours by airplane. ~\\

There's a little caf{\'e} just down the street that has the New York Times everyday for the public to read. It's always yesterday's edition, but it couldn't be better in this part of the world. ~\\

There's a poster of the slogan "\textbf{Everything you know is Wrong}" in my desk. I like it, it's been a guiding line in my life. That's another reason for me being here in the desert. I think about that slogan countless times. It's enlightening for me. And sometimes it's disturbing and distressing. ~\\

Someone, an english lady of about 50 years old staying here in the hotel in holidays, asked me if I was interested in buying a NeXT Cube from 1993. She noticed this SGI laptop and began talking to me imediatly. The price she asked for it, postage included, was 1000 Euros. I guess I'm not interested. What I need is a laptop not a desktop. ~\\

Remembering an old PeterMurphy album from 1992 or 1993 called Holy Smoke. Track5 is particularly good, it reminds me of Bauhaus atmospheres. Dark and mysterious... The Dust album also came to my mind. ~\\

There's always something around that reminds me of the desert. I know it's all around this city, it's almost clautrophobic. And strange. Like a room full of typewriters... Like a day with too many clouds and no rain. Like a lost memory that comes back once after too many years hidden. A stranded ship on an empty ocean. ~\\

\vfill
\columnbreak

The owner of the hotel just showed me his VHS film collection. He has a copy of \textbf{Naked Lunch}!!! I almost fainted when I saw that! I'm watching the movie now in my room. It's been more than 5 years since I last saw this film. And I saw it two or three times. The Clark Nova and the Martinelli typewriters are fabulous!!! ~\\

I go out. I leave the hotel by 10PM. I take a cab and ask the driver for a good bar downtown. In less than 10 minutes I'm standing at the door of "\textbf{L'Aura Automatica}". It's a very well decorated bar with arabic ambient music. Just what I expected, and somehow, needed. ~\\

The bartender tells me after my second shot of whiskey that someone from Russia wants to speak to me. How the hell does anybody know I'm here??? That's still a mistery to me... And he tells me that the russian agent will arrive shortly at the bar. I wonder what the hell he or she wants from me. At this moment I don't have the slightest idea. I wait for 20 minutes untill the bartender tells me that a russian lady, seated at the other end of the bar, wants to talk to me. ~\\

She tells me her name. Natacha is from Gotalonia, a lost city that nobody knows exactly it's location. It's in the north, she tells me, but I guess everybody from Gotalonia says that inspite that the weather there has nothing to do with the northern countries, there's a lot of heat in GotaloNia. It's a mistery city. And I want to go there. ~\\

She's not from russia. Or so she told me. she was born in magicland; \textbf{GotaloNia}. She's a real Gotalonian. I believe her. So what's the buiseness with me? Well, her answer is somehow vague and unclear. She wants me to test a secret NeXT laptop in Gotalonia. I never heared of such machine and I guess this is a setup made by someone who wants to see me dead. I guess I'm too suspicious... I do trust her. I made an arrangement for a meeting with her tomorrow at my hotel. I guess I'll have to leave the boat trip for a later date. And if there is a secret NeXT laptop, I'll have to try it. I'm really curious. ~\\

There's always too many people in the hotel. It's not a quiet place, even in the night. Sometimes it's a group of russians, or servs or french or german, all kinds of people. Tonight's one of those nights. There's a party downstairs. I decide to go and see what's happening there. The music they're playing right now is hardcore techno. Such a bad taste. ~\\

I spend about an hour with a group of german people in their 50's. I drink a lot of vodka in that time, then I decide to take a walk outside. Just to clear things in my mind. And it helps a lot, even if I smoked cigarettes after cigarettes. Someone at that party tells me about a NeXT laptop that was on display at a software fair in Munich a few years ago. So the computer really exists, inspite that every site on the MatriX tells the opposite... Maybe I'm dreaming or hallucinating... I'll find out when I'm in Gotalonia. I'm pretty anxious. I can't wait much longer. If that laptop exists, it should be a wonderfull machine. I'll have to buy a digital photo camera. Then I'll make a website about the NeXT laptop. It should receive a lot of hits... ~\\

Meanwhile I return to the hotel and fetch my portable SGI tangerine [ from Tangiers-Morrocco ] and seat on the sofas near the lobby. I write a few pages about life in general. And that's just to clear things in my mind. A slleping cell is not a sleeping mind... ~\\

{\large ::: Since 16 March 2003 :::} ~\\

Back into writing this report now on another application titled "\textbf{HTMLEditoR}" and previewing in OmniWeb browser, which looks great by any standarts. The browser was made for NeXTStep back in the 90's and is now available only for OSX. It's quite confortable to be able to use this LucidaGrande font face on the Tangerine SGI machine. It makes me feel more at home. An SGI laptop with a source html editor? Well, the world is a weird place with weird things. ~\\

I have a different Icon for this report now. Made it with a shareware application titled Icon Factory. God, it's hot in this room; the temperature must me well above 45 celcius degrees. Extreme heat can be very annoying... ~\\

Surfing with Konqueror... Well... It's rather slow. I prefer Chimera or OmniWeb. And with all this heat it becomes unbearable working with a slow browser. It makes me realize that I'm in a desert and I don't like to think about that... Altough all this sand is becoming more and more present in my life. ~\\

I remember, when I was just an innocent child, noticing the size of the fonts in the books I read. I was very observant of that aspect. Every new book I started reading I imediatly measured the size of the font they used for print. I liked the small sizes better, I recall. That's one of the reasons I still use small fonts for all my reports. I'm now using the Apple Garamond font and it looks beautifull. Clean and elegant. The way all reports should be. ~\\

\vfill
\columnbreak

There was, this morning when I woke up, a note about the NeXT laptop. It was obviously handwritten by the GotaloNian woman and it says: "please meet me at the lobby at 7:30. I have good news about the NeXT laptop. Do you know Corto Maltese?". ~\\

Could I write this report with that machine? I think I'll know it in a few days or weeks. Meanwhile I'll google for this machine. Not sure what the results will be but I must gather some information. ~\\

\textbf{Blade Runner} in the desert. That's the film showing right now on the tv set of the hotel. The room is deserted, I'm the only person here. I watch the film with a great pleasure. And I think of large cities without deserts around them. I miss GotaloNia. And I miss a lot of people and a lot of other stuff. ~\\

It's 7:30 and the woman from GotaloNia arrives at the hotel lobby. She tells me to follow her to one car parked outside where she has a NeXT Cube waiting to be unloaded. She helps me setting up the cube in my hotel room. She leaves and tells me that if I like the Cube I can have it for 600 euros. I have one week to decide. ~\\

The \textbf{Cube} it's loaded with \textbf{NeXTStep 2.2} and is connected to one \textbf{17" grayscale NeXT monitor}. The mouse and the keyboard are also original NeXT gear. There's only one MP3 file on the hard disk, it's "Hack Attack" by Sigue Sigue Sputnik. Since this music is from 1986, it's apropiate... Still, I wonder why this MP3 file was left on the hard disk, since there are no more personal documents there. ~\\

A NeXT Cube was used to create the "http" protocol and the world wide web. It's the most famous achievement of this computer. The original Cube is on display in London at the computer museum. But this is just what I saw on the MatriX, I've never been there. And I guess I never will, never been too interested in London... ~\\

I send all my reports to an email account from my SGI portable and download them into the NeXTCube in under 15 minutes. I'm at the very small desk on my hotel room typing away on the NeXT keyboard this report. I'm listening to Sisterhood's "Gift" album that I ripped to the hard disk of the NeXT machine a few hours ago. It's almost mid-day and the light that enters the window of my room is very intense. And it's very hot too, I'm sweating a bit. Not too much though... I'll have a shower before I go out to have lunch. Right now a cigarette is burning on my lips. A small step towards personal destruction. There's a mosquito flying around the screen and that's annoying me. Another heat induced problem. But what did you expect being in the desert? ~\\

The machine is not blazingly fast, it only has a 25Mhz processor, but it performs most tasks at a very reasonable speed. And word processing, or report writing, is not that demanding from the processor. Even when I scroll back and forth between pages of a document. It behaves as I expected it to behave. It takes about two to three seconds saving a text document, so it's not as bad as someone would thought of a 25Mhz computer. Maybe because there are 64MBytes of RAM being used everytime by the computer. ~\\

First track "\textbf{Jihad}". I hope none of the people here will know why I love this track so much. I guess not... I'm making less and less sense in this report. This track has the strenght of a thousand voices... A wild call to "Jihad" in this city in the midlle of the desert... And it blends well with this black and white NeXT user interface. ~\\

::: There's nothing here but my hotel room, my SGI laptop, the NeXT Machine and extreme heat. All else is desert. At least to me. And it shouldn't be like this at all. And this report is part of that desert too. It's sparse and full of sand. And does it really matter to me? I guess I'll never know. Well, let's continue. ::: ~\\

Although I write most of this report on the SGI laptop I'm also very fond of the word processor of the NeXT machine. It's called "WriteNow" and it's on the version 3, at least on this machine, and it's very stable and elegant along with the overall feel of the system [NeXTStep 2.2]. Another reason for me to like this machine. Enough about the NeXT. Let's change subject... ~\\

In a desert you don't have many choices. You can't do anything. You are forever stuck with a few possibilities. Liberty is almost gone. "Where to?", well, that's still is a mistery to me. But I still have acess to the MatriX, and I still have my laptop... And for a few days or weeks I'll still have this vintage NeXT Cube. Paradoxal, I believe. But, nevertheless it is true. ~\\

I go down to the hotel bar to have a cup of coffee and to get a sense of the general atmosphere today. The bar tender tells me that the news in town today is that they're expecting Madonna to arrive here at the hotel with a few dozen people. She's supposed to start filming a musical video clip for her new album. I wonder if that's true. I wonder if she'll like this shitty hotel, but I guess she's been used to it during those poverty years of hers back in New York City. He tells me that she's arriving here tomorrow and that they prepared 4 rooms for the people that she brings along. But not all the people that comes with her will be staying here, only the top important people will... The rest of the video team will be staying at an even shittier hotel two or three blocks away. I wonder how long will they take to shoot all the scenes for the video... I guess they'll be gone in two or three days at maximum. But that's just my impression, things may be very different, you never know... ~\\

Another bit of news that the bar tender gave me left me with a surprised look. There's a cybercaf{\'e} in town! It opened a week ago, or so he tells me, and it's very cheap. I decide to go there this afternoon. I'm told that the caf{\'e} opens at 4PM and closes at 1AM. I'm really curious about what people should I find there... And what kind of machines, and how many, do they have there. The bar tender doesn't know the name of the cybercaf{\'e} but he knows where it's at. Right in the center of the city, I know the street well because I used to go to another very small caf{\'e} located nearby on that same street. ~\\

I have a light meal and a big cup of orange juice, it helps me to get away from the heat. I look around to sense the atmosphere of this joint. Everything's calm and quiet... Another quiet day in the desert. ~\\

%% \vfill
%% \clearpage

\textbf{"Reminiscences N\degree 1"} ~\\

I'm on the plane to GotaloNia. The year is 1999. The flight began a little more than 20 minutes ago, maybe half an hour ago. I'm feeling pretty homesick. I can't wait to walk again on GotaloNia... And I'm going to GotaloNia mainly becase of one very rare NeXT LapTop that some people are going to sell me for a just amount of money. And then... Boom!!! ~\\

\textbf{End of "Reminiscences N\degree 1"} ~\\

Life in the desert goes well. All the heat is just an excuse to slow down all kinds of activities. Including all room service activities. I ordered a thermus of tea almost an hour ago and they didn't bring it untill now, so everybody's on slow motion... The NeXT Cube doesn't complain about the hot weather, after all, the 25Mhz processor doesn't produce that much heat. ~\\

\textbf{ {\small ::::::::::} } ~\\
\textbf{A few weeks later...} ~\\

The heat is unbearable. And there's very little I can do about it. ~\\

This is not Iraq or Jordan or Saudi Arabia or even Algeria, this is a desert in the middle of the desert, much like GotaloNia... This is nowhere lost between nothing at all. Even the people in this place seem like ghosts sometimes. All dressed in white with stains of dust all over their clothes. ~\\
\textbf{ {\small ::::::::::} } ~\\

The most useful application in the universe is called LaunchBar and runs on NeXTSTEP. I have it installed on this Cube. Just hit "command"+"space bar" and type whatever document, folder or application you want and then hit "enter". It's as fast and simple as that. And LaunchBar learns from your habits so, for instance, if you wanna launch the texteditor you just type "t" and hit "enter" afterwards. How cool is that? And there's a version for Mac OS X too. ~\\

Cigarettes taste good with this heat wave. I can never get enough and, as a result, I'm smoking more each passing day. Which reminds me that I have to send this week's report today and the internet connection has been failing a lot for the last couple of days. But I don't really care if I send it in two or three days from now. I guess it won't make any difference since the report itself has very little substancial items in it. Basically it's just a wild bunch of random thoughts and observations. ~\\

The sweet sound of Trance. It's been a long time since I last heard trance. I guess, more than a year now. Dream induced music? I guess so. I wonder what it would be to have a rave party in this desert with these people... Crazy night, I guess. ~\\

\textbf{ {\small :::::::::Last day of August 2003::::::::::} } ~\\ 

Just received an email from my "upper contact" asking my opinion about having a weblog for the report from the desert. At first I thought that was a stupid idea. But now, on second thought, I tend to agree with that choice. It would save me the pain of not logging in to the mail server, since I have almost no problems dealing with the worldwideweb part of the MatriX. So I'll open up a free website to post the reports which will have a bi-weekly appearence on that site. I guess the boss will be happy... ~\\

Just created a website and uploaded the first report; just two or three short paragraphs. Everything went out smoothly. The tool I'm using to create them is the very good OmniWeb SourceEditor. Very simple and elegant, just what I needed for the reports. OmniWeb is the best browser for the NeXTSTEP environment. It's on version 3.5 for this OS while it's already on version 4.5 for the MacOSX systems. But this old version works just great for my needs. I'll have to look for an iBook with OSX installed, just in case something goes wrong with the NeXT Cube. And because I miss OSX... Even though, OSX is just an evolution of NeXTSTEP. ~\\

I love writing HTML documents in text editors. It gives me total control on how the page will be presented. There's too much heat for me to go to the streets, I'll have a nap instead. Sleep repairs a troubled soul, or something vaguely like that. And with a change of subject like this, it sure will do me good. ~\\

And this NeXT machine is behaving like no other that I've owned before. Rock solid, or maybe like steel... Heat, sand and whatever is around me. Confusing further and further my mind. I guess it will never stop, I mean, this mind decaying... What I am saying here? Nothing, I guess... ~\\

\textbf{ {\small ::::::::: another small intermission :::::::::} } %% ~\\

The situation around here is pretty quiet. Nothing really develops here. There's no evolution or the slightest signal of life. Everybody seems to mind their normal day to day lives and nothing more. Not even in this hotel the tourists seem to be doing anything special. Everybody seems dead... This is starting to feel like a zombie city. I guess I must go, as quickly as possible, to GotaloNia... I've been thinking about it for the last few days. ~\\

Here I am again in this desert. Trying to figure out what the hell these words are for. I guess I'll never know, so I'll just type away and never give another thought to what I'm writing at any given moment... This is getting more and more confusing. Stability of the mind. A big subject. The NeXT Cube sits there on the table, quietly. Je suis fatigu{\'e}... ~\\

This is starting to feel very weird. I mean, this report. It has lost all logic thought that was supposed to be here. Where did it go? I wonder if it'll ever come back. By the way, rain started to fell on the desert tonight. I wonder if it means that a much cooler weather is coming toward us... I guess not, the hot weather will be with this desert for a long time, maybe except at night... Which is a big stress for the NeXT Cube and all of its components. But it will handle it just fine, I'm pretty sure about that. And I'm sure that, in a few days, I'll be heading to GotaloNia... ~\\

I'm trapped inside this fucking desert for too long. Going to GotaloNia to test the NeXT portable. The ultimate myth. The dream machine that drives my will without my knowledge. I'm too tired of this heat and this emptiness that is this desert. I feel like Corto Maltese. I feel the call of the wild, whatever that means... I guess it means that for me the "wild" is spelled "GotaloNia". I look up on a website the timetable of this weeks flights. I choose tomorrow as the date. I make a quick phone call for a ticket reservation and then start to pack one suitcase. That's all I need; that and a rain coat. You never know how's the weather in GotaloNia. Sometimes it's very hot like the desert and sometimes it rains for two weeks without almost any break. It's very unpredictable, especialy when you think about the fact that GotaloNia can be in the southern emisphere or near the equator or sometimes near one of the frozen poles. That's a weird shit about it. I wonder how the pilots of the planes that fly towards GotaloNia, like the one I'll be taking tomorrow, know exactly where the city is at... Weird shit like that will never stop making me wonder. And I thought about this a lot of times before. I just don't understand. I guess I have to think harder. ~\\

I wonder if it's raining there. I wish it were. What really worries me now is the state of mind that I wil face when I arrive there. I hope I'll feel better. ~\\

\textbf{ {\small ::::::::: another small intermission :::::::::} } %% ~\\

\textbf{ {\small ::::: The trip to GotaloNia is postponed :::::} } %% ~\\

\textbf{ {\small No reasons why, apparently...} } ~\\

Sand fills every hole visible in a thousand mile radius. Whatever that means, I don't know. Don't know exactly why but I'm listening to the "Shutov Assembly" by Brian Eno. To relax, I suppose. It suits well the sand all around. It cools my head beneath this heat. It helps me find a better way of thinking. It makes me wanna organize my thoughts in a cleaner way. It induces me in a good state of mind. Nevrmind the fact that all tracks sound almost the same, that's no coincidence. It's meant to be that way. And it sounds better that way too. ~\\

Smoking "Detroit" cigarettes. Watching stupid american movies on tv, downstairs. Life is dull, sometimes. A terrorist attack in Istambul makes me want to listen to the Young Gods [ TV Sky ]. Don't know exactly why. I guess our minds work in mysterious ways... And maybe not just that but it's because the tv sky album reminds me of bombs exploding. It could be that, and it's a simple explanation. And, by the way, Switzerland isn't very far away from Turkey and Istambul. Well, let's just forget about that last assertion, ok? Time for Beck and the exellent "Sea Change". ~\\

Transparent mood. That's what I'm feeling like now. I really wish it was raining here. Reminds me of "hold back the rain" 12" remix by duranduran. ~\\

But things tend to be very different when the rain comes to this desert. Don't know when that will happen, of course, but when it does happen I'll be the first one on the streets enjoying it. Dreaming of the ocean and the wide beaches back home. Even the wind blows in a different way, can't explain why... It's not as hot as here in the desert. ~\\

{\large It's been raining for three whole days now...} ~\\

And the journey to Gotalonia never seemed this close to me. Never. So I go to the bank to get the necessary money to buy the plane tickets. I'm thinking in going in the next 7 or 15 days. Maybe the weather here will come to a normal state by then. I want to leave the desert in a hot fashion. So it seems more natural to me. Maybe I won't miss the desert, but what do I know? ~\\

The SGI laptop is having a lot of work lately. Besides this report I've been writing other html's in the last few days. Don't know if it's because of the rain, but I highly doubt it. Nevermind though... One of the texts I've been writing is about the NeXT computers and Tim Berns Lee, the creator of the WorldWideWeb and HTML in some form or another. And this laptop has been behaving rather well, even beneath this humidity. ~\\ 

replanning the trip to GotaloNia ~\\

Travelling on Air GotaloNia has always been a good and pleasent experience. They only have two types of aircraft; the Lockheed L1100 and only two Concordes. and that makes them the only company in the world who still has Concordes flying. Unfortunatly the Concordes are not used in this route so I'll have to travel in one Lockheed. The trip will last about 4 hours. Maybe a little less if we're lucky with the wind. There's always a certain amount of uncertainty, but that's life, isn't it? And I do like surprises when I travel, who doesn't? ~\\

{\large this desert is just a loop} ~\\

Can't understand it... Tired of loops. They never end, do they? Which reminds me of another thing: don't listen to The SWANS' loops, ok? They're evil and grow on your head like cancer. Or something like that. And this is just not to have to talk about AphexTwin's loops. They sometimes make you feel you're someplace else. Like being transported to another reality instantly. ~\\

end of loop ::::: return to start... ~\\

I finally have a ticket for a flight to GotaloNia. Gotta have the SGI laptop ready for the trip with all my personal documents which I have stored on the NeXT Cube for now. My personal documents bring me to another subject. A different one. A very different one. It's about sadness and the state of being all alone. Just like my present situation here in the desert. How good is it to anyone to be able to sit alone in a room with a cigarette in one hand and a glass of martini in the other? It's not that simple to answer that. In fact, it's very difficult to find a suitable answer to that. ~\\

Ok, let's assume that there's really nothing good in being alone in some desert. So, how come I'm here if I don't have a valid reason to be? The hunt for some rare NeXT laptop that nobody really knows if it exists? Is that a valid reason? I wonder. In my case, I think that any reason is a valid reason. So, why worry? And, by the way, isn't that the same reason I'm getting out of this desert and go to GotaloNia? Yes, that's it. Another endless loop. Another afer-effect of this desert heat that I don't know any more if I really like it or not... ~\\

I must find that NeXT portable in GotaloNia. And, deep within me, I know I will. ~\\

%% \vfill
%% \columnbreak

{\large -------- Reality check:} ~\\
There was a NeXT laptop, I've seen the photo on some NeXT website but it was never released to the stores. I wonder how many still exist... It was called NeXT 9200, I think... ~\\
{\large -------- Reality check ends here.} ~\\

Sometimes when I think of the NeXT laptop I remember seeing, once, in a magazine an article about a Sparc laptop capable of running both Solaris and NeXTStep. It was supposed to be launched in 1998 and in a very small quantity. Only 2000 were supposed to ship from the asian plant where almost all Sun hardware came out to the public. That could be a wonderful machine for me. I never knew if that machine really got out to the hands of the public or if only a few prototypes were made. That's another very valuable machine that I must find someday in the future. Maybe it's not that far away, that near future when I'll find it in the hands of someone who can't give it it's real value. Maybe I'll find one in GotaloNia. I'll have to ask to the person who is giving the NeXT laptop to me. Maybe she'll know something about that Sparc powered model. Maybe it's not made by Sun but by some other company with access to Sparc processors. I have to find out. Although it's a bit strange because a lot of Sun people are using Apple's MacOSX latops... Strange world, isn't it? It seems that everybody that works on some Unix flavor is craving for an OSX laptop or has already bought one. ~\\

{\large The art [ heart ] of the OSX laptop} ~\\

"It's NeXTStep on steroids with a beautiful face enclosed in a brilliantly designed hardware package". That's just one of the million opinions praising Apple laptops loaded with the excelent MacOS X. Many more can be found throughout the web on reviews, on blogs, etc, etc. So why didn't NeXT released a laptop? Well, the answer might lie in this: by the time laptops began to hit some niches of the consumer market in 1993, NeXT decided to quit the hardware business and focus only in the software business. That might be the right answer. But they did try to make that 9200 model. And a few prototypes were tested, not sure how sucessful they were though... And the price would eventualy be too high for a consumer laptop. Performance might be another issue, although I'm not too certain on that. So, when Apple bought NeXT (or was it the other way around?) and was mainly interested on making a really good operating system based on NeXTStep, the NeXT laptop became a feasable product. Much like a dream that came true. ~\\

%% \vfill
%% \columnbreak

One of the first laptops that were built to run OSX were the \textbf{Clamshell iBooks}. They were introduced in September 1999 in two very appealing colors: bondi blue and tangerine/orange. They were equiped with a 300Mhz G3 CPU and came originally with 32MB of memory that on the revision B was upgraded to 64MB. When the first public beta of OSX was released in early 2000 a lot of iBook owners upgraded their memory banks to at least 128MB total, the minimum required to install and run OSX. Of course a lot of people bought more than 64MB additional RAM... So the first iBooks were, in a way, the original NeXT laptops. And they even came with a handle!!! So, by mid 2000 everyone with an iBook could carry a machine loaded with the latest version of the BSD/Mach Kernel based NeXTStep, now known as MacOSX. Finally the NeXT inc and it's software began to feel alive again. And ancient software developers that focused their businesses on NeXTStep began to breathe again with the advent of MacOSX. And to have beautiful machine to develop for was a big step forward in their self confidence and bank accounts. One good example of this is a company called "Objective Development" that have a brilliant piece of OSX software that goes by the name "LaunchBar". It's used by people that don't like to use the mouse too much and prefer the keyboard shortcuts instead and by people who don't like the Dock [one of the best features of OSX]. ~\\

Many people in Europe and in the U.S.A. began to look at their aging NeXT Cubes and Slabs and decided to buy Apple hardware just to have once again the pleasure of running NeXTStep on a modern machine, preferably one laptop. ~\\

{\large The trip towards Gotalonia and the NeXT laptop} ~\\

GotaloNia is bathed by the Mediterranean sea. At least that's what the legend says and the fact is that nobody really knows for sure the name of the sea or even exactly where GotaloNia is. But still, most people say it's indeed the Mediterranean. Of the famous people that live there we can count many; Dali, Corto Maltese, Gaudi, Antonin Artaud, Andy Warhol, John Coltrane, Marilyn Monroe, Catherine Deneuve, and many, many other famous people. And Natacha, as well. ~\\

The city is separated, like many european cities, between the center with old buildings and the peripheral part with state of the art modern buildings. The place where I always stay whenever I go there is in an old building in the center of the city barely five minutes walking from the water front. It's a very nice and quiet neibourghood. It's also very near to a subway station and from the central plaza that holds most of the buses lines in the city. ~\\

I arrived at L'Alma international airport in an Air GotaloNia Concorde at 11 in the morning and 35 minutes later I was in the center of the city and headed to the usual place to rent a room. My contact in GotaloNia was a phone number of the person that had the NeXTbook 9200 model I was craving for. I planned to rest for a few hours before I made that phone call and eventually arranged for a "meeting" with the 9200 owner. ~\\

After about two hours resting in the room and writing for a few minutes this report on the SGI laptop using OpenOffice from Sun I went out to eat something and have a large cup of cortado coffee that nobody makes this good in the whole world like in GotaloNia. I'm wondering how many people here are using Irix to work on computers... I know OpenOffice is very popular in GotaloNia but SGI machines are by defenition a rare thing. ~\\

I walk into L'Opera [ number 51 on the main avenue that leads to the sea ] for a quick sandwich and a large cortado. The person behind the counter seems to recognize me even though I haven't been to Gotalonia in over 4 years. People here have good memories, or so it seems. Well, I missed these cortados, that's for sure. I buy another pack of cigarettes and ask for the public phone. Then I search in my pockets for the card with the telephone number of the 9200 model contact. I dial the number and wait for a long time before a male voice answers with a funny accent. "Hello, who is this?" I introduce myself and I tell him about the 9200 model. His answer is reassuring saying that he, indeed, has such model for sale and can arrange a meeting with me the next day. I ask him if he knows where L'Opera is and he tells me that everybody in GotaloNia knows it. I ask his name and he says he's called Corto Maltese. That's not a complete surprise. He then tells me to look for a guy in a sailor hat with a laptop computer. It's unlikely that I'll find anybody like that, other than Corto, in L'Opera. Great. I'll have a look at the 9200 machine tomorrow. Everything's looking good. ~\\

The music that's playing in L'Opera also came as a surprise: Peter Murphy's Dust album. How cooler can this get? Not much more, I garantee you that. And after the Dust album was over Dead can Dance began to play. I'll stay here for a few more minutes and read some magazine or newspaper. I grab the New York Times and read a long article about terrorist cells working in the New York area. The article grabed my atention because it said that the NYPD found a "lost" Apple laptop with plans for terrorist attacks as well as very detailed maps of New York City. The operating system was in arabic, the newspaper also said. ~\\

Left L'Opera and went directly to the water front. Spent the rest of the afternoon there. Went for dinner at 9PM at a restaurant I've never been to in the Port zone. Ordered a bottle of red wine and a simple steak with french fries and a salad. ~\\

This night in GotaloNia is pretty cool, contrasting with the last few nights I spent on the desert. No horrid hot temperatures at midnight or 2AM... But not cold either, just right. Went to the cinema to watch a movie about the lebanese war. But didn't like or pay much atention to the film because I just can't stop thinking about the NeXTBook. What price will Mister Corto Maltese ask for it? What software is loaded on the laptop? I wonder... ~\\

The place where I'm staying at has this ocean waves atmospheric ambient soundtrack in the lobby all night long. I wonder why. I'll try to remember to ask someone before I leave back to the desert, hopefully, with the NeXTnotebook 9200. ~\\

Before I go up to my room I enter into one supermarket to buy a bottle of Martini Bianco and a pack of cigarettes. I switch on the SGI laptop that's placed on the desk and fire up the internet connection to check for email. I have quite a few messages I really need to reply. As a soundtrack I choose Air's Walkie Talkie and begin to sip martini and write whatever feels right in my mind. ~\\

I have an email from the porter of the hotel in the desert asking me what it feels like to be in GotaloNia again after so long. I found that email quite funny, coming from that guy. In return I ask him how the desert is doing without me. Of course he'll answer something stupid but that's life. And that's what friendly email is for... ~\\

And, yes, I have an email from the russian lady that I met in the desert hotel. Except, as I already knew, she's not from Russia but from GotaloNia and her name is Natacha. That's two Natachas that I know that live in GotaloNia. She's the woman who gave me this NeXT laptop contact; the telephone of Corto Maltese. She asks me when will I be in GotaloNia so she can meet me for lunch or some sort of rendez-vous. I reply imidiatly to her and give her the phone number of the hotel I'm staying at. I also tell her about my meeting tomorrow with Corto and the NeXTBook 9200. ~\\

%% \vfill
%% \columnbreak

I receive, almost imediatly, a reply from Natacha saying that she'll telephone me tomorrow in the morning to arrange a meeting. She's also interested in seeing the NeXTBook. I tap the SGI laptop and wonder what the hell does she want with the NeXTBook... Maybe she's just as curious as me. Or maybe she's interested in taking photographs of this very rare laptop to sell them to a computer magazine in Japan, or... Oh well, I'm flying high again... ~\\

{\small found in the trash bin:} ~\\
	.......... ~\\
    The thought first entered my mind while I was talking to someone in the lobby. Why would someone want to setup a NeXT cube in a large SGI network? It would have to be someone with ancient work habits, for sure. And maybe that someone should have been warned about the almost endless possibilities of an SGI machine. So I decide to make a few phone calls to find out what the hell the cube was meant for. ~\\
	.......... ~\\
	%% .......... ~\\
    It came as no surprise; the cube was to be the main machine for html code editing. And the NeXT cube of the "inventor"of html came also to my mind. Was the choice of a NeXT cube related to this fact? The answer came my way a few days later when I had the chance to talk to the person in question. ~\\
	.......... ~\\
	%% .......... ~\\
    He told me that his interest in doing html code in a cube had to do with his wish to stay faithfull to the machine that first felt html in the whole world. Well, that's what he told me, anyway. I would soon discover that this was not the reason at all. The real reason was a dangerous one. The person in question had some kind of connection with a russian software company. ~\\
	.......... ~\\

I wake up at 10AM. I shower, shave and have a very large cup of cortado while browsing through the news on New York Times and a few other news sites. There's nothing outragiously new to be read, just the usual stuff. When I disconnect from the matrix I refill the cup with another obcenly large cortado and another cigarette. It's my way of waking up to an important day. The day I'll finally see the NeXTBook 9200. ~\\

{\large a few hours go by:::time for business} ~\\

I'm on my way to the L'Opera to meet Corto and the NeXTBook. I get there 10 minutes early so I order a martini and a cup of cortado and start to read the local newspapers. There's a full page ad on a computer shop that sells only second-hand laptops. I take note of the address on one of my wallet papers and plan to go there later today. ~\\

Corto gets here on time and greets me with politeness. He's bringing the famous NeXTBook on a black bag. He opens it and takes the NeXTBook out and places it on the table. It's bigger than I thought it would be and it has a trackpad instead of a trackball. I'm glad because of that since I much prefer trackpads. He opens the latch and raises the screen. Turns the machine on and \textbf{NeXTStep} begins loading. He then starts to show me what software is loaded on the hard drive of the laptop; \textbf{WriteNow} [his most used software], \textbf{OmniWeb} for web browsing and HTML composing and not much else. A copy of \textbf{LaunchBar} is also installed. He then tells me that he has it for over 6 years and mainly uses it to write HTML and to move around the matrix and check email. ~\\

I ask him why he's willing to sell the machine. He answers that he bought a Sony Vaio almost a year ago and he doesn't need the NeXTBook anymore because he has the x86 version of NeXTStep installed there as well as BeOS5 and an old version of Windows 95. He also tells me that he is using more and more BeOS because he found a truly amazing piece of software that only runs in BeOS called "\textbf{The English Editor 2}". He explains that EE2 is a \textbf{XML/XHTML editor} with an incredible user interface and that it suits all his needs for writing. EE2 saves XML/XHTML files that can be read by any web browser. ~\\

We discuss the price. He's asking for 500 Euros. It does seem a reasonable price to me since the NeXTBook is a piece for collectors. We agree on the price and I write him a check with that value. He then invites me to his sail boat to check his Vaio with BeOS5 and EE2. I accept and we leave L'Opera heading down to the harbour. ~\\

Corto's yatch is named Escuna and is quite old, although is very well preserved. Corto asks me if I want a cup of tea and heads down to the cabin to fetch the tea and his Vaio while I seat at a table in the deck. Corto brings from the cabin the tea pot, cups and a box with natural tobacco and rolling paper as well. He then start to roll one of this very famous cigarettes and talks about the beauty of EnglishEditor2 for a few minutes. He then plugs the Vaio to a power outlet on the stairs to the cabin and switches on the laptop. He has it setup to boot BeOS5 on default. It loads extremely fast, under 30 seconds or some sort of a period of time like that. He then launches EE2 and I'm truly amazed at the simplicity and emptiness of the interface. And the beauty as well. ~\\

%% \vfill
%% \columnbreak

About an hour later I ask him if I can plug my NeXTBook as well to start exploring the old machine and it's performance. Well, it's NeXTStep and that OS never stops to amaze me. The performance is also good, I thought it would be somewhat worse, but fortunatly, it's not. It's quite responsive for such an old equipment. I think it's even a little faster than my NeXT Cube. And Corto says that the battery lasts almost two hours which is very good for an old machine. After all this is \textbf{1999}, or so that's what I believe. But it could be any other year, because it doesn't really matter. ~\\

This laptop has almost the basic software I have on the Cube. There are a few programs that I never heard of that are also installed, namely a desktop publishing suite with all the menus in russian or some eastern language like that. But all the basic stuff is here and Corto had the attention to delete his own account and create a new one for me. The hard disk of the NeXTBook is only 3Gigabytes in size so it has little few free space, a little over one gigabyte. That's more than enough for me since I'm not planning in installing any more software on that, just my personal documents which don't ammount to much space. So this is the second laptop i buy in a few months time. The SGI first and now this NeXT. Along with the Cube it already ammounts to three computers. I think I'll stop for now. But then again I might need a x86 computer to install BeOS5 on along with English Editor2, of course. ~\\

{\large TheEnglishEditor2 revolutions} ~\\

Well it's a fantastic xhtml editor in all senses and perspectives possible in this world. There's nothing else like this editor in the whole world of xhtml editors in the vast galaxy of operating systems. And Corto knows this also. He shows me how the look of a document can be changed by editing a text file with the stylesheet's parameters. But the user interface is absolutly beautiful and the most clutter free I've seen in a long time, I mean, ever. Corto never saw, also, a user interface like this before in his life and points to me the "hidden" features of the interface. When you move the mouse pointer over each corner of the window [minus the upper left corner] the single pull-down menu appears or the scroll bar appears or a frames per second live counter appears. Simply amazing! Nobody thought of this before, not even on MacOSX applications I've seen this simplicity and functionality. ~\\

%% \vfill
%% \columnbreak

.......... ~\\
    Corto tells me that he's writing all his documents in this editor and he can send them to the matrix [ to his own website ] to be read by any browser capable of reading the UTF-8 encoding, which most, if not all browsers do nowadays. What about printing, I ask him. "Well, I don't print too much, I don't even have a printer. Whenever I need to print something I just go to the nearest CyberCaf{\'e} and print from there, it's not that expensive because I have the habit of formatting the fonts in really small sizes". Well that's what I also normally do. There's always a solution no matter what kind of problem arises with any system. Always. Sometimes you only have to dig deep enough. And the best place to dig is Google.com, as everybody already knows. And sometimes the solution floats on the surface almost hidden from sight because it's too obvious. ~\\

    He tells me that he converted a lot of his documents from whatever format they had into simple HTML docs so that EE2 could open them and edit them. There's a lot of word processors that can convert to HTML so that wasn't a problem for Corto. He told me that when he was using the NeXTBook he was mainly writing using the \textbf{OmniWeb Composer} and \textbf{WriteNow} was the first word processor that NeXTStep knew. He said he often just copy text from WriteNow and pasted it on OW Composer and then filled out the tags to make an HTML doc. ~\\

    We talk for another two or three hours and then Corto invites me to dinner in his yatch to celebrate my buying of the NeXTBook. I agree and he goes on to the kitchen to make some pasta with pesto, one of his favorite meals. ~\\

    \textbf{pesto scents in GotaloNia} ~\\

    While I was waiting for Corto to finish setting up the table for dinner I asked him if I could connect to the matrix just to check my email account. He said sure and I tried the modem on the NeXTBook for the first time. Everything worked as expected and I was able to launch OmniWeb and log on to my account. There was an interesting email waiting for me... My "upper" agent sugested me that I go in a few days time to Algeria and wait there for a further contact. He had a new mission for me. I'm surprised but also pleased to go to Algeria. And I'm sure I'll like the mission that's assigned to me. But that also means that I'll have to leave GotaloNia in two or three days. ~\\

\vfill
\columnbreak

    {\small :: the smell of Algeria ::} ~\\
    We dine while the boat rocks gently and softly. We have almost two bottles of red gotalonian wine which make me feel a little dizzy and snoozy. We rest in the boat talking for another two or three hours before we go out to have a cortado in some bar in the center of the city. I ask Corto if he knows Natasha but he never heared of her so my hopes that I'll find her in the remaining two or three days here are kind of lost. ~\\

    We leave the Escuna after midnight and go to a quiet bar near L'Opera. Well, not as quiet as we thought. The place is almost full and people keep going in and out to the streets to drink and talk. I tell Corto that I'll be going to Algeria in a few days. He then tells me that coincidences do happen since he, also, is going to Algeria in a week or so in a work assignment. We will find each other there. He doesn't tell me the nature of the work he'll be performing and, of course, I do exactly the same. And even if I wanted to tell him I still don't know exactly what I'll be doing there but I already have my suspicions. ~\\

    I'm listening to a live CD of \textbf{Suicide} in 1998 that Corto had in the yatch. He has quite a feine collection. I imediatly asked to copy the files to my flash drive since I never heared of the existance of this CD before and it's quite good. Execelent sound quality and more than an hour long. And there's also wild versions of "Sister Ray" and "GhostRider"! ~\\

    {\small :: intermission [ desertic version ] ::} ~\\
	.......... ~\\
        How long is one year? 365 days seems to be the correct answer. Well, not in GotaloNia. Time here doesn't stop or wildly accelerate. It just flows like there's no tomorrow. Will ocean waves ever stop crashing into the shores? I guess not. One year is like 20 minutes when you have all the time in the world. ~\\
	.......... ~\\
    {\small :: END OF intermission [ desertic version ] ::} ~\\

    I'm doing the report on the NeXTBook in XHTML format. I use OmniWeb's Composer and nothing else. It's not the same version that's in the SGI / OSX notebook but it's almost the same. It does all the things I need/want. I think I really made a great deal with this notebook from NeXT. The software is brilliant and the battery runs a bit longer than two hours which is more than I need. The speed is also good for such an old machine. And it will look extremely well with the Cube. But that's not the point, this will be the portable I will take to Algeria. I'll mail the SGI laptop to my address in the desert. Turn on the NeXTBook and poured a glass of Martini. Having one cigarette. The report continues... ~\\

    I'm thinking about the mission in Algeria. If I'm right I'll be there for just a week or so. I'll be organizing some new unexperienced group and teach them some techniques. I hope it won't be much longer than one week. But, as usual, I'll only know the facts when I get there. It's a matter of personal and organizational security. It's just better this way. I only hope that I'll leave Algeria one or two days before the mission is fully acomplished, once again, for personal security reasons. Anyway, the mission will be completed as always without the need of any active participation during the execution day by myself. I don't like taking high risks, it's just my nature. I'll only take them if there's no other choice available at the moment. ~\\

    I order another coffee from room-service and try to think about the report. I'm really enjoying LaunchBar. It's almost the same as the MacOSX version I have on the SGI notebook. I spent about 15 minutes configuring the application to my personal taste and needs. I'm now organizing all the contacts in the Adress Book application. Once again very similar to the OSX version. I just wish there was a version of English Editor for NeXTStep... But I guess I'm out of luck. Went to the website that hosts the application and there's only one flavour; BeOS. But that's not a big problem because OmniWeb Composer is simple and effective enough for me. And the NeXT user interface makes it even more beautiful. Black is beautiful. I forgot who said that. It really doesn't matter. ~\\

    I wake up at 10AM. I'm thinking automatically about one big gotalonian cortado wraped with the taste of a brand new fresh cigarette. The european way of life, can't shake it away. I go to a small caf{\'e} on the other side of the street to enjoy the cortado. I grab a local newspaper and read a few articles. Then I remembered the ad I saw yesterday on another newspaper about the used laptops shop here in GotaloNia. I ask the lady of the caf{\'e} about the street where the shop's located and she gives me some simple directions. She tells me that it's only ten minutes or so away from where we are. ~\\

    I get to the shop in less than ten minutes. It's absolutely loaded with hundreds of laptops from all sorts of brands. I spend about half an hour just walking along the shelves looking at many models from all sizes and brands; Compaq, Sony, Toshiba, IBM, Apple, SGI, Acer, etc, etc. And the variety of prices are also amazing. A three year old laptop costs about 500 or 600 Euros. A four year old laptop costs 300 or 400 Euros. I decide to telephone Corto so he can help me out choosing a BeOS5 compatible laptop. he told me yesterday that BeOS5 doesn't support fully most of the laptops and it's kind of useless to buy anything that's three years old or newer. I walk out the shop and go to a public telephone booth to call Corto. He answers the call and we arrange a meeting in a nearby caf{\'e}. ~\\

    He gets there about half an hour later and we begin discussing a good laptop to run BeOS5 and, of course, English Editor 2. He says that a 4 or 5 year old laptop is more than enough. He also brought along a CD with BeOS5 and EnglishEditor2 in it so I can test the laptop compatibility in the shop before I take it home. ~\\

    I end up choosing an old Sony model with a good working battery and a 12" screen. It has a Pentium 266MMX processor and 64 MBytes of RAM. More than enough to run BeOS5 decently. The price also reflects the old age of the machine: 320 Euros. One important thing is that BeOS recognizes the video, sound and modem cards without the need of any third party drivers. Just what I was looking for. I leave the store with \textbf{BeOS5 and English Editor 2} already installed and ready to be used. This is one of the fastests computer buys I ever made in my life. So now I have three laptops and an old typewriter when electricity and batteries run out in the desert. I'm quite equiped for everything. But the cherry in the cake is the NeXTBook. Not only for it's beauty but also for it's rariness. Only a few hundreds of these machines saw the light of day, and I'm one of them with a working machine. ~\\

    We go to Corto's Yatch to install some additional software. Corto has a few CD's with BeOS software including ArtPaint, BePDF reader, PDF writer, FireFox web browser and a few others that are of interest to me. He also pointed me to a very useful website to download software; bebits.com. Almost everything available to BeOS is there. And most of that software is freeware. BeOS5 also come with another gem in audio applications: \textbf{3DMix}. It's a wonderful sound editor, and as EE2, it's user interface is at the minimum revolutionary. I'll try it more seriously in the near future, I'm sure. He also own a few books by \textbf{Aleister Crowley} in PDF format which I also copy to the hard drive. One of those books is the complete "AC confessions" book which I have been searching for in the last few years. ~\\

    We have a curious conversation about the \textbf{Xanadu} project by Ted Nelson. Corto says that the original idea has been surpassed by the rise of the world wide web and HTML in general. I agree with him. We wonder how Ted Nelson might have felt when he saw the first bursts of light of the world wide web and the first browser and html editor. He would have freak out! He also shows me an interview of Nelson he had in an old magazine. The man was obviously a crazy person! But he also, without any doubt, was a genius. Because he envisioned a lot about how people in the future would work with documents inter-linked throughout a big network. He saw the world wide web in many of it's present aspects. And he wanted a more well designed web. The documents would have, in his vision, two way links instead of just one way links. But I guess that's not quite possible or of any interest right now. But he never really gave up his own dream and is still trying to give the Xanadu Project a new life. But since HTML and the world wide web is so entranched now that I really doubt his project will ever be accepted by anyone. But we never know, do we? ~\\

    I end up installing a lot of HTML editors that Corto shows me on his machine. There's one in particular that I find interesting; it's called \textbf{Globe HTML Editor} and was made by someone in Hungary. There are a lot of HTML editors for BeOS. It's kind of strange, then again, maybe not. But nothing really compares to \textbf{EE2}. There's nothing like it, period. ~\\

    BeOS was known for it's little yellow tabs in the window title bar. It sure is original and no other OS has adopted that design. I guess it only looks good in BeOS, not on other systems. There was once, or so Corto tells me, a website with that name "little yellow tabs" that focused on BeOS news and opinion. It's now defunct since BeOS itself died a few years ago. Nevertheless a few people still manage to develop applications for it. probably with a lot of dreams in their heads and a lot of caffeine in their veins. Those people are resistants, they still believe BeOS has a future waiting somewhere. Maybe they're right. We never know untill some time passes by. It would be wonderful if people would break through and delivered some insanely great applications for BeOS. Like some other people do now for MacOSX and other operating systems. One of the reasons, the strongest one, for me to buy a sony portable was English Editor 2. If that application was available for MacOSX or IriX I wouldn't have bought it, surely. But nevertheless I'm quite happy I bought the sony laptop. It's always a pleasure to write in EE2. The other good experience I've had in writing was with two apps for MacOSX; \textbf{TextEdit} and \textbf{SubEthaEdit}. Well, almost every editor in MacosX or NeXTStep has a beautiful interface. If not all... ~\\

    I just shiped by airmail the \textbf{SGI laptop} to the hotel in the desert earlier today. The package is going to be picked up by the receptionist of the hotel. I just phoned him advising of the arrival of the package in the next few days. So I'm down to two laptops: the NeXTBook and the Sony with BeOS5. I wonder if I'll miss MacOSX too much. I guess not because of the NeXTBook with NeXTStep. In fact I'm writing everything on the NBook with \textbf{OmniWeb Composer}. And I'm very pleased with the responsiveness of NS and OW Composer as well. Anyway I'm planning to download all the documents to the Sony machine just to get the feel for a few days of EE2 and BeOS5. So I make a backup to the flash drive from the NBook and then plug the drive into the Sony. I zip all the documents just to test if BeOS can unzip them. It can, as I thought. ~\\

    I've been writing a report on the two machines I bought in GotaloNia to my superiors. All the software installed and the laptops' specifications are also included. I emailed the report in XHTML format [regular html tags plus other custom tags] from the Mail.app here on the NBook. All went perfectly well, as expected. I didn't have any answer to an email I sent to the russian woman that lives in GotaloNia. Strange, she said she would answer quickly. Well, it's her interested to see the NeXTBook, not mine... ~\\

    Corto tells me he's going to Algeria in a few days on his yatch to make "some business". I tell him that I'm also going in three days to Algeria to, also, do some business. We smile knowing that our mission are somewhat connected. I'm sure we'll meet there. Maybe Corto is one of my contacts there for the mission... ~\\

    {\small Dawning GotaloNia} %% ~\\

    Because the weather was so good last night I ended up sleeping in the deck of the yatch in a sleeping bag above a thin layer of foam. Woke up at 7:30AM to watch the morning rise above GotaloNia and the port. Made a cup of cortado on the yatch's kitchen and went back outside to enjoy the atmosphere a bit more. Rolled up one Amsterdammer and listened to the seagulls flying overhead. Fetched my NBook and continued writing for a few minutes the report. Had to connect the NBook to the power outlet because the battery ran dry after more than two and a half hours. Then I decide to log on to the matrix to check my email accounts. Yes, there was a reply from the russian woman asking me to meet her at some caf{\'e} at 5:30PM today and if I could bring the NBook with me. I reply a short "no problem, I'll be there". ~\\

    At about 11 in the morning I write Corto a short note and head back to the hotel. I go to the plaza where the caf{\'e} the russian woman told me is suposed to be located. I find it without any problem and finally go to my hotel. I take out my two laptops and turn on the Sony. I experiment with English Editor 2 for about half an hour before I go out for lunch. I take the Sony with me to continue exploring the BeOS system. I was about to leave the NBook in the room but I put it back in the bag with the Sony because I don't know if I'll be coming back to the hotel before I meet the russian woman. ~\\

    I choose an old restaurant in the center of GotaloNia, not very far from the hotel, to have lunch. I've been in this restaurant before for quite a few times and I like it's food very much. Never managed to remember it's name but I can always find it whenever I'm in GotaloNia. On the way I found a rather peculiar music shop with a lot of alternative music and old records. It's always an attraction to me. I end up buying two very old CDs; one by \textbf{Coil "Gold is the Metal"} and \textbf{Peter Murphy's "Dali's Car"}. I also bought Dust here in GotaloNia, the last time I was here. At the restaurant I manage to recognize two of the people working here. Both women and young. One of them, the waitress, also seems to recognize me and greets me in a very kind way. I order Spaghetti al Pesto and a glass of red Gotalonian wine. I just love the pasta they have here and the pesto sauce is my all time favorite. I take a long time eating since the food is so delicious. By the time I finish the main plate there's almost half a bottle of wine yet to be drunk. I sip the wine while testing other software on the Sony laptop. The afternoon is warm but not overwhelming. Not like in the desert. One of the times I was looking out the window besides me I spotted a woman that reminded me of the russian that lives here in GotaloNia. After staring at her for a few seconds, while she stared back I made a greeting gesture with my hand and then she began to walk towards the restaurant's door. It was really her. She seated at my table and ordered a big cortado and a small chocolate cake. She was just passing by and spotted me in the restaurant. Uncertain if it was really me she hesitaded for a while but when I greeted her she had no more doubts. Now she stares profoudly at the screen of the NBook. She asks a few questions about the WriteNow app and if it can save in RTF format. I explain that it can even produce some decently formed HTML. Ah, and it supports cyrilic encoding! She's absolutly salivating over the machine. She then asks me, partly serious partly joking, if I want to sell the laptop. I smile and say no, not yet, and maybe never. I tell her that if she loves so much the software she can install NeXTStep on a x86 laptop if she wants because Corto made a copy of the installer CD and it has the x86 version on it. She takes out of her bag a very small Sony Vaio [ the screen is only 10" or even less ] and wonders if the software will install correctly on that machine. Well, we have to try. And since we're close to the port we decide to check if Corto is in the yatch and if she can borrow the CD and try an install. ~\\

    Corto is hosing the deck of the yatch and greets us as we aproach the boat. Corto seems to know very well Natalia and goes inside to look for the NeXTStep install CD. She plugs-in the small Vaio and boots it up to a very rare version of \textbf{MacOSX for x86}! I never knew that version existed. She tells me this OS it's not supposed to be running outside of Apple's headquarters but she managed to get hold of a copy and has it installed on the Vaio as her main OS. It's an early beta version of MacOSX.1.4. How much did she pay for it? I ask. Nothing, someone working at Apple gave her a copy of the install CD as a birthday gift two or three years ago. It includes a WYSIWYG html editor and the browser is a beta version of OmniWeb 4.02. I tell her that this version of MacOSX is the dream of a million PC users out there. She smiles and tells me that she's aware of that. Of there's almost no software available for this version of OSX but she's quite happy with the email, net and writing apps. She doesn't need more she tells me. She also uses it mainly to write reports and other html stuff. She maintains a very minimalistic website with it. She writes about fashion shows and other irrelevant stuff, in her own words. ~\\

    Corto comes out with the CD and we begin the instalation of \textbf{NeXTStep}. We use an unused partition of 2 GBytes she has on the disk and the installer begins it's work. After 30 minutes or so the installer prompts for a reboot. The boot manager asks what OS we want to boot; MacOSX, BeOS5, Windows98 or NeXTStep. We choose the last option and after about one and a half minute the little Vaio has the default NeXTStep desktop. Everything seems to be working, even the sound card and the modem which we weren't expecting to see recognized by the OS. Corto has downloaded all the latest patches so that might be the reason why. We then install from another CD LaunchBar, OmniWeb and a few other image manipulating software. Everything installs as expected. ~\\

    We spend the rest of the afternoon in the yatch discussing politics, music, restaurants and a lot of other things. We drink martinis and coffee as we talk. The sun is almost setting on us and we three watch in bliss with the atmospheric music of Cocteau Twins in the background, playing lightly. It's always a magical moment, the setting of the sun, anywhere in the world. Natalia invites Corto and myself for a true GotaloNian dinner at her favorite restaurant, a place called \emph{Torre Ghotic}. What's their speciality, we ask. She spits out two words: fish cakes. I'm delighted, I haven't eaten fish cakes for quite a long time now. We accept her invitation and set up an hour for us to be there. I then leave the boat with Natalia and we call a cab. I leave the cab just a few meters from the hotel entrance and she continues the ride towards her house. I arrive at the room and turn on the hot water in the shower. ~\\

\vfill
\columnbreak

    My guess, right now, is that both Natalia and Corto work for the same organization and are trying, somehow, to recruit me. Or maybe they are willing to work with my organization. I guess, also, that I'll know exactly if this is true at tonight's dinner. I wonder what's their current mission. I wonder what kind of buisiness Corto has to do in Algeria. It's a small world, after all. Meanwhile, here I am pouring this into the report with my NBook. And then, out of the blue, a thought came to me. I must try someday a Java text editor. I wonder what it would feel like... I know Java apps tend to run very slowly but a simple text editor wouldn't be a problem for me. I'll have to check out the matrix for that. And as the train of thought progresses I remember the CD Natalia is supposed to be burning for me with that very exquisite and rare version of \textbf{OSX.1.4 for x86}. I have to install that on my sony. ~\\

    {\small dinner : the awake of the sleeping cell : no more wine, please, thank you} ~\\

    The dinner started really well with us 3 drinking a few liquor shots just to spice up the appetite. Matalia brought me a copy of that very rare MacOSX.1.4 for x86. Now my Sony will be able to run a basic MacOSX. I'm glad. Not just BeOS5, but OSX.1.4 also. I'm getting drunk with every passing minute. I'm enjoying every second that goes by. And in the meantime, before dinner arrives, I'll plug the sony to an outlet power on the restaurant wall and try to install OSX now. If I run into some kind of problem maybe Corto or Natalia will be able to help me out. The restaurant sound system is playing "\textbf{Computador}" by Von Magnet. Which only adds up to the experience. There's a lot of support here in GotaloNia for that group of performers. Another example of GotaloNian good taste. Of course there are plenty of other good examples like the food and drinks they serve here, for instance... ~\\

    The instalation of OSX.1.4 went really well. It took almost an hour but everything's working perfectly. It's a shame that no other apps besides OmniWeb and LaunchBar are available for this x86 version. But that's enough for me. I closed the sony screen and went back to finish my dinner. Natalia said that there are a few experimental applications for this version on the matrix. I just might try that, out of curiosity. But the beauty of an old \textbf{Aqua} [version X.1.4] is something to behold with great pleasure. ~\\

    I order one coca-cola bottle and have a large sip. I'm missing something with sugar. The red wine is also making me dizzy. I'll have a alchool break for an hour or so. Corto and Natalia continue with the wine, gladly. I start to talk about my plans to go to Algeria and ask Corto if he'll be there shortly like he mentioned to me earlier. He's planning to go there "in business" in one or two weeks time. He stresses the word "business" with a strange smile and says that we'll probably meet there. He'll email me when he gets there. Natacha also says that she'll be doing an article for a newspaper in GotaloNia and that she might need to go there too. She doesn't know when but she suspects it won't take too much time for that to happen. So it seems that we'll all meet there! Can't wait for that. Although I must be very carefull doing my business there. I don't want them to know what it is I'm really doing there. I guess they wouldn't aprove my line of business. But, then again, you never know, do you? ~\\

    While the sony is idle I notice the long forgotten \textbf{AquaIcons screensaver} of the first versions of OSX. It doesn't ship anymore with newer versions. But that happens with all operating systems, I guess. And sometimes I wonder about the state of this report... I'm doing it right now. Ah, OSX.1.4... ~\\

    {\small sshhhhh....} ~\\

    I'm on the roof of the hotel writing this on the NBook. I'm feeling lost but not that lonely . I'm wondering very hard where the hell Natacha is. I know she's here in Gotalonia. I just sense it. I may be wrong but, deep in me, I know I'm not. And that's enough for me to keep on dreaaming. I brought one cortado with me but it's almost cold. Still, it's delicious and it's another fuel for my dreams. I guess I should be sleeping by now. I'm tired and it's past three in the morning. I'm leaving GotaloNia tomorrow afternoon on my way to another mission in Algeria. Will it be my last one? Maybe I'm just sad because I'll be leaving this beautiful city. I wonder when will I return. Or, maybe I'm sad because I know nothing about Natacha for the last few years. ~\\

    What if she knew I was here and decided to call me? I wonder how I would react... Would I have the nerve to see her. Ah, I guess so. In fact, there's nothing else I'd like better to happen to me than to be able to see her. I don't even know if she's alive. But it's not very probable that she is dead, either. Maybe one of these days I'll send her an email, the problem is that I really don't know if she keeps the same email account, but I guess she does. Anyway, I really don't think she'll answer my message. I have to think about it in the next days... ~\\

%% \vfill
%% \columnbreak

    {\small the noise continues... from \emph{computador} to \emph{mezclador} } ~\\

    Corto and Natalia are now ordering the special spiritual drinks to start the night in an insane manner. I also order one very disturbing strong drink. It tastes like fire with a faint twist of sugar. The taste of this drink reminds me of an expression I heared a few years ago: "half awake, half asleep". It's a mysterious expression that never left my mind. It keeps popping up in my mind at the most unexpected times and situations. ~\\

    Suddenly Natalia starts to tell us about a plan she heared about something happening in \textbf{Algeria}. A big "event". And then she smiles as if she's waiting for a reaction from me or Corto. "We don't know anything, yet. But we'll know more in a few days". ~\\

    {\small noise fades into... cosmogonia by cybergypsies... but wait, or is it Gotalonia by the CyberGypsies?} ~\\

    The rest of the night kept on going fast and faster with each passing hour untill we ended up at Corto's yatch for the rest of dawn. Listening softly to some old music Corto had on the boat. I think it was some old vynil records he had the chance to save in CD format at somebody's house here in GotaloNia. Old Broadway songs. Quite nice for a wild dawn of endless conversations and laughter. And then suddenly, nothing else matters. No mission in Algeria, no nothing. It's seems like time itself had stopped and we are all living in some kind of fucked up dawn twilight and nothing around us is real. Everything seems nice and gentle. ~\\

    This morning I'm gonna catch the plane to Algeria. Then, when I get there I'll email my boss with the number of the hotel where I'm staying so he can phone me. I guess I'll have one or two days before going into action. The NBook is waiting to be packed in my backpack. It's the only hand baggage I'll carry on the plane. ~\\

%% \vfill
%% \columnbreak

    {\small the smell of Algeria [ part II ]} ~\\

    hope everything goes as planned... ~\\

    {\small new, warmer, air} ~\\

    Algeria is, as always, a very hot country in the spring/summer time. And this time is no exception either; the temperatures for the next three days are all above 40 degrees celcius. Welcome to the desert. ~\\

    When I rent a room in the hotel it's already 9PM. I set up the NBook and connect to the matrix. Omniweb nad I'm reading my email. Nothing new there. So I go out for a cup of coffee and to get the feel of the city. Tomorrow I'll have some contacts to make and to check the process of the mission. Fire up the MP3 reader and start listening to "\textbf{Another Green World}" by Brian Eno. Gets me in a mysterious mood. Never really knew why... ~\\

posted by rd @ 2:54 PM ~\\

\begin{center} \rule{0.45\textwidth}{0.001cm} \end{center}

{\large Myself, Natasha and Corto} ~\\

We will change the fucking world! ~\\

That's all we have to say at this time. ~\\

posted by rd @ 2:47 PM ~\\

\begin{center} \rule{0.45\textwidth}{0.001cm} \end{center}

\end{multicols*}

\clearpage

\end{document}
